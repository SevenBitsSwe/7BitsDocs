\documentclass[11pt]{article}

\usepackage[utf8]{inputenc}
\usepackage{tabularx}
\usepackage{hyperref}
\usepackage{array}  
\usepackage{graphicx} % Per inserire immagini (loghi)
\usepackage{geometry} % Per personalizzare i margini
\usepackage{fancyhdr} % Per gestire intestazioni e piè di pagina
\usepackage{tikz}
\usepackage{ragged2e}
\usepackage{anyfontsize}
\usepackage[table,xcdraw]{xcolor}
\usepackage{tabularx, etoolbox} % aggiungi etoolbox per condizioni% Load xcolor
\usepackage{eso-pic} % Per aggiungere elementi grafici su tutte le pagine

\graphicspath{{images/}}

\setcounter{secnumdepth}{4}
\setcounter{tocdepth}{4}

%cambio misure della pagina
\geometry{a4paper,left=25mm,right=25mm,top=25mm,bottom=25mm}
%ebdfc7
\definecolor{colorePie}{HTML}{ebdfc7}
\pagestyle{fancy}
\fancyhf{}
\renewcommand{\headrulewidth}{0.4pt}
\lhead{
    \parbox[c]{1cm}{\includegraphics[width=1.1cm]{Sevenbitslogo.png}}
}
\rhead{\textcolor[HTML]{9e978a}{ PIANO DI QUALIFICA v0.1.0}
}
\setlength{\headheight}{25pt}
\cfoot{\thepage}




\renewcommand*\contentsname{Indice}
\renewcommand{\listfigurename}{Elenco delle figure}

\begin{document}

% Pagina del titolo
\begin{titlepage}
    \setcounter{page}{0}
    \centering
    % Inserisci il logo del gruppo (modifica il percorso dell'immagine)
    \includegraphics[width=7.2cm]{Sevenbitslogo.png} \\[2cm] 
    
    % Titolo
     {\fontsize{40}{40}\bfseries Piano di Qualifica}\selectfont \\[3.9em]
    
    % Sottotitolo
    {\huge NearYou\\ \vspace{3mm }Smart custom advertising platform} \\[2.7em]
    
    % Email del gruppo
    {\large sevenbits.swe.unipd@gmail.com} \\[3em]
    
    % Spazio per il logo dell'università
    \hfill
    
        
    \AddToShipoutPictureBG{ % Imposta il triangolo con logo
        \ifnum\value{page}=0
        \begin{tikzpicture}[overlay]
        
            % Definisce un triangolo blu in basso a destra
            \fill[colorePie] 
                (current page.south east) -- ++(-9cm,0) -- ++(9cm,9cm);
            
            % Inserisce il logo all'interno del triangolo
            \node[anchor=south east, xshift=-0.3cm, yshift=0.3cm] at (current page.south east) {
                \includegraphics[width=4.5cm]{LogoUnipd.png}
            };
        \end{tikzpicture}
        \fi
    }
        
    

    \vfill % Aggiunge spazio verticale per centrare il contenuto
\end{titlepage}
\newpage
\clearpage
\setcounter{page}{1}



\centering\textbf{Registro modifiche}\\
\vspace{2mm}
\begin{tabular}{|l|l|l|l|l|l|}
\hline
\textbf{Versione} & \textbf{Data} & \textbf{Autore} & \textbf{Verificatore} & \textbf{Descrizione} \\
\hline
0.1.0 & 2024-11-21 & Uncas Peruzzi  & Federico Pivetta & Inizio redazione del documento\\
\hline
\end{tabular}
\newpage
\tableofcontents
\listoffigures %elenco delle figure sarà da usare per ogni immagine
%degli use cases , utilizzando begin figures, caption ecc..

\newpage
\begin{justify}
    

\section{Introduzione}


\subsection{Scopo del documento}

Il seguente documento ha l'obiettivo di garantire la qualità del prodotto e dei processi coinvolti nell'intero progetto. Al fine di assicurare che il prodotto soddisfi le qualità attese, il documento
verrà aggiornato nel tempo per riflettere eventuali modifiche, integrazioni e i risultati delle verifiche effettuate.


\subsection{Glossario}
Con l'intendo di evitare ambiguità interpretative del linguaggio utilizzato, viene fornito un Glossario che si occupa di esplicitare il significato dei termini che riguardano il contesto del progetto. I termini presenti nel glossario sono contrasegnati con una \textit{G} a pedice : Termine\(_G\).\\
Le definizioni sono presenti nell'apposito documento \textit{Glossario v1.0.0}


\subsection{Riferimenti}


\subsubsection{Riferimenti normativi}
\begin{itemize}
    \item[-] Norme di Progetto v1.0.0\\
    \textcolor{blue}{\texttt{\url{linkdamettere.com}}}
    
    \item[-] Regolamento del progetto didattico  \\
    \textcolor{blue}{\texttt{\url{https://www.math.unipd.it/~tullio/IS-1/2024/Dispense/PD1.pdf}}}
    
\end{itemize}
\subsubsection{Riferimenti informativi}
\begin{itemize}
    \item[-] Capitolato C4- NearYou - 
Smart custom advertising platform\\
    \textcolor{blue}{\texttt{\url{https://www.math.unipd.it/~tullio/IS-1/2024/Progetto/C4p.pdf}}}
    \item[-] Standard ISO/IEC 9126\\
    \textcolor{blue}{\texttt{\url{https://en.wikipedia.org/wiki/ISO/IEC_9126}}}
    \item[-] Standard ISO/IEC/IEEE 12207:1995\\
    \textcolor{blue}{\texttt{\url{https://www.math.unipd.it/~tullio/IS-1/2009/Approfondimenti/ISO_12207-1995.pdf}}}
    \item[-] Qualità di prodotto\\
    \textcolor{blue}{\texttt{\url{https://www.math.unipd.it/~tullio/IS-1/2024/Dispense/T07.pdf}}}
    \item[-] Qualità di processo\\
    \textcolor{blue}{\texttt{\url{https://www.math.unipd.it/~tullio/IS-1/2024/Dispense/T08.pdf}}}
    \item[-] Verifica e validazione\\
    \textcolor{blue}{\texttt{\url{https://www.math.unipd.it/~tullio/IS-1/2024/Dispense/T09.pdf}}}\\
    \textcolor{blue}{\texttt{\url{https://www.math.unipd.it/~tullio/IS-1/2024/Dispense/T10.pdf}}}\\
    \textcolor{blue}{\texttt{\url{https://www.math.unipd.it/~tullio/IS-1/2024/Dispense/T11.pdf}}}
    
    

\end{itemize}

\section{Obiettivi metrici di qualità}
\subsection{Qualità di processo}


\subsubsection{Processi Primari}
\paragraph{Analisi dei requisiti}\mbox{}\\
\paragraph{Progettazione}\mbox{}\\
\paragraph{Fornitura}\mbox{}\\
\paragraph{Codifica}\mbox{}\\


\subsubsection{Processi di Supporto}
\paragraph{Documentazione}\mbox{}\\
\paragraph{Verifica}\mbox{}\\
\paragraph{Gestione della qualità}\mbox{}\\


\subsubsection{Processi Organizzativi}
\paragraph{Pianificazione}\mbox{}\\

\subsection{Qualità di prodotto}
\subsubsection{Funzionalità}
\subsubsection{Affidabilità}
\subsubsection{Efficienza}
\subsubsection{Usabilità}
\subsubsection{Manutenibilità}
\subsubsection{Portabilità}



\section{Modalità di Testing}
\subsection{Test di unità}
\subsection{Test di sistema}
\subsection{Test di integarzione}
\subsection{Tesr di accettazione}


\section{Cruscotto di valutazione delle qualità}
\subsection{Qualità di processo - fornitura}
\subsection{Qualità di processo - documentazione}
\subsection{Qualità di processo - analisi dei requisiti}
\subsection{Qualità di processo - verifica}

\end{justify}
\end{document}