\documentclass[10pt]{article}

\usepackage[table,xcdraw]{xcolor}
\usepackage[utf8]{inputenc}
\usepackage{tabularx}
\usepackage{hyperref}
\usepackage{array}
\usepackage{graphicx} % Per inserire immagini (loghi)
\usepackage{geometry} % Per personalizzare i margini
\usepackage{fancyhdr} % Per gestire intestazioni e piè di pagina
\usepackage{tikz}
\usepackage{ragged2e}
\usepackage{anyfontsize}
\usepackage{tabularx, etoolbox}
\usepackage{eso-pic} % Per aggiungere elementi grafici su tutte le pagine
\usepackage{float}
\usepackage{longtable}

\newcommand\version{0.4.2} %aggiunta versione come variabile

%\graphicspath{{../images/}}
\graphicspath{{images/}}

\setcounter{secnumdepth}{4}
\setcounter{tocdepth}{4}

%cambio misure della pagina
\geometry{a4paper,left=25mm,right=25mm,top=25mm,bottom=25mm}
%ebdfc7
\definecolor{colorePie}{HTML}{ebdfc7}
\pagestyle{fancy}
\fancyhf{}
\renewcommand{\headrulewidth}{0.4pt}
\lhead{
    \parbox[c]{1cm}{\includegraphics[width=1.1cm]{Sevenbitslogo.png}}
}
\rhead{\textcolor[HTML]{9e978a}{ PIANO DI QUALIFICA v\version}
}
\setlength{\headheight}{25pt}
\cfoot{\thepage}

\renewcommand*\contentsname{Indice}
\renewcommand{\listfigurename}{Elenco delle figure}

\begin{document}

% Pagina del titolo
\begin{titlepage}
    \setcounter{page}{0}
    \centering
    % Inserisci il logo del gruppo (modifica il percorso dell'immagine)
    \includegraphics[width=7.2cm]{Sevenbitslogo.png} \\[2cm]

    % Titolo
     {\fontsize{40}{40}\bfseries Piano di Qualifica}\selectfont \\[3.9em]

    % Sottotitolo
    {\huge NearYou\\ \vspace{3mm }Smart custom advertising platform} \\[2.7em]

    % Email del gruppo
    {\large sevenbits.swe.unipd@gmail.com} \\[3em]

    % Spazio per il logo dell'università
    \hfill

    \AddToShipoutPictureBG{ % Imposta il triangolo con logo
        \ifnum\value{page}=0
        \begin{tikzpicture}[overlay]

            % Definisce un triangolo blu in basso a destra
            \fill[colorePie]
                (current page.south east) -- ++(-9cm,0) -- ++(9cm,9cm);

            % Inserisce il logo all'interno del triangolo
            \node[anchor=south east, xshift=-0.3cm, yshift=0.3cm] at (current page.south east) {
                \includegraphics[width=4.5cm]{LogoUnipd.png}
            };
        \end{tikzpicture}
        \fi
    }

\vfill % Aggiunge spazio verticale per centrare il contenuto
\end{titlepage}
\newpage
\clearpage
\setcounter{page}{1}

% Registro Modifiche
\begin{center}
 \textbf{Registro modifiche}\\   
\end{center}

\renewcommand{\arraystretch}{1.5}
\rowcolors{0}{gray!11}{white} % Aggiunge colore alternato alle righe

\begin{longtable}{|>{\centering\arraybackslash}m{1.5cm}|>{\centering\arraybackslash}m{2cm}|>{\centering\arraybackslash}m{2.5cm}|>{\centering\arraybackslash}m{2.5cm}|>{\centering\arraybackslash}m{5cm}|}
\hline
\textbf{Versione} & \textbf{Data} & \textbf{Autore} & \textbf{Verificatore} & \textbf{Descrizione}\\
\endhead
    \hline
    0.4.3 & 2025-01-10 & Leonardo Trolese & Uncas Peruzzi & Aggiunta dei test di integrazione\\
    \hline
    0.4.2 & 2025-01-10 & Federico Pivetta & Leonardo Trolese & Aggiunta dei test di sistema e dei test di accettazione\\
    \hline
    0.4.1 & 2025-01-08 & Riccardo Piva & Uncas Peruzzi & Refactor generale sezione qualità processo e qualità prodotto \\
    \hline
    0.4.0 & 2025-01-07 & Riccardo Piva & Uncas Peruzzi & Creazione cruscotto\\
    \hline
    0.3.3 & 2025-01-03 & Riccardo Piva & Uncas Peruzzi & Correzioni minori generali \\
    \hline
    0.3.2 & 2024-12-16 & Alfredo Rubino & Manuel Gusella & Aggiunta acronimi metriche e correzioni minori\\
    \hline
    0.3.1 & 2024-12-13 & Riccardo Piva & Alfredo Rubino & Correzione standard IEEE \\
    \hline
    0.3.0 & 2024-12-12 & Riccardo Piva & Alfredo Rubino & Arricchimento sezioni Qualità di processo, Qualità di prodotto e inizio redazione modalità testing \\
    \hline
    0.2.0 & 2024-12-06 & Manuel Gusella  & Alfredo Rubino & Inizio redazione sezione Qualità di prodotto \ref{qpd}\\
    \hline
    0.1.0 & 2024-11-21 & Uncas Peruzzi  & Federico Pivetta & Inizio redazione del documento\\
    \hline
\end{longtable}
\rowcolors{0}{}{} % Riporta le righe alla colorazione originale

\newpage
\tableofcontents
\listoffigures %elenco delle figure sarà da usare per ogni immagine
\listoftables %lista delle tabelle presenti nel documento
\newpage
\begin{justify}

\section{Introduzione}

\subsection{Scopo del documento}
Il seguente documento ha l'obiettivo di garantire la qualità del prodotto e dei processi coinvolti nell'intero progetto. Al fine di assicurare che il prodotto soddisfi le qualità attese, il documento
verrà aggiornato nel tempo per riflettere eventuali modifiche, integrazioni e i risultati delle verifiche effettuate.


\subsection{Glossario}
Con l'intento di evitare ambiguità interpretative del linguaggio utilizzato, viene fornito un Glossario che si occupa di esplicitare il significato dei termini che riguardano il contesto del progetto. I termini presenti nel glossario sono contrassegnati con una $_G$ a pedice : Termine$_G$.\\
Le definizioni sono presenti nell'apposito documento \textit{Glossario v1.0.0}


\subsection{Riferimenti}


\subsubsection{Riferimenti normativi}
\begin{itemize}
    \item[-] Norme di Progetto v1.0.0\\
    \textcolor{blue}{\texttt{\url{linkdamettere.com}}}

    \item[-] Regolamento del progetto didattico  \\
    \textcolor{blue}{\texttt{\url{https://www.math.unipd.it/~tullio/IS-1/2024/Dispense/PD1.pdf}}}

\end{itemize}
\subsubsection{Riferimenti informativi}
\begin{itemize}
    \item[-] Capitolato C4 - NearYou -
Smart custom advertising platform\\
    \textcolor{blue}{\texttt{\url{https://www.math.unipd.it/~tullio/IS-1/2024/Progetto/C4p.pdf}}}
    \item[-] Standard ISO/IEC 9126 \label{ISO 9126} \\
    \textcolor{blue}{\texttt{\url{https://en.wikipedia.org/wiki/ISO/IEC_9126}}}
    \item[-] Standard ISO/IEC/IEEE 12207:1995 \label{ISO 12207:1995}\\
    \textcolor{blue}{\texttt{\url{https://www.math.unipd.it/~tullio/IS-1/2009/Approfondimenti/ISO_12207-1995.pdf}}}
    \item[-] Qualità di prodotto\\
    \textcolor{blue}{\texttt{\url{https://www.math.unipd.it/~tullio/IS-1/2024/Dispense/T07.pdf}}}
    \item[-] Qualità di processo\\
    \textcolor{blue}{\texttt{\url{https://www.math.unipd.it/~tullio/IS-1/2024/Dispense/T08.pdf}}}
    \item[-] Verifica e validazione\\
    \textcolor{blue}{\texttt{\url{https://www.math.unipd.it/~tullio/IS-1/2024/Dispense/T09.pdf}}}\\
    \textcolor{blue}{\texttt{\url{https://www.math.unipd.it/~tullio/IS-1/2024/Dispense/T10.pdf}}}\\
    \textcolor{blue}{\texttt{\url{https://www.math.unipd.it/~tullio/IS-1/2024/Dispense/T11.pdf}}}
\end{itemize}
\newpage

\section{Obiettivi metrici di qualità}
Per far sì che un prodotto raggiunga uno standard qualitativo, è necessario definire delle metriche precise che permettano di monitorare e indicare il grado di qualità del prodotto e che quindi permettano di definire questo standard.
Queste metriche vengono definite nel documento \textit{Norme di Progetto v1.0.0}.
Questa sezione si occuperà di definire i parametri di queste metriche, queste metriche potranno essere accettabili o ottimali in base alla rigidità del parametro.\\

\subsection{Qualità di prodotto}\label{qpd}
La qualità di prodotto è intesa come valutazione del software. Più precisamente per la determinazione del grado di conformità alle attese.\\
Si rivolge l'attenzione su aspetti come Usabilità, Affidabilità e Manutenibilità, ma più in generale alla qualità esterna (funzionale) ed interna (strutturale) del prodotto software.\\
Quindi non basta che il software implementi le funzionalità volute dal proponente, ma le esegua secondo specifici standard di qualità.\\
In seguito sono presenti le metriche definite dallo \hyperref[ISO 9126]{standard ISO/IEC 9126} che il gruppo si impegna a soddisfare per la qualità del prodotto software.
\subsubsection{Funzionalità}
\begin{table}[H]
  \centering
\begin{tabular}{|c|c|c|c|}
  \hline
  \textbf{Metrica} & \textbf{Descrizione} & \textbf{Valore accettazione} & \textbf{Valore ideale}\\
  \hline
  MPD01 & Requisiti Obbligatori Soddisfatti & 100\% & 100\%\\
  \hline
  MPD02 & Requisiti Desiderabili Soddisfatti  & $\geq$0\% & 100\% \\
  \hline
  MPD03 & Requisiti Opzionali Soddisfatti & $\geq$0\% & 100\% \\
  \hline
\end{tabular}
\caption{Funzionalità - Qualità di prodotto}
\label{tab:funzionalità}
\end{table}

%%%%%%%%%%%%%%%%%%%%%%%%%%%%%%%%%%%%%%%%%%%%%%%%%%%%%%%%%%%%%%%%%%%%%%%%%%%%%%%%%%%%%%%%%%%%%%%%%%%%%%%%%%%%%%%%%%%%%%%%%%%%%%
%%% ATTENZIONE DA RIVEDERE NOMENCLATURA METRICHE, DATO IL GRANDE REFACTOR ANDRANNO MODIFICATE IN BASE ALL'ORDINE FINALE
%%%%%%%%%%%%%%%%%%%%%%%%%%%%%%%%%%%%%%%%%%%%%%%%%%%%%%%%%%%%%%%%%%%%%%%%%%%%%%%%%%%%%%%%%%%%%%%%%%%%%%%%%%%%%%%%%%%%%%%%%%%%%%

\subsubsection{Affidabilità}
\begin{table}[H]
  \centering
\begin{tabular}{|c|c|c|c|}
  \hline
  \textbf{Metrica} & \textbf{Descrizione} & \textbf{Valore accettazione} & \textbf{Valore ideale}\\
  % \hline
  % MPD04 & Code coverage & $\geq$80\% & 100\% \\
  %% Code coverage rimossa perché è una metrica generale
  %% Quindi a parere mio è preferibile tenere quelle più specifiche
  %% E lasciare la code coverage per la sezione per i processi
  \hline
  MPD05 & Statement coverage & $\geq$80\% & 100\% \\
  \hline
  MPD06 & Branch coverage & $\geq$80\% & 100\% \\
  \hline
  MPD07 & Condition coverage & $\geq$80\% & 100\% \\
  \hline
  MPD08 & Indice Gulpease & $\geq$40\% & $\geq$60\% \\
  \hline
  MPD09 & Correttezza ortografica & 0 errori & 0 errori \\
  \hline
\end{tabular}
\caption{Affidabilità - Qualità di prodotto}
\label{tab:affidabilità}
\end{table}

\subsubsection{Efficienza}
\begin{table}[H]
  \centering
\begin{tabular}{|c|c|c|c|}
  \hline
  \textbf{Metrica} & \textbf{Descrizione} & \textbf{Valore accettazione} & \textbf{Valore ideale}\\
  \hline
  MPD10 & Utilizzo risorse & $\geq$80\% & 100\% \\
  \hline
  MPD11 & Tempo medio di risposta & $\leq$10 secondi & $\leq$4 secondi \\
  \hline
\end{tabular}
\caption{Efficienza - Qualità di prodotto}
\label{tab:efficienza}
\end{table}

\subsubsection{Usabilità}
\begin{table}[H]
  \centering
\begin{tabular}{|c|c|c|c|}
  \hline
  \textbf{Metrica} & \textbf{Descrizione} & \textbf{Valore accettazione} & \textbf{Valore ideale}\\
  \hline
  MPD12 & Facilità di utilizzo & $\leq$5 errori nell'interagire & $\leq$3 errori nell'interagire  \\
  \hline
  MPD13 & Tempo medio di apprendimento & $\leq$15 secondi & $\leq$5 secondi \\
  \hline
\end{tabular}
\caption{Usabilità - Qualità di prodotto}
\label{tab:usabilità}
\end{table}

\subsubsection{Manutenibilità}
\begin{table}[H]
  \centering
\begin{tabular}{|c|c|c|c|}
  \hline
  \textbf{Metrica} & \textbf{Descrizione} & \textbf{Valore accettazione} & \textbf{Valore ideale}\\
  % \hline
  % MPD14 & Complessità ciclomatica & $\leq$20 & $\;leq$10 \\
  %% Complessità ciclomatica è stata rimossa perché non è stata ritenuta utile
  % \hline
  % MPD15 & Coefficiente di accoppiamento fra classi & $\leq$10 & $\leq$5 \\
  \hline
  MPD16 & Linee di codice per metodo & $\leq$50 & $\leq$25 \\
  \hline
  MPD17 & Parametri per metodo & $\leq$7 & $\leq$4 \\
  % \hline
  % MPD18 & Attributi per classe & $\leq$7 & $\leq$5 \\
  %% Dubbio sulle due metriche riguardanti le classi se tenere o meno
  \hline
\end{tabular}
\caption{Manutenibilità - Qualità di prodotto}
\label{tab:manutenibilità}
\end{table}

\subsubsection{Portabilità}
\begin{table}[H]
  \centering
\begin{tabular}{|c|c|c|c|}
  \hline
  \textbf{Metrica} & \textbf{Descrizione} & \textbf{Valore accettazione} & \textbf{Valore ideale}\\
  \hline
  MPD19 & Versioni browser supportati & $\geq$80\% & 100\% \\
  \hline
\end{tabular}
\caption{Portabilità - Qualità di prodotto}
\label{tab:portabilità}
\end{table}


\subsection{Qualità di processo}
La qualità di processo è intesa come valutazione delle attività svolte per la realizzazione del prodotto.\\
Seguendo delle buone pratiche e delle linee guida nello sviluppo software, si può garantire che il prodotto finale avrà rispettato a sua volta degli standard qualitativi rendendolo così un prodotto di qualità.\\
Qui sotto divideremo le metriche di qualità di processo seguendo lo \hyperref[ISO 12207:1995]{standard ISO/IEC 12207:1995} in tre categorie: Processi primari, Processi di supporto e Processi organizzativi.\\
\subsubsection{Processi Primari}
I processi primari si possono dividere in parti primarie e una parte primaria è quella che inizia o esegue lo sviluppo, l'operazione o la manutenzione di prodotti software.\\

\newpage

%%
%% Questo volendo si puo integrare in sviluppo
%%

% \paragraph{Analisi dei requisiti}\mbox{}\\
% L'analisi dei requisiti è il processo che si occupa di raccogliere, analizzare e definire i requisiti del sistema che si intende sviluppare.\\
% Serve per capire le esigenze degli stakeholder e tradurle in requisiti dettagliati e comprensibili per il team di sviluppo.\\
% \begin{table}[H]
%   \centering
% \begin{tabular}{|c|c|c|c|}
%   \hline
%   \textbf{Metrica} & \textbf{Descrizione} & \textbf{Valore accettazione} & \textbf{Valore ideale}\\
%   \hline
%   MPC01 & Requisiti Obbligatori Soddisfatti (ROS) & 100\% & 100\%\\
%   \hline
%   MPC02 & Requisiti Desiderabili Soddisfatti (RDS) & $\geq$0\% & 100\% \\
%   \hline
%   MPC03 & Requisiti Opzionali Soddisfatti (RPS) & $\geq$0\% & 100\% \\
%   \hline
% \end{tabular}
% \caption{Processi primari - Analisi dei requisiti}
% \label{tab:analisi dei requisiti}
% \end{table}

%%
%% Questa sezione la vedo piú da qualità di prodotto e in piú è poco comune come processo
%%

% \paragraph{Progettazione}\mbox{}\\
% La progettazione è il processo che si occupa di definire l'architettura del sistema.\\
% Serve per definire come il sistema sarà implementato e come le varie parti del sistema interagiranno tra di loro.\\
% \begin{table}[H]
%   \centering
% \begin{tabular}{|c|c|c|c|}
%   \hline
%   \textbf{Metrica} & \textbf{Descrizione} & \textbf{Valore accettazione} & \textbf{Valore ideale}\\
%   % \hline
%   % MPC04 & Coefficiente di accoppiamento fra classi & $\leq$10 & $\leq$5 \\
%   \hline
%   MPC05 & Facilità di utilizzo & $\leq$5 errori nell'interagire & $\leq$3 errori nell'interagire  \\
%   \hline
%   MPC06 & Tempo medio di apprendimento & $\leq$15 secondi & $\leq$5 secondi \\
%   \hline
% \end{tabular}
% \caption{Processi primari - Progettazione}
% \label{tab:progettazione}
% \end{table}

\paragraph{Fornitura}\mbox{}\\
La fornitura è il processo che si occupa di consegnare il prodotto software al cliente.\\
Serve per garantire che il prodotto soddisfi i requisiti di tempi e costi definiti con il cliente.\\
\begin{table}[H]
  \centering
\begin{tabular}{|p{1.5cm}|p{5cm}|p{4cm}|p{5cm}|}
  \hline
  \textbf{Metrica} & \textbf{Descrizione} & \textbf{Valore accettazione} & \textbf{Valore ideale}\\
  \hline
  MPC07 & Estimated at completion (EAC) & \textpm5\% rispetto al preventivo & Come dichiarato da preventivo\\
  \hline
  MPC08 & Estimate to complete (ETC) & $\geq$0\% & $\leq$EAC\(_G\) \\
  \hline
  MPC09 & Actual cost (AC) & $\geq$0 & $\leq$EAC\(_G\) \\
  \hline
  MPC10 & Earned value (EV) & $\geq$0 & $\leq$EAC\(_G\) \\
  \hline
  MPC11 & Planned value (PV) & $\geq$0 & $\leq$ Budget at completion (BAC)\(_G\) \\
  %% Queste metriche qui sotto arrivano dalla sezione di pianificazione sotto processi organizzativi
  \hline
  MPC28 & Schedule variance (SV) & $\geq$ -10\% & $\geq$0\% \\
  \hline
  MPC29 & Cost variance (CV) & $\geq$ -10\% & $\geq$0\% \\
  \hline
  %% Queste due sotto sono nuove, non sono ancora state scritte in norme di progetto
  MPC30 & Cost Performance Index (CPI) & \textpm10\% & 0\% \\
  \hline
  %% Questo può essere messo in sviluppo
  % MPC31 & Requirements Stability Index (RSI) & $\geq$ 80\% & 100\% \\
  % \hline
\end{tabular}
\caption{Processi primari - Fornitura}
\label{tab:fornitura}
\end{table}

\newpage


%%
%% Questa non so se tenerla dopotutto è più parte della qualità di prodotto
%%

% \paragraph{Codifica}\mbox{}\\
% La codifica è il processo riguardante la scrittura del codice del prodotto software.\\
% Serve per implementare le funzionalità del software e garantire che il software rispetti gli standard di qualità.\\
% \begin{table}[H]
%   \centering
% \begin{tabular}{|c|c|c|c|}
%   \hline
%   \textbf{Metrica} & \textbf{Descrizione} & \textbf{Valore accettazione} & \textbf{Valore ideale}\\
%   % Questa era stata rimossa a priori perché poco utile
%   % \hline
%   % MPC12 & Complessità ciclomatica & $\leq$20 & $\leq$10 \\
%   \hline
%   MPC13 & Parametri per metodo & $\leq$7 & $\leq$4 \\
%   % \hline
%   % MPC14 & Attributi per classe & $\leq$7 & $\leq$5 \\
%   \hline
%   MPC15 & Linee di codice per metodo & $\leq$50 & $\leq$25 \\
%   \hline
%   MPC16 & Tempo medio di risposta & $\leq$10 secondi & $\leq$4 secondi \\
%   \hline
%   MPC17 & Versioni browser supportati & $\geq$80\% & 100\% \\
%   \hline
% \end{tabular}
% \caption{Processi primari - Codifica}
% \label{tab:codifica}
% \end{table}

%%
%% Un eventuale alternativa è la sezione sviluppo, molto simile
%%

\paragraph{Sviluppo}\mbox{}\\
Lo sviluppoè  il processo riguardante la scrittura del codice del prodotto software.\\
Questa metrica serve a garantire che il software rispetti le richieste del cliente e che la codifica avvenga in modo efficiente\\
\begin{table}[H]
  \centering
\begin{tabular}{|c|c|c|c|}
  \hline
  \textbf{Metrica} & \textbf{Descrizione} & \textbf{Valore accettazione} & \textbf{Valore ideale}\\
  \hline
  MPC01 & Requisiti Obbligatori Soddisfatti (ROS) & 100\% & 100\%\\
  \hline
  %% Questi due sotto potrebbero essere messi peró sono poco comuni, quello essenziale è quello sopra
  %% È essenziale se si mette sviluppo altrimenti se si mette Analisi dei requisiti diventano essenziali tutti e tre a mio parere
  % MPC02 & Requisiti Desiderabili Soddisfatti (RDS) & $\geq$0\% & 100\% \\
  % \hline
  % MPC03 & Requisiti Opzionali Soddisfatti (RPS) & $\geq$0\% & 100\% \\
  % \hline
  MPC31 & Requirements Stability Index (RSI) & $\geq$ 80\% & 100\% \\
  \hline
\end{tabular}
\caption{Processi primari - Codifica}
\label{tab:codifica}
\end{table}


\subsubsection{Processi di Supporto}
Un processo di supporto è un processo che supporta un altro processo come parte integrante con uno scopo distinto e contribuisce al successo e alla qualità del progetto software.\\
Un processo di supporto è impiegato ed eseguito, se necessario, da un altro processo.\\
\paragraph{Documentazione}\mbox{}\\
La documentazione è essenziale per la comprensione del prodotto e per la sua manutenzione.\\
Di conseguenza è essenziale che questa sia chiara, comprensibile e corretta.\\
\begin{table}[H]
  \centering
\begin{tabular}{|c|c|c|c|}
  \hline
  \textbf{Metrica} & \textbf{Descrizione} & \textbf{Valore accettazione} & \textbf{Valore ideale}\\
  \hline
  MPC18 & Indice Gulpease & $\geq$40\% & $\geq$60\% \\
  \hline
  MPC19 & Correttezza ortografica & 0 errori & 0 errori \\
  \hline
\end{tabular}
\caption{Processi di supporto - Documentazione}
\label{tab:documentazione}
\end{table}

\paragraph{Verifica}\mbox{}\\
La verifica serve a garantire che il prodotto software sia conforme alle specifiche e non contenga errori.\\
\begin{table}[H]
  \centering
\begin{tabular}{|c|c|c|c|}
  \hline
  \textbf{Metrica} & \textbf{Descrizione} & \textbf{Valore accettazione} & \textbf{Valore ideale}\\
  \hline
  MPC20 & Code coverage & $\geq$80\% & 100\% \\
  % \hline
  % MPC21 & Statement coverage & $\geq$80\% & 100\% \\
  % \hline
  % MPC22 & Branch coverage & $\geq$80\% & 100\% \\
  % \hline
  % MPC23 & Condition coverage & $\geq$80\% & 100\% \\
  \hline
  % Questo qui sopra sono state raggruppate in code coverage
  MPC24 & Passed test cases percentage & $\geq$80\% & 100\% \\
  \hline
\end{tabular}
\caption{Processi di supporto - Verifica}
\label{tab:verifica}
\end{table}

\newpage

%%
%% Questa si potrebbe inglobare sulla sezione sopra ma è separata perché più corretto
%%

\paragraph{Gestione della qualità}\mbox{}\\
La gestione della qualità è necessaria per garantire che tutte le metriche di qualità vengano effettivamente soddisfatte.\\
\begin{table}[H]
  \centering
\begin{tabular}{|c|c|c|c|}
  \hline
  \textbf{Metrica} & \textbf{Descrizione} & \textbf{Valore accettazione} & \textbf{Valore ideale}\\
  \hline
  MPC25 & Metriche di qualità soddisfate & $\geq$85\% & 100\% \\
  \hline
\end{tabular}
\caption{Processi di supporto - Gestione della qualità}
\label{tab:gestione della qualità}
\end{table}

\subsubsection{Processi Organizzativi}
I processi organizzativi servono per creare un sottostruttura per il ciclo di vita e
per garantire che i processi principali e i loro processi di supporto siano ben strutturati e vengano continuamente migliorati.\\
\paragraph{Gestione dei processi}\mbox{}\\
La gestione dei processi indica come vengono gestiti i processi all'interno del progetto.\\
\begin{table}[H]
  \centering
\begin{tabular}{|c|c|c|c|}
  \hline
  \textbf{Metrica} & \textbf{Descrizione} & \textbf{Valore accettazione} & \textbf{Valore ideale}\\
  \hline
  MPC26 & Rischi non previsti & $\leq$3 & 0 \\
  \hline
  MPC27 & Efficienza temporale (ET) & $\leq$3 & $\leq$1 \\
  \hline
\end{tabular}
\caption{Processi organizzativi - Gestione dei processi}
\label{tab:gestione dei processi}
\end{table}

% I componenti di questa sezione sono stati spostati fra i processi primari sotto fornitura e sviluppo
% \paragraph{Pianificazione}\mbox{}\\
% La pianificazione è un indispensabile per far si che che i parametri rientrino siano i più vicini possibili a quelli del preventivo.\\
% \begin{table}[H]
%   \centering
% \begin{tabular}{|c|c|c|c|}
%   \hline
%   \textbf{Metrica} & \textbf{Descrizione} & \textbf{Valore accettazione} & \textbf{Valore ideale}\\
%   \hline
%   MPC28 & Schedule variance (SV) & $\geq$ -10\% & $\geq$0\% \\
%   \hline
%   MPC29 & Cost variance (CV) & $\geq$ -10\% & $\geq$0\% \\
%   \hline
%   %% Queste due sotto sono nuove, non sono ancora state scritte in norme di progetto
%   MPC30 & Cost Performance Index (CPI) & \textpm10\% & 0\% \\
%   \hline
%   MPC31 & Requirements Stability Index (RSI) & $\geq$ 80\% & 100\% \\
%   \hline
% \end{tabular}
% \caption{Processi organizzativi - Pianificazione}
% \label{tab:pianificazione}
% \end{table}

\section{Modalità di Testing}
Qui sotto sono elencati i vari test che vengono eseguiti automaticamente sul prodotto software.\\
Questo serve a garantire che il prodotto soddisfi i requisiti e le aspettative indicate nel documento \textit{Analisi dei requisiti v1.0.0}.\\
I test sono divisi in quattro categorie: Test di unità, Test di sistema, Test di integrazione e Test di accettazione.\\
E per indicare lo stato come indicato in \textit{Norme di progetto v1.0.0} vengono utilizzate le seguenti abbreviazioni:
\begin{itemize}
\item \textbf{P}: Passato
\item \textbf{NP}: Non Passato
\item \textbf{NI}: Non Implementato
\end{itemize}

\subsection{Test di unità}
I test di unità servono a verificare che ogni singola unità del software funzioni correttamente.\\

\begin{longtable}{|>{\centering\arraybackslash}m{2cm}|>{\centering\arraybackslash}m{7cm}|>{\centering\arraybackslash}m{2cm}|}
\hline
\textbf{Codice} & \textbf{Descrizione} & \textbf{Stato}\\
\endhead
\hline
 & & \\
\hline
\caption{Test di unità}\\
\end{longtable}

\subsection{Test di sistema} % Manca TS relativo al requisito funzionale UC1.2
I test di sistema servono a verificare la completa copertura dei requisiti concordati nel documento Analisi dei Requisiti.\\

\begin{longtable}{|>{\centering\arraybackslash}m{2cm}|>{\centering\arraybackslash}m{7cm}|>{\centering\arraybackslash}m{2cm}|>{\centering\arraybackslash}m{2cm}|}
\hline
\textbf{Codice} & \textbf{Descrizione} & \textbf{Requisito} & \textbf{Stato}\\
\endhead
\hline
TS1 & Verificare che l'utente privilegiato possa visualizzare la dashboard$_G$ composta da una mappa interattiva con i vari marker e punti di interesse su di essa. & RF01 & NI\\
\hline
TS2 & Verificare che l'utente privilegiato possa visualizzare una dashboard$_G$ che rappresenti i vari percorsi effettuati in tempo reale dagli utenti presenti nel sistema. & RF02 & NI\\
\hline
TS3 & Verificare che l'utente privilegiato possa visualizzare una dashboard$_G$ relativa ad un singolo utente quando seleziona un marker. & RF03 & NI\\
\hline
TS4 & Verificare che l'utente privilegiato possa visualizzare i dettagli del marker riguardante una singola posizione di un utente nella rispettiva dashboard. & RF04 & NI\\
\hline
TS5 & Verificare che l'utente privilegiato possa visualizzare tutti i punti di interesse riconosciuti dal sistema. & RF05 & NI\\
\hline
TS6 & Verificare che l'utente privilegiato possa visualizzare l'area di influenza di un punto di interesse selezionato. & RF06 & NI\\
\hline
TS7 & Verificare che l'utente privilegiato possa visualizzare le informazioni dettagliate di un punto di interesse quando selezionato. & RF07 & NI\\
\hline
TS8 & Verificare che l'utente privilegiato possa visualizzare gli annunci pubblicitari provenienti da un determinato punto di interesse. & RF08 & NI\\
\hline
TS9 & Verificare che l'utente privilegiato possa visualizzare i dettagli dell'annuncio generato. & RF09 & NI\\
\hline
TS10 & Verificare che l'utente possa visualizzare l'annuncio pubblicitario proveniente dal punto di interesse situato nell'area che sta attraversando. & RF10 & NI\\
\hline
TS11 & Verificare che l'utente privilegiato possa effettuare l'accesso per visualizzare la dashboard$_G$. & RF11 & NI\\
\hline
TS12 & Verificare che l'utente privilegiato possa visualizzare un messaggio di errore nel caso le credenziali inserite durante l'accesso non siano riconosciute. & RF12 & NI\\
\hline
TS13 & Verificare che l'utente privilegiato possa visualizzare una tabella contenente le informazioni dei singoli PoI e la quantità di messaggi inviati nel mese. & RF13 & NI\\
\hline
TS14 & Verificare che il sensore sia in grado di trasmettere i dati rilevati in tempo reale al sistema di Stream Processing. & RF14 & NI\\
\hline
TS15 & Verificare che i dati ricevuti dal sensore siano filtrati e validati dal sistema di Stream Processing. & RF15 & NI\\
\hline
TS16 & Verificare che i dati validati dal sistema di Stream Processing siano elaborati tramite un servizio LLM. & RF16 & NI\\
\hline
TS17 & Verificare che i dati elaborati dal sistema di Stream Processing siano storicizzati su un database adeguato. & RF17 & NI\\
\hline
TS18 & Verificare che il servizio LLM sia in grado di selezionare il punto di interesse più rilevante per l'utente in base alla profilazione e alla posizione in tempo reale. & RF18 & NI\\
\hline
TS19 & Verificare che il servizio LLM sia in grado di generare un messaggio custom per l'utente in base al suo profilo e al punto di interesse selezionato in tempo reale. & RF19 & NI\\
\hline
TS20 & Verificare che il servizio LLM sia in grado di omettere la generazione di un messaggio custom per l'utente nel caso non sia presente alcun punto di interesse adatto per la specifica rilevazione. & RF20 & NI\\
\hline
\caption{Test di sistema}\\
\end{longtable}

\subsection{Test di integrazione}
I test di integrazione servono a verificare che le componenti del sistema si integrino correttamente e in maniera efficace. L'obiettivo dei test
è identificare eventuali problemi di interoperabilità e integrazione fra le componenti del software.\\

\begin{longtable}{|>{\centering\arraybackslash}m{2cm}|>{\centering\arraybackslash}m{7cm}|>{\centering\arraybackslash}m{2cm}|}
\hline
\textbf{Codice} & \textbf{Descrizione} & \textbf{Stato}\\
\endhead
\hline
TI1 & Verificare che i dati simulati vengano correttamente pubblicati sul broker di messaggi & NI\\
\hline
TI2 & Assicurarsi che il modulo di stream processing elabori correttamente i dati ricevuti dal broker e li invii al motore di generative AI & NI\\
\hline
TI3 & Verificare che il motore LLM generi messaggi pubblicitari coerenti e contestualizzati in base ai dati forniti & NI\\
\hline
TI4 & Controllare che i messaggi generati vengano correttamente salvati nella piattaforma di storage & NI\\
\hline
TI5 & Controllare che i dati posizionali generati vengano correttamente salvati nella piattaforma di storage & NI\\
\hline
TI6 & Verificare che i dati geospaziali vengano aggiornati in tempo reale sulla dashboard contente la mappa & NI\\
\hline
TI7 & Assicurarsi che i messaggi generati dal motore LLM siano correttamente visualizzati nella dashboard & NI\\
\hline
\caption{Test di integrazione}\\
\end{longtable}

\subsection{Test di accettazione}
I test di accettazione sono finalizzati a verificare che tutte le esigenze concordate con il proponente siano soddisfatte, e di conseguenza saranno
svolti al termine del progetto dai membri del gruppo in coordinazione con i componenti dell'azienda SyncLab.\\

\begin{longtable}{|>{\centering\arraybackslash}m{2cm}|>{\centering\arraybackslash}m{7cm}|>{\centering\arraybackslash}m{2cm}|}
\hline
\textbf{Codice} & \textbf{Descrizione} & \textbf{Stato}\\
\endhead
\hline
TA1 & Verificare che l'utente privilegiato possa usufruire dell'applicazione solamente a seguito di un'autenticazione avvenuta con successo. & NI \\
\hline
TA2 & Verificare che, una volta effettuato l'accesso, l'utente privilegiato possa visualizzare una dashboard$_G$ contenente una mappa che mostri i PoI, i vari utenti presenti nel sistema con i percorsi effettuati in tempo reale, e l'ultimo annuncio generato per ogni utente. & NI \\
\hline
TA3 & Verificare che, una volta effettuato l'accesso, l'utente privilegiato possa visualizzare la dashboard$_G$ generale e, selezionando uno specifico utente, accedere ad una dashboard dedicata con informazioni dettagliate su tale utente, tra cui nome, cognome, email, genere, data di nascita e stato civile. & NI \\
\hline
TA4 & Verificare che, una volta effettuato l'accesso, l'utente privilegiato possa visualizzare la dashboard$_G$ generale e, selezionando un PoI, accedere alle informazioni dettagliate relative al punto di interesse, tra cui nome, posizione, indirizzo, tipologia e descrizione. & NI \\
\hline
TA5 & Verificare che, una volta effettuato l'accesso, l'utente privilegiato possa visualizzare la dashboard$_G$ generale e, selezionando un annuncio, accedere ai dettagli specifici di tale annuncio. & NI \\
\hline
TA6 & Verificare che, una volta effettuato l'accesso, l'utente privilegiato possa visualizzare una tabella contenente le informazioni relative ai singoli PoI, inclusa la quantità di messaggi inviati nel mese corrente per ognuno di essi. & NI \\
\hline
TA7 & Verificare che il sensore, una volta connesso al sistema, possa trasmettere i dati rilevati in tempo reale al sistema di stream processing. & NI \\
\hline
TA8 & Verificare che il sistema di stream processing, una volta ricevuti i dati dal sensore, possa filtrare e validare tali dati e, se necessario, elaborarli tramite un modello LLM, per poi storicizzarli in un database adeguato. & NI \\
\hline
TA9 & Verificare che il servizio LLM possa selezionare il punto di interesse (PoI) più rilevante per l'utente in base alla sua profilazione e posizione in tempo reale e, di conseguenza, generare un messaggio personalizzato utilizzando sia i dati del PoI scelto sia quelli dell'utente. & NI \\
\hline
\caption{Test di accettazione}\\
\end{longtable}


\section{Cruscotto di valutazione delle qualità}
\subsection{Qualità di processo - fornitura}
\subsubsection{MPC01 - Estimated at completion (EAC)}
% Inserire grafico
% completare con valori effettivi e spiegazione più accurata
\textbf{RTB}: Il grafico mostra l'andamento del costo totale del progetto rispetto al budget preventivato. Questo valore è quindi influenzato dalla quantità di lavoro del team e dai ruoli ricoperti ogni periodo. Quindi in certi periodi la stima risulta in linea con il budget preventivato mentre in altri periodi possiamo essere sopra di molto o sotto di molto. Ad esempio nel periodo 2 e 3 siamo sopra il budget preventivato, questo è dovuto al fatto che in questi periodi il team ha lavorato di più rispetto a quanto preventivato. Mentre nel periodo 5 siamo sotto il budget preventivato, questo è dovuto al fatto che in questo periodo il team ha lavorato di meno rispetto a quanto preventivato a causa del periodo natalizio.\\
\subsubsection{MPC01 - Planned Value (PV) \& MPC01 - Earned Value (EV)}
% Inserire grafico
% Controllare testo (anche se mi pare ok)
\textbf{RTB}: Il grafico mostra che le curve del valore guadagnato (Earned Value) e del valore pianificato (Planned Value) si sovrappongono, questo indica che il lavoro effetivamente svolto è conforme alla pianificazione. Questa sovrapposizione dimostra un avanzamento positivo rispetto alla pianificazione del progetto.
\subsubsection{MPC01 - Actual Cost (AC) \& MPC01 - Estimate to Complete (ETC)}
% Inserire grafico
% Rivedere testo se piace
\textbf{RTB}: Il grafico rappresenta l'Actual Cost (AC), ovvero i costi sostenuti per portare il progetto al suo stato corrente, e l'Estimate to Complete (ETCG), cioè la stima del costo da sostenere per completare il progetto durante i vari periodi. \\
Ovviamente L'ETC tende a diminuire, dato che il progetto si avvicina alla sua conclusione, mentre l'AC mostra una crescita proporzionale alla velocità con cui l'ETC decresce.
\subsection{Qualità di processo - Pianificazione}
\subsubsection{MPC01 - Cost Variance (CV) \& MPC01 - Schedule Variance (SV)}
% Inserire grafico
% Rivedere testo in generale e periodi discrepanze tempo e costi
\textbf{RTB}: Il grafico rappresenta la differenza in percentuale tra Cost Variance (CV) e la Schedule Variance (SV), che a loro volta rappresentano rispettivamente: la differenza tra il valore guadagnato (EV) e i costi sostenuti (AC) e la differenza tra il valore guadagnato (EV) e il valore pianificato (PV). La Budget Variance non è molto stabile, abbiamo sforato anche se di poco il limite accettabile apparte nel primo periodo e nel periodo pre-natalizio dove abbiamo avuto un valore molto vicino a zero. La diminuzione della Budget Variance è dovuta ai ruoli che sono stati svolti in questi due periodi che erano meno costosi. Per quanto riguarda la Schedule Variance risulta anch'essa altalenante questo è dovuto al fatto che inizialmente il team ha impiegato parecchio tempo nell'apprendimento delle tecnologie e in alcuni periodi più liberi il team ha lavorato di più recuperando così il tempo perso che dopo andava a perdere di nuovo dopo gli incontri con la proponente dove si presentavano eventuali cambiamenti da implementare il che portava a un rallentamento del lavoro. Anche se con la continuazione del progetto il team è stato capace di migliorare la Schedule Variance, mantenendola al di sotto del limite accettabile.
\subsection{Qualità di processo - Gestione dei processi}
\subsubsection{MPC01 - Rischi non previsti}
% Inserire grafico
% assolutamente da rivedere aggiungendo gli effettivi rischi incontrati e il loro impatto
\textbf{RTB}: Nel corso del progetto, sono stati riscontrati alcuni rischi non previsti. Tuttavia, il team è riuscito a gestirli in modo efficace, riducendo al minimo il loro impatto sul progetto.
\subsubsection{MPC01 - Efficienza temporale}
% Inserire grafico
\textbf{RTB}: La metrica dell'efficienza temporale oscilla molto nel corso del progetto. Questo è dovuto alla necessità del gruppo di apprendere le nuove tecnologie, non solo legate al capitolato. Anche se non abbiamo scartato del tutto il nostro way of working, ha ricevuto comunque molte revisioni soprattutto nel periodo iniziale. Tuttavia, il team negli ultimi sprint, dopo che la situazione iniziale si è stabilizzata, è stato in grado di mantenere l'efficienza temporale al di sotto del limite accettabile.
\subsection{Qualità di processo - documentazione}
\subsubsection{MPC01 - Errori ortografici}
% Inserire grafico
\textbf{RTB}: Inizialmente, la documentazione presentava alcuni errori ortografici. Tuttavia, con il tempo, il team ha migliorato il processo di verifica e correzione dei documenti, riducendo il numero di errori ortografici drasticamente.
\subsubsection{MPC01 - Indice Gulpease}
% Inserire grafico
\textbf{RTB}: Inizialmente, l'indice Gulpease superava già il limite accettabile. Tuttavia, con il tempo, il team ha imparato a rispettarlo e a mantenerlo costante, migliorandone di poco a poco il punteggio. Questo è stato ottenuto rendendo tutti i documenti più leggibili, limitando l'uso di termini tecnici non necessari, soprattutto nei documenti più specifici o tecnici.
\subsection{Qualità di processo - Gestione della qualità}
\subsubsection{MPC01 - Metriche di qualità soddisfatte}
% Inserire grafico
\textbf{RTB}: Sul subito tutte le metriche inizialmente stabilite venivano applicate, anche se magari non perfettamente. Grazie al loro numero limitato il team è riuscito ad applicarle ma data l'inesperienza all'inizio ci sono state delle difficoltà. Con il tempo però, l'aggiunta di ulteriori metriche (alcune delle quali sconosciute ai membri del team) ha aumentato ulteriormente la difficoltà. Ma dopo un primo periodo di apprendimento e adeguamento a queste metriche abbiamo superato il limite inferiore.

%%%%%%%%%%%%%%%%%%%%%%%%%%%%%%%%%%%%%%%%%%%%%%%%%%%%%%%%%%%%%%%%%%%%%%%%%%%%%%%%%%%%%%%%%%%%%%%%%%%%%%%%%%%%%%%%%%%%%%%%%%%%%%%%%%%%%%%%%%%%%%%%%
%% Queste metriche vanno posizionato sotto la sezione finale in cui si decide di metterli (probabilmente rispettivamente in fornitura e sviluppo)
%%%%%%%%%%%%%%%%%%%%%%%%%%%%%%%%%%%%%%%%%%%%%%%%%%%%%%%%%%%%%%%%%%%%%%%%%%%%%%%%%%%%%%%%%%%%%%%%%%%%%%%%%%%%%%%%%%%%%%%%%%%%%%%%%%%%%%%%%%%%%%%%%

%% Questa metrica è il rapporto fra il valore del lavoro effettivamente svolto e il costo sostenuto per realizzarlo
%% Il Cost Performance Index (CPI) è un indice che misura l'efficienza del progetto. Un valore maggiore di 1 indica
%% che il progetto sta utilizzando le risorse in modo efficiente, mentre un valore minore di 1 indica che il progetto
%% sta utilizzando le risorse in modo inefficiente.
%% CPI = EV / AC
\subsubsection{MPC01 - Cost performance index (CPI)}
% Inserire grafico
\textbf{RTB}: Il grafico mostra che il CPI è prossimo ad 1 ma con un leggero calo nel periodo 3. Questo è dovuto al fatto che in questo periodo il team ha ricoperto ruoli più onerosi. Questo ha portato a un aumento dei costi e quindi a un calo del CPI. Questo calo è comunque accettabile e non preoccupante dato che nel periodo 3 dato il calo di lavoro e avendo ricoperto ruoli più economici c'è stato un bilanciamento.

%% Questa metrica la percentuale di requisiti svolti in percentuale di periodo in periodo
%% Questo dovrebbe valutare la stabilità dei requisiti nel tempo
%% Quindi ogni periodo si raggiunge una percentuale di requisiti svolti
%% E alla consenga della baseline RTB si raggiunge il 100%, poi per la PB si raggiunge il 100% di nuovo.
%% Si va ad obbiettivi, o così l'ho interpretato io
% RSI = 100 − {[(RA+RC+RR) / TR] * 100}
% RA: numero di requisiti aggiunti nel periodo considerato;
% RC: numero di requisiti cambiati nel periodo considerato;
% RR: numero di requisiti rimossi nel periodo considerato;
% TR: numero totale di requisiti al momento dell’analisi.
\subsubsection{MPC01 - Requirements stability index (RSI)}
% Inserire grafico
\textbf{RTB}: Questo grafico serve a mostrare l'avanzamento del progresso dei requisiti svolti nel tempo, indicando anche la stabilità nel progresso dei requisiti in percentuale. Inizialmente ovviamente nel primo perio era a 0. La maggior parte dei progressi sono stati fatti nel 3 periodo con l'analisi dei requisiti e le norme di progetto e nel 5 periodo con la finalizzazione della documentazione citata precedentemente e il completamento del PoC. Il 4 periodo ha segnato un calo dovuto alle festività natalizie, mentre il 6 periodo ha segnato un calo dovuto alla persante revisione della documentazione. Questo indice è molto importante per valutare la stabilità dei requisiti nel tempo e per valutare l'efficienza del team nel soddisfare i requisiti.

% Le due sezioni sottostanti saranno da completare per la PB
% \subsection{Qualità di processo - analisi dei requisiti}
% \subsection{Qualità di processo - verifica}

%% Ci sarebbero altre metriche come il code smell e la densità di errori ma non le ho ritenute utili o particolarmente interessanti

\end{justify}
\end{document}
