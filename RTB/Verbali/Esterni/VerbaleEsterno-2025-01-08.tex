\documentclass[10pt]{article}

\usepackage[utf8]{inputenc}
\usepackage{geometry}
\usepackage{tabularx}
\usepackage{graphicx}
\usepackage{hyperref}
\usepackage{array}  

%path per conversione in locale
%\graphicspath{{../../../images/}}

%path per quando caricare in repo
\graphicspath{{images/}}

%cambio misure della pagina
\geometry{a4paper,left=20mm,right=20mm,top=20mm}

\title{Verbale Esterno del meeting in data 2025-01-08}
\date{A.A 2024/2025}

\renewcommand*\contentsname{Indice}
\begin{document}
%contenuti principali
\maketitle
\begin{center}
\includegraphics[width=0.25\textwidth]{LogoUnipd}\\
\includegraphics[width=0.25\textwidth]{Sevenbitslogo}\\
sevenbits.swe.unipd@gmail.com\\
\vspace{2mm}

\textbf{Registro modifiche}\\
\vspace{2mm}
\begin{tabularx}{\textwidth}{|l|l|l|l|X|}
\hline
\textbf{Versione} & \textbf{Data} & \textbf{Autore} & \textbf{Verificatore} & \textbf{Descrizione} \\
\hline
0.1.1 & 2025-01-10 & Leonardo Trolese & Alfredo Rubino & Correzioni strutturali e grammaticali minori \\
\hline
0.1.0 & 2025-01-09 & Leonardo Trolese & Alfredo Rubino & Redazione del verbale \\
\hline
\end{tabularx}
\end{center}

\newpage
\tableofcontents
\newpage
\section{2025-01-08}
\subsection{Durata e partecipanti}
\begin{itemize}
\item Ora: 17:00 - 17:25;
\item Partecipanti: 	
	\begin{itemize}
	\item 	SevenBits:
			\begin{itemize}
				\item Gusella Manuel;
				\item Peruzzi Uncas;
				\item Piva Riccardo;
				\item Pivetta Federico;
				\item Rubino Alfredo;
				\item Cristellon Giovanni;
				\item Trolese Leonardo.
			\end{itemize}
	\item 	SyncLab:
			\begin{itemize}
				\item Dorigo Andrea;
				\item Pallaro Fabio;
				\item Zorzi Daniele.
			\end{itemize}
	\end{itemize}
\item Piattaforma: Google meet (online)
\end{itemize}
\subsection{Oggetto}
Quarto SAL con Dorigo Andrea, Pallaro Fabio e Zorzi Daniele di SyncLab.

\subsection{Sintesi}
Durante il quarto SAL il proponente è stato messo al corrente dello stato del progetto: sono state esposte le novità implementate nel PoC
ed è stato esposto quanto prodotto dal punto di vista della documentazione fino ad ora. Relativamente al Proof of Concept il gruppo ha
esposto un problema rilevato alla creazione dei topic Kafka e la relativa soluzione adottata; il team ha infine confermato di avere 
completato l'implementazione del PoC in vista della RTB.
Nella fase finale della riunione è stato discusso un dubbio emerso durante il colloquio del gruppo con il professor Cardin relativo 
alla sicurezza del software in produzione.

\subsubsection{Domande}
È stato posto un unico dubbio:
\begin{enumerate}
	\item È necessario adottare una soluzione che mitighi o risolva il problema di sicurezza del potenziale Kafka Poisoning, ovvero
	del possibile invio di dati nocivi per il sistema attraverso Kafka da parte di utenti esterni?
\end{enumerate}

\subsubsection{Risposte}
La risposta fornita dal proponente è stata la seguente:
\begin{enumerate}
	\item Il focus del progetto è la generazione di messaggi pubblicitari personalizzati mediante IA, pertanto, sebbene il problema esposto
	andrebbe sicuramente affrontato e risolto in un contesto reale, non è necessario che ciò venga fatto nel contesto del progetto in questione.
	Si suggerisce comunque al gruppo di studiare autonomamente il problema e le possibili soluzioni ad esso, valutando la fattibilità
	dell'adozione di ognuna di esse rispetto al tempo a disposizione per il completamento del progetto. Se il gruppo valutasse accettabile
	l'impegno orario derivante dall'adozione di una di queste soluzioni allora il proponente acconsentirebbe alla sua implementazione. 
\end{enumerate}

\subsubsection{Considerazioni sulla sicurezza}
A seguito della domanda sul potenziale Kafka Poisoning il proponente ha esposto tre possibili soluzioni al problema da approfondire per il gruppo:
\begin{itemize}
	\item Encryption: crittografia dei dati in transito per proteggere il sistema da attacchi del tipo man-in-the-middle, oppure a riposo
	per proteggere i dati anche da attaccanti che hanno accesso ai dati salvati. Si possono configurare i broker Kafka per richiedere connessioni
	sicure tramite SSL/TLS.
	\item Authentication: limitare l'accesso al servizio ai soli client autorizzati. Si può realizzare attraverso SASL o MSA. 
	\item Access Control: definire dei permessi specifici garantendo che anche un client compromesso non possa causare danni al sistema.
\end{itemize}
Il proponente ha quindi suggerito la possibilità di introdurre uno strato di controllo che si limiti a verificare la validità dei dati ricevuti
dai sensori, garantendo la loro coerenza. Questa funzionalità non è da considerasi però una soluzione per il Kafka Poisoning, che in un contesto reale non può 
essere prevenuto così facilmente.

\subsubsection{Decisioni prese}
È stato deciso di terminare la redazione dei documenti per gli ultimi dettagli durante il prossimo sprint, e di fissare conseguentemente la revisione RTB con il committente.
Inoltre, è stato assegnato al gruppo il compito di studiare le possibili soluzioni per i problemi di sicurezza in preparazione a una discussione su di esse
che avverrà durante il successivo SAL, e infine si è scelto di non fissare ancora una data precisa per il prossimo incontro per via delle incertezze
relative a quando si svolgerà la consegna RTB.

\section{Obiettivi prossimo SAL}
Gli obiettivi stabiliti per il prossimo SAL sono:
\begin{itemize}
	\item Studio soluzioni per il Kafka Poisoning;
	\item Conclusione documentazione per RTB.
\end{itemize}
\begin{center}
	\begin{tabular}{|>{\hspace{20pt}}c<{\hspace{20pt}}|>{\hspace{20pt}}c<{\hspace{20pt}}|}
		\hline
		\textbf{Rif.Issue} & \textbf{Dettaglio Decisione}\\
			\hline
				\href{https://github.com/SevenBitsSwe/PoC/issues/18}{Issue \#18} & Studio delle possibili soluzioni del Kafka Poisining\\
			\hline
				\href{https://github.com/SevenBitsSwe/PoC/issues/19}{Issue \#19} & Studio encryption dei dati in transito\\
			\hline
				\href{https://github.com/SevenBitsSwe/PoC/issues/20}{Issue \#20} & Studio access control dei client\\
			\hline
				\href{https://github.com/SevenBitsSwe/PoC/issues/21}{Issue \#21} & Studio authentication dei client\\
			\hline
				\href{https://github.com/SevenBitsSwe/7BitsDocs/issues/111}{Issue \#111} & Redazione verbale esterno 2025-01-08\\
			\hline
	\end{tabular}
	\end{center}

%parte di firma
\vfill
\begin{minipage}{10cm}
Firma: \hrulefill \\
\vspace{2mm}
Data: \dotfill
\end{minipage}

\end{document}
