\documentclass[10pt]{article}

\usepackage[utf8]{inputenc}
\usepackage{geometry}
\usepackage{tabularx}
\usepackage{graphicx}
\usepackage{hyperref}
\usepackage{array}  

%path per quando caricare in repo
\graphicspath{{images/}}

%cambio misure della pagina
\geometry{a4paper,left=20mm,right=20mm,top=20mm}

\title{Verbale Esterno del meeting in data 2024-12-19}
\date{A.A 2024/2025}

\renewcommand*\contentsname{Indice}
\begin{document}
%contenuti principali
\maketitle
\begin{center}
\includegraphics[width=0.25\textwidth]{LogoUnipd}\\
\includegraphics[width=0.25\textwidth]{Sevenbitslogo}\\
sevenbits.swe.unipd@gmail.com\\
\vspace{2mm}

\textbf{Registro modifiche}\\
\vspace{2mm}
\begin{tabularx}{\textwidth}{|l|l|l|l|X|}
\hline
\textbf{Versione} & \textbf{Data} & \textbf{Autore} & \textbf{Verificatore} & \textbf{Descrizione} \\
\hline
1.0.0 & 2024-12-20 & Manuel Gusella & Alfredo Rubino & Approvazione del verbale\\
\hline
0.1.0 & 2024-12-19 & Federico Pivetta & Manuel Gusella & Redazione del verbale\\
\hline
\end{tabularx}
\end{center}

\newpage
\tableofcontents

\newpage
\section{2024-12-19}
\subsection{Durata e partecipanti}
\begin{itemize}
\item Ora: 17:00 - 17:30;
\item Partecipanti: 	
	\begin{itemize}
            \item SevenBits:
            \begin{itemize}
	              \item Gusella Manuel;
                    \item Peruzzi Uncas;
                    \item Piva Riccardo;
                    \item Pivetta Federico;
                    \item Rubino Alfredo;
                    \item Cristellon Giovanni;
                    \item Trolese Leonardo.
	    \end{itemize}
            \item SyncLab:
            \begin{itemize}
                \item Pallaro Fabio;
                \item Dorigo Andrea.
            \end{itemize}
        \end{itemize}
\item Piattaforma: Google meet (online)
\end{itemize}

\subsection{Oggetto}
Terzo SAL con Pallaro Fabio e Dorigo Andrea di SyncLab.

\subsection{Sintesi}
Durante questo SAL sono state poste delle domande al proponente per alcuni chiarimenti ed è stato presentato il lavoro svolto durante il terzo sprint.

    \subsubsection{Domande}
    Sono state poste le seguenti domande:
    \begin{enumerate}
        \item Se l'utente si trova all'interno del range di un punto di interesse e, spostandosi al suo interno, entra nell'area di un altro punto, deve essere generato anche il messaggio relativo a quest'ultimo?
        \item Cosa deve verificarsi quando l'utente esce e successivamente rientra nell'area dello stesso punto di interesse?
    \end{enumerate}

    \subsubsection{Risposte}
    Le risposte fornite dal proponente sono le seguenti:
    \begin{enumerate}
        \item Qualora un utente si trovi all'interno del range di un punto di interesse e, spostandosi al suo interno, entri nell'area appartenente ad un altro punto, il messaggio relativo a quest'ultimo verrà inviato con un ritardo gestito tramite un cooldown, idealmente dopo 20/30 secondi dal precedente messaggio. Nel caso in cui l'utente entri contemporaneamente nell'area di due o più punti, verrà inviato un unico messaggio relativo all'attività la cui tipologia risulti più pertinente rispetto agli interessi dell'utente.
        \item Quando un utente esce e rientra nell'area di un punto di interesse, non deve ricevere nuovamente lo stesso messaggio. È inoltre essenziale tenere traccia dei messaggi già inviati a ciascun utente, in modo da evitare che lo stesso punto di interesse invii più di un messaggio allo stesso utente nel corso della stessa giornata.
    \end{enumerate}

    \subsubsection{PoC}
    È stato mostrato al proponente il lavoro effettuato durante il terzo sprint. Come richiesto nel precedente SAL, sono state integrate le due parti del PoC precedentemente sviluppate, implementate le chiavi Kafka e generato un numero sufficiente di punti di interesse.\\
    Il proponente ha approvato quanto presentato, dichiarando che l'obiettivo prefissato per la revisione RTB in relazione al PoC è stato pienamente conseguito. Si può pertanto procedere con il completamento della documentazione da allegare al PoC.

    \subsubsection{Suggerimenti per il futuro}
    In seguito alla presentazione del PoC, il proponente ha fornito alcuni suggerimenti in vista della revisione PB. È stata evidenziata la necessità di migliorare la chiarezza della rappresentazione della direzione di spostamento dell’utente e, in prospettiva della gestione di più utenti, di implementare una soluzione che ne consenta una distinzione chiara e immediata. Inoltre è stato richiesto di ottimizzare la rappresentazione grafica dei messaggi, migliorandone l’aspetto visivo e assicurando che siano posizionati sopra l’utente, evitando così sovrapposizioni con i marker relativi alle attività.\\
    Per quanto riguarda l’interazione su Grafana, è stato richiesto di risolvere il problema della perdita di focus conseguente allo zoom-out durante i refresh del sistema.\\
    Infine, per i punti di interesse, è necessario integrare ulteriori informazioni significative, come il numero di annunci generati mensilmente mentre per ciascun utente, si richiede l’implementazione di un pannello dedicato, accessibile con un singolo click, per facilitare la consultazione dei dati individuali. La dashboard principale dovrà offrire una visione sintetica e mirata, mostrando esclusivamente una mappa con gli utenti in movimento, le attività fisse e l'ultimo messaggio ricevuto da ciascun utente.
    
    \subsubsection{Decisioni Prese}
    Con il proponente si è deciso di ultimare il PoC con le ultime correzioni minori, aggiungere i casi d'uso relativi alle funzionalità discusse e terminare la stesura dei documenti.\\
    Infine è stata fissata la data per il prossimo SAL, che coinciderà con la fine del quarto sprint, della durata di 3 settimane invece delle consuete 2, per il 2025-01-08 alle ore 17:00.

\subsection{Obiettivi prossimo SAL}
Come obiettivi per il prossimo SAL è stato deciso con il proponente di:
\begin{itemize}
    \item Ultimare il PoC con le ultime correzioni minori;
    \item Aggiungere i casi d'uso relativi alle funzionalità discusse durante questo SAL;
    \item Terminare la stesura dei vari documenti da presentare alla revisione RTB.
\end{itemize}
\begin{center}
\begin{tabular}{|>{\hspace{20pt}}c<{\hspace{20pt}}|>{\hspace{20pt}}c<{\hspace{20pt}}|}
	\hline
	\textbf{Rif.Issue} & \textbf{Dettaglio Decisione}\\
	\hline
            \href{https://github.com/SevenBitsSwe/7BitsDocs/issues/77}{Issue \#77} & Aggiungere UC : Dashboard singolo utente + sotto casi d'uso collegati\\
        \hline
            \href{https://github.com/SevenBitsSwe/7BitsDocs/issues/78}{Issue \#78} & Aggiungere UC : Visualizzazione quantità di messaggi generati per POI\\
        \hline
            \href{https://github.com/SevenBitsSwe/7BitsDocs/issues/80}{Issue \#80} & Redazione Verbale Esterno 2024-12-19\\
        \hline
\end{tabular}
\end{center}

%parte di firma
\vfill
\begin{minipage}{10cm}
Firma: \hrulefill \\
\vspace{2mm}
Data: \dotfill
\end{minipage}

\end{document}
