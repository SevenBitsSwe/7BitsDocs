\documentclass[12pt]{article}

\usepackage[utf8]{inputenc}
\usepackage{geometry}
\usepackage{tabularx}
\usepackage{graphicx}

%path per conversione in locale
%\graphicspath{{../../../images/}}

%path per quando caricare in repo
\graphicspath{{images/}}

%cambio misure della pagina
\geometry{a4paper,left=20mm,right=20mm,top=20mm}

\title{Verbale Esterno del meeting in data 2024-11-11}
\date{A.A 2024/2025}

\renewcommand*\contentsname{Indice}
\begin{document}
%contenuti principali
\maketitle
\begin{center}
\includegraphics[width=0.25\textwidth]{LogoUnipd}\\
\includegraphics[width=0.25\textwidth]{Sevenbitslogo}\\
sevenbits.swe.unipd@gmail.com\\
\vspace{2mm}

\textbf{Registro modifiche}\\
\vspace{2mm}
\begin{tabular}{|l|l|l|l|l|l|}
\hline
\textbf{Versione} & \textbf{Data} & \textbf{Autore} & \textbf{Verificatore} & \textbf{Descrizione} \\
\hline
1.0.0 & 2024-11-11 & Manuel Gusella & Uncas Peruzzi & Stesura iniziale del verbale\\
& & & Alfredo Rubino &\\
\hline
\end{tabular}
\end{center}
\newpage
\tableofcontents
\newpage
\section{2024-11-11}
\subsection{Durata e partecipanti}
\begin{itemize}
\item Ora: 17:00 - 17:25;
\item Partecipanti: 	
	\begin{itemize}
	\item Gusella Manuel;
	\item Cristellon Giovanni;
	\item Peruzzi Uncas;
	\item Piva Riccardo;
	\item Pivetta Federico;
	\item Rubino Alfredo;
	\item Trolese Leonardo.
	\end{itemize}
\item Piattaforma: Google meet (online)
\end{itemize}
\subsection{Oggetto}
Riunione organizzativa con Andra Rodrigo, Fabio Pallaro e Daniele Zorzi di SyncLab.
Si sono definiti i periodi di sprint e il tema di questo primo sprint che durerà fino al 2024-11-25.

\subsection{Sintesi}
In questo meeting si è discusso principalmente dell'organizzazione per questa prima fase di progetto.
\subsubsection{Organizzazione}
Si sono già fissati due meeting, per poter già avere delle date fissate di metà e scadenza sprint.\\
Come periodo di sprint si è deciso che debba essere di 2 settimane, per darci il tempo di portare risultati significativi al SAL, con un incontro intermedio   per visionare gli avanzamenti durante quello sprint.

\subsubsection{Tecnologie}
Suggeriscono l'utilizzo di docker compose e consigliano di partire con una costruzione effettiva limitata al dato iniziale, con una generazione dati degli utenti e messaggi in streaming.\\
Propongono lo studio delle tecnologie consigliate nella documentazione di capitolato per poter avere un'idea sul come realizzare il progetto e quali effettivamente utilizzare nella sua realizzazione.

\subsubsection{Documentazione}
Si è parlato di alcune parti di documentazione che si devono creare durante questa fase di RTB, che saranno: l'analisi dei requisiti, il piano di progetto e lo studio di fattibilità.\\
Nella fase iniziale di analisi hanno suggerito di non vincolarsi troppo ai dettagli, cosa che però sarà richiesta nei Casi d'uso.

\subsection{Conclusioni}
\subsubsection{Decisioni Prese}
Queste sono le decisioni prese durante il meeting:
\begin{itemize}
	\item Discord come piattaforma di contatto rapido tra proponente e fornitore;
	\item Sprint della durata di 2 settimane con incontro di revisione intermedio;
	\item Date prossimi incontri:
	\begin{itemize}
		\item Intermedio: 2024-11-20
		\item SAL: 2024-11-25
	\end{itemize}
\end{itemize}

\subsubsection{Compiti concordati}
Queste sono i compiti che si siamo assegnati durante questo periodo di sprint:
\begin{itemize}
	\item Studio conoscitivo delle tecnologie consigliate dai proponenti;
	\item Inizio analisi dei requisiti e struttura del progetto.
\end{itemize}

%parte di firma
\vfill
\begin{minipage}{10cm}
Firma: \hrulefill \\
\vspace{2mm}
Data: \dotfill
\end{minipage}




\end{document}
