\documentclass[12pt]{article}

\usepackage[utf8]{inputenc}
\usepackage{geometry}
\usepackage{tabularx}
\usepackage{graphicx}

%\graphicspath{{images/}}

%cambio misure della pagina
\geometry{a4paper,left=20mm,right=20mm,top=20mm}

\title{Verbale Esterno del meeting in data 2024-11-11}
\date{A.A 2024/2025}

\renewcommand*\contentsname{Indice}
\begin{document}
%contenuti principali
\maketitle
\center 
%\includegraphics[width=0.25\textwidth]{LogoUnipd}\\
%\includegraphics[width=0.25\textwidth]{Sevenbitslogo}\\
sevenbits.swe.unipd@gmail.com\\
\vspace{2mm}

\textbf{Registro modifiche}\\
\vspace{2mm}
\begin{tabular}{|l|l|l|l|l|l|}
\hline
\textbf{Versione} & \textbf{Data} & \textbf{Autore} & \textbf{Verificatore} & \textbf{Descrizione} \\
\hline
0.1.0 & 2024-11-11 & Manuel Gusella & Uncas Peruzzi & Stesura dell'oggetto e della sintesi\\
\hline
\end{tabular}

\raggedright
\tableofcontents
\newpage
\section{2024-11-11}
\subsection{Durata e partecipanti}
\begin{itemize}
\item Ora: 17:00 - 17:25;
\item Partecipanti: 	
	\begin{itemize}
	\item Gusella Manuel;
	\item Cristellon Giovanni;
	\item Peruzzi Uncas;
	\item Piva Riccardo;
	\item Pivetta Federico;
	\item Rubino Alfredo;
	\item Trolese Leonardo.
	\end{itemize}
\item Piattaforma: Google meet (online)
\end{itemize}
\subsection{Oggetto}
Riunione organizzativa con Andra Rodrigo, Fabio Pallaro e Daniele Zorzi di SyncLab.
Si sono definiti i periodi di sprint e il tema di questo primo sprint che durerà fino al 2024-11-25.

\subsection{Sintesi}
In questo meeting si è discusso principalmente dell'organizzazione per questa prima fase di progetto. Si sono già fissati due meeting, per poter già avere delle date fissate di metà e scadenza sprint. \\
Sono state fissate degli obiettivi, a cui i proponenti tengono molto, che sono: Una prima analisi dei requisiti senza troppe vincolazioini ed informarsi sulle tecnologie consigliate nel capitolato per avere già un'idea sul come realizzare il capitolato.\\
Per quanto riguarda le tecnologie suggeriscono l'utilizzo di docker compose e consigliano di partire con una costruzione effettiva, su cui produrre il dato iniziale (generazione dati degli utenti e i messaggi in streaming).\\
\subsection{Conclusioni}

\end{document}
