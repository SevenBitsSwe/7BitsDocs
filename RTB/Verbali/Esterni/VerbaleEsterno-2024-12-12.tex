\documentclass[10pt]{article}

\usepackage[utf8]{inputenc}
\usepackage{geometry}
\usepackage{tabularx}
\usepackage{graphicx}
\usepackage{hyperref}
\usepackage{array}  

%path per quando caricare in repo
\graphicspath{{images/}}

%cambio misure della pagina
\geometry{a4paper,left=20mm,right=20mm,top=20mm}

\title{Verbale Esterno del meeting in data 2024-12-12}
\date{A.A 2024/2025}

\renewcommand*\contentsname{Indice}
\begin{document}
%contenuti principali
\maketitle
\begin{center}
\includegraphics[width=0.25\textwidth]{LogoUnipd}\\
\includegraphics[width=0.25\textwidth]{Sevenbitslogo}\\
sevenbits.swe.unipd@gmail.com\\
\vspace{2mm}

\textbf{Registro modifiche}\\
\vspace{2mm}
\begin{tabularx}{\textwidth}{|l|l|l|l|X|}
\hline
\textbf{Versione} & \textbf{Data} & \textbf{Autore} & \textbf{Verificatore} & \textbf{Descrizione} \\
\hline
0.1.0 & 2024-12-12 & Federico Pivetta & Alfredo Rubino & Redazione del verbale\\
\hline
\end{tabularx}
\end{center}

\newpage
\tableofcontents

\newpage
\section{2024-12-12}
\subsection{Durata e partecipanti}
\begin{itemize}
\item Ora: 17:00 - 17:30;
\item Partecipanti: 	
	\begin{itemize}
            \item SevenBits:
            \begin{itemize}
	              \item Gusella Manuel;
                    \item Peruzzi Uncas;
                    \item Piva Riccardo;
                    \item Pivetta Federico;
                    \item Rubino Alfredo;
                    \item Trolese Leonardo;
	    \end{itemize}
            \item SyncLab:
            \begin{itemize}
                \item Zorzi Daniele;
                \item Dorigo Andrea;
            \end{itemize}
        \end{itemize}
\item Piattaforma: Google meet (online)
\end{itemize}

\subsection{Oggetto}
Incontro intermedio del terzo sprint con Zorzi Daniele e Dorigo Andrea di SyncLab.

\subsection{Sintesi}
Nel corso del meeting è stato esaminato il punto di avanzamento del gruppo nello sprint e sono state poste delle domande ai proponenti per alcuni chiarimenti.

\subsubsection{Domande}
Le domande sottoposte sono le seguenti:
\begin{enumerate}
    \item Rimane confermato l'obbligo di raggiungere una copertura dei test pari o superiore all'80\%, mentre il raggiungimento del 100\% è considerato un valore aggiunto?
    \item Vi sono requisiti specifici relativi alla qualità, oltre a quelli già indicati nel capitolato, o ulteriori metriche da fornire che non siano già previste?
    \item Per quanto riguarda i test da eseguire sul prodotto, sono previsti dei test specifici o la scelta è lasciata alla discrezione del gruppo?
    \item È appropriato utilizzare il metodo geodistance di ClickHouse per il calcolo della distanza tra i punti?
    \item Quale distanza si ritiene più ragionevole per l'invio dell'annuncio pubblicitario, nel caso in cui si entri all'interno dell'area di un punto di interesse?
    \item Vi sono dettagli da sistemare riguardo al documento Analisi dei Requisiti?
\end{enumerate}

\subsubsection{Risposte}
Le risposte fornite sono le seguenti:
\begin{enumerate}
    \item Si, viene confermato quanto specificato nel capitolato.
    \item Per quanto riguarda gli aspetti di qualità, la scelta degli strumenti e delle metodologie è lasciata alla discrezione del gruppo. In merito alla manutenibilità, si suggerisce di considerare l'adozione della metrica Function Point, che associa ad un requisito una specifica sezione di codice. Per la complessità ciclomatica, si ritiene che sia difficile da misurare nel contesto del progetto, pertanto si consiglia di non prenderla in considerazione, tuttavia si richiede di supportare tale decisione mediante una motivazione dettagliata.   
    \item La scelta dei test da eseguire è lasciata alla discrezione del gruppo, in base a quanto ritenuto utile per il progetto. Si consiglia di includere test end-to-end manuali, che prevedano la produzione di dati, il passaggio attraverso Kafka, l'elaborazione tramite Flink e la visualizzazione su Grafana. Inoltre, è stato suggerito di effettuare delle prove di carico, ad esempio per misurare il numero di dati processati entro un determinato intervallo di tempo (ad esempio, 10 secondi), utilizzando eventualmente i limiti dei container.
    \item Questa scelta è stata ritenuta opportuna, sebbene comporti uno spostamento del carico di lavoro nel database. Sebbene in alcuni casi possa non essere conveniente sovraccaricare il database, nel presente contesto tale soluzione è stata ritenuta adeguata. La decisione di delegare il calcolo della distanza a ClickHouse ha un impatto diretto sulla qualità e sulle prestazioni complessive del sistema  e, pertanto, deve essere documentata.
    \item Si ritiene ragionevole definire una distanza pari a 100 m, 200 m o 1 km, individuando l'intervallo più adatto alle esigenze e al contesto applicativo del progetto.
    \item È stato suggerito di rivedere l'ordine dei casi d'uso e per quanto riguarda le precondizioni, quando necessario, va specificato che l'utente esista ed è in possesso delle credenziali. I casi d'uso attualmente forniti sono stati ritenuti sufficienti, tuttavia potrebbe essere utile aggiungere ulteriori dettagli. Qualora si desideri includere nuovi casi d'uso, si potrebbe considerare la trasmissione dei dati e l'interazione tra i vari moduli.
\end{enumerate}

\subsection{Decisioni Prese}
Le decisioni prese, in comune accordo, sono le seguenti:
\begin{itemize}
    \item Si è deciso di documentare la scelta del metodo geodistance di ClickHouse nell'Analisi dei Requisiti, specificandola come requisito prestazionale, in quanto influisce significativamente sulle performance del sistema.
    \item Si è deciso di creare una tabella separata nel database al fine di mantenere lo storico dei dati posizionali.
    \item Per quanto riguarda le riunioni durante il periodo delle vacanze, sarà garantita la presenza di almeno un membro della proponente per rispondere ad eventuali quesiti ed è stato concordato di svolgere esclusivamente i SAL con cadenza bi-settimanale, escludendo gli incontri intermedi.
\end{itemize}


%parte di firma
\vfill
\begin{minipage}{10cm}
Firma: \hrulefill \\
\vspace{2mm}
Data: \dotfill
\end{minipage}

\end{document}
