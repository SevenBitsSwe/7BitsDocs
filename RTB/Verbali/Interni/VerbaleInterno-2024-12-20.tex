\documentclass[10pt]{article}

\usepackage[utf8]{inputenc}
\usepackage{geometry}
\usepackage{tabularx}
\usepackage{graphicx}
\usepackage{hyperref}
\usepackage{array}  

%path per conversione in locale
%\graphicspath{{../../../images/}}

%path per quando caricare in repo
\graphicspath{{images/}}

%cambio misure della pagina
\geometry{a4paper,left=20mm,right=20mm,top=20mm}

\title{Verbale Interno del meeting in data 2024-12-20}
\date{A.A 2024/2025}

\renewcommand*\contentsname{Indice}
\begin{document}
%contenuti principali
\maketitle
\center 
\includegraphics[width=0.25\textwidth]{LogoUnipd}\\
\includegraphics[width=0.25\textwidth]{Sevenbitslogo}\\
sevenbits.swe.unipd@gmail.com\\
\vspace{2mm}

\textbf{Registro modifiche}\\
\vspace{2mm}
\begin{tabular}{|l|l|l|l|l|l|}
\hline
\textbf{Versione} & \textbf{Data} & \textbf{Autore} & \textbf{Verificatore} & \textbf{Descrizione} \\
\hline
0.1.0 & 2024-12-20 & Federico Pivetta & Riccardo Piva & Stesura del verbale\\
\hline
\end{tabular}

\newpage
\raggedright
\tableofcontents

\newpage
\section{2024-12-20}
\subsection{Durata e partecipanti}
\begin{itemize}
\item Ora: 9:15 - 11:00;
\item Partecipanti: 	
	\begin{itemize}
	    \item Gusella Manuel;
            \item Cristellon Giovanni;
            \item Peruzzi Uncas;
            \item Piva Riccardo;
            \item Pivetta Federico;
            \item Rubino Alfredo;
            \item Trolese Leonardo.
	\end{itemize}
\item Piattaforma: Discord (online)
\end{itemize}

\subsection{Sintesi}
Il gruppo si è riunito per pianificare il quarto sprint. Nel corso della riunione sono stati definiti i ruoli di ciascun membro per il prossimo periodo, elaborato il preventivo e redatto il consuntivo delle ore. Inoltre, sono state identificate le issue da completare durante questo nuovo sprint.\\
Come concordato con la proponente, il quarto sprint terminerà il 2025-01-08 e non è stato fissato un incontro intermedio.

\subsubsection{Assegnazione Ruoli}
Anche per il quarto sprint, si è deciso di riassegnare i ruoli in modo tale che nessun membro possa ricoprire lo stesso incarico svolto negli sprint precedenti. Sono state previste alcune eccezioni per i ruoli di Analista e Verificatore, in quanto questi ruoli, in ogni sprint concluso fino ad ora, sono stati ricoperti da almeno due membri del gruppo.

\subsubsection{Preventivo e Consuntivo delle ore}
Nel corso dell'incontro è stato analizzato il numero di ore dedicate a ciascun ruolo durante lo sprint appena concluso. Successivamente, sono stati elaborati il preventivo delle ore per il quarto sprint e il consuntivo delle ore per il terzo sprint, entrambi inseriti nel documento Piano di Progetto.

\subsubsection{Issue}
Per garantire che ogni membro del gruppo sia informato sulle task da completare durante il quarto sprint e per ottimizzare il lavoro asincrono, sono state create diverse issue che definiscono chiaramente i compiti da svolgere. È stata data particolare priorità al completamento dei documenti da presentare alla revisione RTB e alle correzioni minori da apportare al PoC.

\subsubsection{Analisi dei Requisiti}
A seguito dell'incontro con il committente, il Professore Cardin Riccardo, il gruppo ha discusso riguardo agli argomenti trattati. È stata presa la decisione di correggere i casi d'uso già individuati, sistemando eventuali errori e specificando le parti ritenute troppo astratte, nonché di introdurre nuovi attori, come il Large Language Model (LLM), il sottosistema che visualizza i dati e il sottosistema che invia/elabora i dati.

\subsubsection{Procedimento di verifica}
Per facilitare la comunicazione tra il Verificatore e l'autore di un compito, è stata adottata la funzionalità dei suggerimenti fornita da GitHub. Da ora in avanti, il Verificatore comunicherà l'esito della sua ispezione attraverso un commento direttamente dalla Pull Request di GitHub e taggando l'autore. Questo consentirà all'autore di ricevere una notifica via email riguardo l'esito della verifica e di visualizzare le sezioni da modificare, con un suggerimento sulla possibile soluzione da applicare.

\subsection{Decisioni Prese}
Di seguito vengono riportate le decisioni finali emerse dalle diverse discussioni:
\begin{itemize}
    \item Assegnazione dei ruoli per il quarto periodo di sprint:\\
            \vspace{1mm}
            Responsabile: Leonardo Trolese;\\
            Amministratore: Federico Pivetta;\\
            Analisti: Alfredo Rubino, Giovanni Cristellon, Manuel Gusella;\\
            Verificatori: Riccardo Piva,Uncas Peruzzi.\\
    \item Adottare il nuovo procedimento per la verifica;
    \item Correggere e ampliare il documento Analisi dei Requisiti a seguito del ricevimento effettuato;
    \item Terminare ogni documento da consegnare alla revisione RTB;
    \item Apportare le ultime correzioni minori al PoC.
\end{itemize}

\begin{center}
\begin{tabular}{|>{\hspace{20pt}}c<{\hspace{20pt}}|>{\hspace{20pt}}c<{\hspace{20pt}}|}
	\hline
	\textbf{Rif.Issue} & \textbf{Dettaglio Decisione}\\
        \hline
            \href{https://github.com/SevenBitsSwe/7BitsDocs/issues/82}{Issue \#82} & Sistemare UC su visualizzazione delle informazioni\\
        \hline
            \href{https://github.com/SevenBitsSwe/7BitsDocs/issues/83}{Issue \#83} & Creazione UC per attore secondario LLM\\
        \hline
            \href{https://github.com/SevenBitsSwe/7BitsDocs/issues/84}{Issue \#84} & Aggiungere UC: sistema di visualizzazione\\
        \hline
            \href{https://github.com/SevenBitsSwe/7BitsDocs/issues/85}{Issue \#85} & Aggiungere UC: sistema di invio/elaborazione dei dati\\
        \hline
            \href{https://github.com/SevenBitsSwe/7BitsDocs/issues/86}{Issue \#86} & Sistemazione UC5\\
        \hline
\end{tabular}
\end{center}

\end{document}