
\documentclass[10pt]{article}

\usepackage[utf8]{inputenc}
\usepackage{geometry}
\usepackage{tabularx}
\usepackage{graphicx}
\usepackage{hyperref}
\usepackage{array}

%path per conversione in locale
%\graphicspath{{../../../images/}}

%path per quando caricare in repo
\graphicspath{{images/}}

%cambio misure della pagina
\geometry{a4paper,left=20mm,right=20mm,top=20mm}

\title{Verbale Interno del meeting in data 2025-01-22}
\date{A.A 2024/2025}

\renewcommand*\contentsname{Indice}
\begin{document}
%contenuti principali
\maketitle
\begin{center}
\includegraphics[width=0.25\textwidth]{LogoUnipd}\\
\includegraphics[width=0.25\textwidth]{Sevenbitslogo}\\
sevenbits.swe.unipd@gmail.com\\
\vspace{2mm}

\textbf{Registro modifiche}\\
\vspace{2mm}
\begin{tabularx}{\textwidth}{|l|l|l|l|X|}
\hline
\textbf{Versione} & \textbf{Data} & \textbf{Autore} & \textbf{Verificatore} & \textbf{Descrizione} \\
\hline
1.0.0 & 2025-01-22 & Peruzzi Uncas & Rubino Alfredo & Approvazione del verbale \\
\hline
0.1.0 & 2025-01-22 & Riccardo Piva & Peruzzi Uncas & Redazione del verbale \\
\hline
\end{tabularx}
\end{center}

\newpage
\tableofcontents
\newpage
\section{2025-01-22}
\subsection{Durata e partecipanti}
\begin{itemize}
\item Ora: 15:00 - 16:00;
\item Partecipanti:
	\begin{itemize}
    	\item Giovanni Cristellon;
		\item Gusella Manuel;
		\item Peruzzi Uncas;
		\item Piva Riccardo;
		\item Pivetta Federico;
		\item Rubino Alfredo;
		\item Trolese Leonardo.
	\end{itemize}
\item Piattaforma: Discord (online)
\end{itemize}

\subsection{Sintesi}
Il gruppo si è riunito dopo la riunione con il professor Cardin, per organizzarsi riguardo le modifiche da apportare al documento Analisi dei requisiti.
Nel corso della riunione il team ha discusso l'insieme di modifiche da apportare all'analisi dei requisiti, il calcolo delle metriche nel cruscotto, la gestione della continuazione dello sprint e la richiesta di un incontro con la proponente.

\subsubsection{Issue}
Durante l'incontro sono state create cinque issue riguardanti alcune modifiche da apportare all'analisi dei requisiti e dei termini da inserire nel glossario.

\subsection{Decisioni Prese}
Segue un elenco delle decisioni prese durante la riunione:
\begin{itemize}
    \item Nuova riunione con la proponente per chiarimenti riguardanti modifiche UC Analisi Dei Requisiti.
    \item Lo sprint viene prolungato di una settimana dato il periodo degli esami e la prossimità alla consegna RTB ma anche nell'attesa dell'incontro con la proponente.
    \item Aggiungere termini poco chiari al glossario, per spiegarne il significato specifico in questo contesto.
    \item Togliere gli attori secondari che si sono rivelati errati dall'Analisi Dei Requisiti.
    \item Sistemazione di vari UC spostando le informazioni dalle postcondizioni a degli UC specifici.
\end{itemize}
\begin{center}
\begin{tabular}{|>{\hspace{20pt}}c<{\hspace{20pt}}|>{\hspace{20pt}}c<{\hspace{20pt}}|}
	\hline
	    \textbf{Rif.Issue} & \textbf{Dettaglio Decisione}\\
	\hline
		\href{https://github.com/SevenBitsSwe/7BitsDocs/issues/125}{Issue \#125} & Aggiungere termini "marker" e "percorso" al glossario\\
	\hline
		\href{https://github.com/SevenBitsSwe/7BitsDocs/issues/126}{Issue \#126} & Rimuovere attori secondari da UC3, UC6, UC9, UC10, UC11\\
	\hline
		\href{https://github.com/SevenBitsSwe/7BitsDocs/issues/127}{Issue \#127} & Sistemare UC 1.1.1.1\\
	\hline
		\href{https://github.com/SevenBitsSwe/7BitsDocs/issues/128}{Issue \#128} & Sistemare UC 1.2.2\\
	\hline
		\href{https://github.com/SevenBitsSwe/7BitsDocs/issues/129}{Issue \#129} & Sistemare UC 1.3.1\\
	\hline
\end{tabular}
\end{center}
\end{document}
