\documentclass[10pt]{article}

\usepackage[utf8]{inputenc}
\usepackage{geometry}
\usepackage{tabularx}
\usepackage{graphicx}
\usepackage{hyperref}
\usepackage{array}  

%path per conversione in locale
%\graphicspath{{../../../images/}}

%path per quando caricare in repo
\graphicspath{{images/}}

%cambio misure della pagina
\geometry{a4paper,left=20mm,right=20mm,top=20mm}

\title{Verbale Interno del meeting in data 2024-11-27}
\date{A.A 2024/2025}

\renewcommand*\contentsname{Indice}
\begin{document}
%contenuti principali
\maketitle
\begin{center}
\includegraphics[width=0.25\textwidth]{LogoUnipd}\\
\includegraphics[width=0.25\textwidth]{Sevenbitslogo}\\
sevenbits.swe.unipd@gmail.com\\
\vspace{2mm}

\textbf{Registro modifiche}\\
\vspace{2mm}
\begin{tabularx}{\textwidth}{|l|l|l|l|X|}
\hline
\textbf{Versione} & \textbf{Data} & \textbf{Autore} & \textbf{Verificatore} & \textbf{Descrizione} \\
\hline
1.0.0 & 2024-11-27 & Leonardo Trolese & Giovanni Cristellon & Piccole modifiche ortografiche e verifica del verbale\\
\hline
0.1.0 & 2024-11-27 & Manuel Gusella & Leonardo Trolese & Stesura del verbale\\
\hline
\end{tabularx}
\end{center}

\newpage
\tableofcontents
\newpage
\section{2024-11-27}
\subsection{Durata e partecipanti}
\begin{itemize}
\item Ora: 16:00 - 17:45;
\item Partecipanti: 	
	\begin{itemize}
	\item Gusella Manuel;
	\item Cristellon Giovanni;
	\item Peruzzi Uncas;
	\item Piva Riccardo;
	\item Pivetta Federico;
	\item Rubino Alfredo;
	\item Trolese Leonardo.
	\end{itemize}
\item Piattaforma: Discord (online)
\end{itemize}
\subsection{Sintesi}
Riunione interna per l'assegnazione dei ruoli per questo secondo periodo di sprint e per discutere su alcuni dubbi interni del gruppo sui temi scritti a seguire.

\subsubsection{Assegnazione Ruoli}
Per il secondo sprint si è deciso di suddividere i ruoli cercando di non riassegnare lo stesso ruolo a membri che lo hanno già ricoperto.

\subsubsection{Uso dell'impersonale nella documentazione}
Visto un suggerimento da parte del proponente di utilizzare l'impersonale per documenti tecnici e di supporto, il gruppo ha deciso di utilizzare la terza persona per i prossimi documenti da redarre.

\subsubsection{Priorità delle issue}
D'ora in avanti si è deciso di assegnare alle issue successive anche un livello di priorità per capire più facilmente su che lavori concentrarsi e per avere una suddivisione più dettagliata delle issue.

\subsubsection{Visualizzazione dei termini di Glossario}
Per i termini di Glossario, soprattutto per quelli composti, si è deciso di aggiungere, oltre al G a pedice, anche dei trattini per una più semplice lettura del termine che è effettivamente presente nel Glossario.\\
Non si è voluto utilizzare il corsivo, visto che viene utilizzato per far riferimento ai Documenti.

\subsubsection{PoC}
In questa riunione interna si è discusso sull'avanzamento del PoC per poterlo poi mostrare al proponente durante il prossimo SAL.\\
In questi giorni si è riuscito a creare un ambiente Flink funzionante, l'unica cosa che manca è il connettore tra Flink e ClickHouse per reperire i dati dalla tabella.

\subsection{Decisioni Prese}
Queste sono le decisioni finali dei componenti
\begin{itemize}
\item Decisione dei ruoli fino al 2024-12-05:\\
Responsabile: Uncas Peruzzi\\
Amministratore: Alfredo Rubino\\
Analisti: Manuel Gusella, Riccado Piva, Federico Pivetta\\
Verificatori: Leonardo Trolese, Giovanni Cristellon\\
\item Utilizzo dell'informale sui prossimi documenti
\item Adozione di 4 livelli di priorità per le issue
\item Cambiamento della visualizzazione dei termini di Glossario in: termine$_G$ o termine-composto$_G$
\item Aggiornare i termini di Glossario già presenti nel formato di visualizzazione deciso in riunione
\item Connettere l'ambiente di Apache Flink con ClickHouse ed iniziare la popolazione del DB in ClickHouse con i dati degli utenti
\end{itemize}
\begin{center}
\begin{tabular}{|>{\hspace{20pt}}c<{\hspace{20pt}}|>{\hspace{20pt}}c<{\hspace{20pt}}|}
	\hline
	\textbf{Rif.Issue} & \textbf{Dettaglio Decisione}\\
	\hline
	\href{https://github.com/SevenBitsSwe/PoC/issues/1}{Issue \#1} &  Creazione di un collegamento tra Apache Flink e Clickhouse\\
	\hline
	\href{https://github.com/SevenBitsSwe/7BitsDocs/issues/47}{Issue \#47} &  Redazione Verbale Interno 2024-11-27\\
	\hline
	\href{https://github.com/SevenBitsSwe/7BitsDocs/issues/48}{Issue \#48} &  Cambiare i termini di Glossario presenti nel documento di Piano di Progetto\\
	\hline
	\href{https://github.com/SevenBitsSwe/7BitsDocs/issues/49}{Issue \#49} &  Cambiare i termini di Glossario presenti nel documento di Analisi dei Requisiti\\
	\hline
	\href{https://github.com/SevenBitsSwe/7BitsDocs/issues/50}{Issue \#50} &  Cambiare i termini di Glossario presenti nel documento di Norme di Progetto\\
	\hline
\end{tabular}
\end{center}
\end{document}
