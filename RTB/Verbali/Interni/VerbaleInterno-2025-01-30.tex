
\documentclass[10pt]{article}

\usepackage[utf8]{inputenc}
\usepackage{geometry}
\usepackage{tabularx}
\usepackage{graphicx}
\usepackage{hyperref}
\usepackage{array}

%path per conversione in locale
%\graphicspath{{../../../images/}}

%path per quando caricare in repo
\graphicspath{{images/}}

%cambio misure della pagina
\geometry{a4paper,left=20mm,right=20mm,top=20mm}

\title{Verbale Interno del meeting in data 2025-01-30}
\date{A.A 2024/2025}

\renewcommand*\contentsname{Indice}
\begin{document}
%contenuti principali
\maketitle
\begin{center}
\includegraphics[width=0.25\textwidth]{LogoUnipd}\\
\includegraphics[width=0.25\textwidth]{Sevenbitslogo}\\
sevenbits.swe.unipd@gmail.com\\
\vspace{2mm}

\textbf{Registro modifiche}\\
\vspace{2mm}
\begin{tabularx}{\textwidth}{|l|l|l|l|X|}
\hline
\textbf{Versione} & \textbf{Data} & \textbf{Autore} & \textbf{Verificatore} & \textbf{Descrizione} \\
\hline
1.0.0 & 2025-02-06 & Giovanni Cristellon & Uncas Peruzzi& Approvazione del verbale \\
\hline
0.1.0 & 2025-01-30 & Riccardo Piva & Giovanni Cristellon & Redazione del verbale \\
\hline
\end{tabularx}
\end{center}

\newpage
\tableofcontents
\newpage
\section{2025-01-30}
\subsection{Durata e partecipanti}
\begin{itemize}
\item Ora: 16:00 - 16:30;
\item Partecipanti:
	\begin{itemize}
    	\item Giovanni Cristellon;
		\item Gusella Manuel;
		\item Peruzzi Uncas;
		\item Piva Riccardo;
		\item Pivetta Federico;
		\item Rubino Alfredo;
		\item Trolese Leonardo.
	\end{itemize}
\item Piattaforma: Discord (online)
\end{itemize}

\subsection{Sintesi}
Il gruppo si è riunito per concludere lo sprint. Durante l'incontro si sono discusse le correzioni apportate al documento di analisi
dei requisiti in seguito all'incontro con il professor Cardin e per confrontarsi riguardo le prossime modifiche relative allo stesso
documento sulla base delle informazioni ricevute dalla proponente.
Nel corso della riunione il gruppo ha analizzato lo scorso sprint e di conseguenza anche il consuntivo
del periodo precedente per poi ruotare i ruoli per il prossimo sprint stimandone anche il preventivo.
L'incontro è terminato con una discussione riguardante le prossime attività e la pianificazione delle stesse
per programmare la consegna RTB.

\subsubsection{Assegnazione ruoli}
Il gruppo ha deciso di assegnare i nuovi ruoli sulla base dei ruoli che un membro non avesse ancora svolto.
Avendo ormai stilato gran parte della documentazione, il gruppo ha deciso d'impegnarsi nel completamento delle modifiche decise
per il documento di analisi dei requisiti. Ma anche di verificare i documenti già completi o ultimare i documenti parzialmente terminati.

\subsubsection{Preventivo e Consuntivo delle ore}
Durante la riunione il gruppo ha confermato il consuntivo delle ore relative allo svolgimento del quinto sprint e compilato il preventivo del sesto sprint.
Entrambi sono consultabili nel documento Piano di Progetto nelle sezioni corrispondenti.

\subsubsection{Retrospettiva}
Data anche la lunghezza del quinto periodo il gruppo dopo aver confermato il consuntivo orario della quinta iterazione ha deciso di confrontarsi sulle
attività svolte e soprattutto sulla divisione dei carichi fra i vari membri visto il periodo degli esami e la consegna RTB imminente.
Il gruppo si è quindi organizzato per correggere il documento di analisi dei requisiti e revisionare gli altri documenti.

\subsubsection{Issue}
Durante l'incontro non è stato necessario creare nuove issue in quanto erano state appena create e assegnate in seguito all'incontro con la proponente.

\subsection{Decisioni Prese}
Segue un elenco delle decisioni prese durante la riunione:
\begin{itemize}
    \item   Assegnazione dei ruoli per il quinto sprint:\\
            Responsabile: Alfredo Rubino;\\
            Amministratore: Riccardo Piva;\\
            Analisti: Manuel Gusella, Leonardo Trolese;\\
            Verificatori: Giovanni Cristellon, Federico Pivetta, Uncas Peruzzi.\\
    \item Correggere UC già presenti nel documento di analisi dei requisiti;
    \item Aggiungere UC mancanti o precedentemente poco dettagliati al documento di analisi dei requisiti;
    \item Terminare e verificare la documentazione in vista della RTB.
\end{itemize}
\begin{center}
\begin{tabular}{|>{\hspace{20pt}}c<{\hspace{20pt}}|>{\hspace{20pt}}c<{\hspace{20pt}}|}
	\hline
	    \textbf{Rif.Issue} & \textbf{Dettaglio Decisione}\\
	\hline
        \href{https://github.com/SevenBitsSwe/7BitsDocs/issues/123}{Issue \#123} & Sistemare documento di analisi dei requisiti \\
	\hline
\end{tabular}
\end{center}
\end{document}