\documentclass[10pt]{article}

\usepackage[utf8]{inputenc}
\usepackage{geometry}
\usepackage{tabularx}
\usepackage{graphicx}
\usepackage{hyperref}
\usepackage{array}  

%path per conversione in locale
%\graphicspath{{../../../images/}}

%path per quando caricare in repo
\graphicspath{{images/}}

%cambio misure della pagina
\geometry{a4paper,left=20mm,right=20mm,top=20mm}

\title{Verbale Interno del meeting in data 2025-01-08}
\date{A.A 2024/2025}

\renewcommand*\contentsname{Indice}
\begin{document}
%contenuti principali
\maketitle
\begin{center}
\includegraphics[width=0.25\textwidth]{LogoUnipd}\\
\includegraphics[width=0.25\textwidth]{Sevenbitslogo}\\
sevenbits.swe.unipd@gmail.com\\
\vspace{2mm}

\textbf{Registro modifiche}\\
\vspace{2mm}
\begin{tabularx}{\textwidth}{|l|l|l|l|X|}
\hline
\textbf{Versione} & \textbf{Data} & \textbf{Autore} & \textbf{Verificatore} & \textbf{Descrizione} \\
\hline
1.0.0 & 2025-01-10 & Leonardo Trolese & Alfredo Rubino & Approvazione verbale \\
\hline
0.1.0 & 2025-01-10 & Leonardo Trolese & Giovanni Cristellon & Redazione del verbale \\
\hline
\end{tabularx}
\end{center}

\newpage
\tableofcontents
\newpage
\section{2025-01-08}
\subsection{Durata e partecipanti}
\begin{itemize}
\item Ora: 17:30 - 18:40;
\item Partecipanti: 	
	\begin{itemize}
		\item Gusella Manuel;
		\item Peruzzi Uncas;
		\item Piva Riccardo;
		\item Pivetta Federico;
		\item Rubino Alfredo;
		\item Cristellon Giovanni;
		\item Trolese Leonardo.
	\end{itemize}
\item Piattaforma: Discord (online)
\end{itemize}

\subsection{Sintesi}
Il gruppo si è riunito per pianificare e discutere lo sprint successivo (quinto). Nel corso della riunione è stato analizzato il consuntivo
del periodo precedente, sono stati stabiliti i ruoli per l'iterazione a venire e ne è stato stimato il preventivo. È seguita poi una discussione
sulle attività da svolgere nel quinto sprint, con particolare attenzione al completamento del Piano di Qualifica.

\subsubsection{Assegnazione ruoli}
Il gruppo ha deciso di assegnare i nuovi ruoli evitando che lo stesso membro ricoprisse nuovamente un ruolo già precedentemente assegnato.
È stata inoltre presa la decisione di concentrare gli sforzi del gruppo nell'ambito di analisi e verifica, poiché, in vista della consegna RTB,
è necessario rivedere e verificare nuovamente i documenti completati e terminare la redazione di alcuni documenti ancora incompleti.

\subsubsection{Preventivo e Consuntivo delle ore}
Durante la riunione il gruppo ha confermato il consuntivo delle ore relative allo svolgimento del quarto sprint e ha quindi formulato il 
preventivo orario dell'imminente quinto periodo del progetto. Entrambi sono consultabili nel documento Piano di Progetto nelle sezioni
corrispondenti.

\subsubsection{Retrospettiva}
Dopo aver confermato il consuntivo orario della quarta iterazione, i componenti hanno anche condiviso quanto realizzato durante tale periodo
con il resto del gruppo. Si è discusso degli aspetti positivi e negativi dello sprint appena concluso ed è stato stabilito quali strategie e
miglioramenti adottare per l'iterazione successiva. In particolare, è emersa la mancata conclusione della documentazione per la RTB, 
originariamente prevista per il quarto sprint, che il gruppo si impegna a completare entro il quinto sprint con gli ultimi dettagli mancanti.

\subsubsection{Issue}
Durante l'incontro sono state definite delle issue da completare nello sprint successivo, con particolare attenzione al completamento
di tutta la documentazione necessaria per la RTB.

\subsection{Decisioni Prese}
Segue un elenco delle decisioni prese durante la riunione:
\begin{itemize}
    \item   Assegnazione dei ruoli per il quinto sprint:\\
            Responsabile: Riccardo Piva;\\
            Amministratore: Manuel Gusella;\\
            Analisti: Federico Pivetta, Leonardo Trolese;\\
            Verificatori: Alfredo Rubino, Giovanni Cristellon, Uncas Peruzzi, Leonardo Trolese.\\
    \item Aggiungere i test da implementare al documento Piano di Qualifica;
    \item Concludere il cruscotto delle metriche nel documento Piano di Qualifica;
    \item Associare ad ogni processo le metriche relative nel documento Norme di Progetto;
    \item Terminare e verificare la documentazione in vista della RTB.
\end{itemize}
\begin{center}
\begin{tabular}{|>{\hspace{20pt}}c<{\hspace{20pt}}|>{\hspace{20pt}}c<{\hspace{20pt}}|}
	\hline
	    \textbf{Rif.Issue} & \textbf{Dettaglio Decisione}\\
	\hline
		\href{https://github.com/SevenBitsSwe/7BitsDocs/issues/102}{Issue \#102} & Aggiungere sottosezioni relative alle metriche nelle sezioni delle NdP\\
	\hline
        \href{https://github.com/SevenBitsSwe/7BitsDocs/issues/104}{Issue \#104} & Terminare redazione cruscotto metriche del PdQ\\
	\hline
        \href{https://github.com/SevenBitsSwe/7BitsDocs/issues/114}{Issue \#114} & Aggiungere test nel PdQ\\
	\hline
        \href{https://github.com/SevenBitsSwe/7BitsDocs/issues/105}{Issue \#105} & Verifica finale Analisi dei Requisiti\\
	\hline
        \href{https://github.com/SevenBitsSwe/7BitsDocs/issues/106}{Issue \#106} & Verifica finale Piano di Progetto\\
	\hline
        \href{https://github.com/SevenBitsSwe/7BitsDocs/issues/107}{Issue \#107} & Verifica finale Norme di Progetto\\
	\hline
        \href{https://github.com/SevenBitsSwe/7BitsDocs/issues/108}{Issue \#108} & Verifica finale Glossario\\
	\hline
        \href{https://github.com/SevenBitsSwe/7BitsDocs/issues/109}{Issue \#109} & Verifica finale Piano di Qualifica\\
	\hline
\end{tabular}
\end{center}
\end{document}
