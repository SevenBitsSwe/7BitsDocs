\documentclass[12pt]{article}

\usepackage[utf8]{inputenc}
\usepackage{tabularx}
\usepackage{hyperref}
\usepackage{array}  
\usepackage{graphicx} % Per inserire immagini (loghi)
\usepackage{geometry} % Per personalizzare i margini
\usepackage{fancyhdr} % Per gestire intestazioni e piè di pagina
\usepackage{tikz}
\usepackage{anyfontsize}
\usepackage[table,xcdraw]{xcolor}
\usepackage{tabularx, etoolbox} % aggiungi etoolbox per condizioni% Load xcolor
\usepackage{eso-pic} % Per aggiungere elementi grafici su tutte le pagine

\newcommand\version{0.3.2} %aggiunta versione come variabile



\graphicspath{{images/}}
%\graphicspath{{../images/}}




%cambio misure della pagina
\geometry{a4paper,left=20mm,right=20mm,top=20mm}
%ebdfc7
\definecolor{colorePie}{HTML}{ebdfc7}


\pagestyle{fancy}
\fancyhf{}
\renewcommand{\headrulewidth}{0.4pt}
\lhead{
    \parbox[c]{1cm}{\includegraphics[width=1.1cm]{Sevenbitslogo.png}}
}
\rhead{\textcolor[HTML]{9e978a}{ Piano di Progetto v\version}
}
\setlength{\headheight}{25pt}
\cfoot{\thepage}




\renewcommand*\contentsname{Indice}

\begin{document}

% Pagina del titolo
\begin{titlepage}
    \setcounter{page}{0}
    \centering
    % Inserisci il logo del gruppo (modifica il percorso dell'immagine)
    \includegraphics[width=7.2cm]{Sevenbitslogo.png} \\[2cm] 
    
    % Titolo
     {\fontsize{40}{40}\bfseries Piano di Progetto}\selectfont \\[3.9em]
    
    % Sottotitolo
    % Email del gruppo
    {\large sevenbits.swe.unipd@gmail.com} \\[3em]
    
    % Spazio per il logo dell'università
    \hfill
    
        
    \AddToShipoutPictureBG{ % Imposta il triangolo con logo
        \ifnum\value{page}=0
        \begin{tikzpicture}[overlay]
        
            % Definisce un triangolo blu in basso a destra
            \fill[colorePie] 
                (current page.south east) -- ++(-9cm,0) -- ++(9cm,9cm);
            
            % Inserisce il logo all'interno del triangolo
            \node[anchor=south east, xshift=-0.3cm, yshift=0.3cm] at (current page.south east) {
                \includegraphics[width=4.5cm]{LogoUnipd.png}
            };
        \end{tikzpicture}
        \fi
    }
        
    

    \vfill % Aggiunge spazio verticale per centrare il contenuto
\end{titlepage}
\newpage
\clearpage

\setcounter{page}{1}
%registro modifiche
\begin{center}
\textbf{Registro modifiche}\\
\vspace{2mm}
\begin{tabularx}{\textwidth}{|l|l|l|l|X|}
\hline
\textbf{Versione} & \textbf{Data} & \textbf{Autore} & \textbf{Verificatore} & \textbf{Descrizione}\\
\hline
0.3.2 & 2024-11-28 & Manuel Gusella & Giovanni Cristellon & Fine Stesura primo sprint$_G$ \ref{primo-sprint$_G$} e aggiunta immagini\\
\hline
0.3.1 & 2024-11-22 & Manuel Gusella & Riccardo Piva & Stesura sottosezione \ref{struttura-espositiva}\\
\hline
0.3.0 & 2024-11-22 & Manuel Gusella & Riccardo Piva & Stesura iniziale della sezione \ref{modello-sviluppo} Modello di sviluppo, preventivo e consuntivo\\
\hline
0.2.1 & 2024-11-17 & Manuel Gusella & Federico Pivetta & Modifiche di stile delle liste e dei link nel verbale \\
\hline
0.2.0 & 2024-11-15 & Manuel Gusella & Federico Pivetta & Stesura iniziale sezione Pianificazione\\
\hline
0.1.0 & 2024-11-14  & Manuel Gusella & Riccardo Piva & Stesura sezione Introduzione\\
\hline

\end{tabularx}   
\end{center}
\newpage
\tableofcontents
\listoffigures %elenco delle figure sarà da usare per ogni immagine

\newpage
\section{Introduzione}
\subsection{Scopo del documento}
Questo documento ha lo scopo di definire in modo chiaro le modalità con cui le attività saranno svolte dai membri del gruppo per la realizzazione del progetto.\\
Saranno trattati in dettagli i seguenti temi:
\begin{itemize}
    \item [-] Analisi dei rischi;
    \item [-] Organizzazione delle attività nei singoli periodi;
    \item [-] Suddivisione dei ruoli tra i membri del gruppo;
    \item [-] Stima dei costi e delle risorse nelle varie iterazioni.
\end{itemize}

\subsection{Scopo del capitolato}
Il capitolato C4 ha come scopo la realizzazione di una dashboard "amministrativa" in grado di proporre ad ogni utente degli annunci personalizzati tramite l'utilizzo di LLM$_G$.\\ La dashboard deve mostrare una mappa con ipotetici utenti generati virtualmente, che poi verranno rappresentati come punti in movimento, e ogni volta che un utente passa per un'area di interesse appare un annuncio generato tramite IA$_G$.
\subsection{Glossario}
Al fine di evitare ambiguità relative alla terminologia utilizzata all'interno del documento, è presente il \textit{Glossario v}, in cui vengono riportate tutte le definizione delle parole con un significato specifico. Questi termini veranno marcati con una$_G$ a pedice, mentre i termini composti, oltre alla $_G$ a pedice, saranno uniti da un "-" come segue: termine-composto$_G$. 
\subsection{Riferimenti}
\subsubsection{Informativi}
Slide del corso di Ingegneria del Software:
\begin{itemize}
  \item [-] Modelli di sviluppo software:\\ \textcolor{blue}{\texttt{\url{https://www.math.unipd.it/~tullio/IS-1/2024/Dispense/T03.pdf}}}
    \item [-] Gestione di Progetto:\\ \textcolor{blue}{\texttt{\url{https://www.math.unipd.it/~tullio/IS-1/2024/Dispense/T04.pdf}}}
\end{itemize}

\subsubsection{Normativi}
\begin{itemize}
	\item [-] \textit{Norme di Progetto v1.0.0}
    \item [-] Documento e presentazione del capitolato C4 - NearYou - Smart custom advertising platform:\\
    \textcolor{blue}{\texttt{\url{https://www.math.unipd.it/~tullio/IS-1/2024/Progetto/C4.pdf}}}\\
    \textcolor{blue}{\texttt{\url{https://www.math.unipd.it/~tullio/IS-1/2024/Progetto/C4p.pdf}}}
    
    \item [-] Regolamento del progetto didattico\\ \textcolor{blue}{\texttt{\url{https://www.math.unipd.it/~tullio/IS-1/2024/Dispense/PD1.pdf}}}
\end{itemize}
\newpage
\section{Analisi dei Rischi}
\newpage
\section{Modello di sviluppo, preventivo e consuntivo} 
\label{modello-sviluppo}
\subsection{Modello di sviluppo}
Il team ha deciso di utilizzare principalmente il framework$_G$ Scrum come modello di sviluppo.\\
Scrum, essendo un modello di sviluppo agile, permette di poter avanzare con il progetto tramite periodi chiamati sprint$_G$, della durata di 2-3 settimane, alla fine dei quali consente di avere una baseline$_G$ di prodotto da poter mostrare al proponente$_G$.
\subsubsection{Vantaggi del Modello utilizzato}
Il framework$_G$ Scrum fornisce numerosi vantaggi per lo svolgimento di progetti di gruppo, soprattutto per lo svolgimento del nostro progetto. Alcuni principali vantaggi sono:
\begin{itemize}
  \item \textbf{Riduzione dei rischi:} il framework$_G$ Scrum , visto la breve durata degli sprint$_G$, permette di minimizzare lo sviluppo e la gravità di rischi nello svolgimento del progetto;
  \item \textbf{Flessibilità e Adattabilità:} questo modello di sviluppo permette una risposta veloce e tempestiva ai cambiamenti nei requisiti da parte degli stakeholders$_G$;
  \item \textbf{Consegna incrementale:} gli approcci di tipo agile permettono di effettuare rilasci frequenti del progetto permettendo al proponente di avere sempre una baseline$_G$ di prodotto da poter valutare e fornire un feedback$_G$;
  \item \textbf{Collaborazione e Comunicazione:} il framework$_G$ Scrum promuove una comunicazione aperta e costante tra i membri del team e i proponenti, migliorando la comprensione tra le due parti;
  \item \textbf{Miglioramento continuo:} le retrospettive permettono di portare un miglioramento continuo, permettendo al team di poter identificare e sistemare eventuali problemi riscontrati durante lo svolgimento di uno sprint$_G$.
\end{itemize}

\subsection{Preventivo}
Stima delle risorse necessarie per svolgere e terminare le attività pianificate. Include una previsione del consumo di risorse, dovendo tener conto delle limitazioni orarie ed economiche sostenuti dal team.
\subsection{Consuntivo}
Riporta le risorse effettivamente utilizzate per lo svolgimento delle attività proposte nel preventivo e se tali attività sono state portate al termine.\\
Questo confronto ci permette di identificare eventuali scostamenti rispetto al piano iniziale e reagire di conseguenza, portando un miglioramento continuo.
\newpage
\subsection{Struttura espositiva dei periodi} \label{struttura-espositiva}
Ogni periodo di avanzamento verrà esposto nella seguente configurazione:
\begin{enumerate}
 \item \textbf{Durata:} Esprime la durata del periodo di sprint$_G$ scritta in "Dal data-inizio al data-fine" con data-inizio e data-fine espresse in aaaa-mm-gg.
 \item \textbf{Obiettivi:} Lista degli obiettivi da raggiungere entro fine sprint$_G$.
 \item \textbf{Rischi incontrati:} Lista di rischi imbattuti durante il periodo. Nel caso  della verifica di eventuali rischi sarà presente anche una sezione di come il team li ha risolti e che impatto hanno avuto sulle attività pianificate. 
 \item \textbf{Ruoli:} Tabella con i ruoli occupati dai componenti del team durante il periodo.
 \item \textbf{Preventivo:} Espone le informazioni di ore e costi preventivati per il periodo di sprint$_G$.
 \item \textbf{Consuntivo:} Espone le informazioni di ore e costi effettivi per il periodo di sprint$_G$.
\end{enumerate}
\newpage
\section{Pianificazione}
\subsection{Requirements and Technology Baseline RTB$_G$}
\subsubsection{Primo sprint$_G$}
\label{primo-sprint$_G$}
\begin{enumerate}
\item\textbf{Durata:} Dal 2024-11-11 al 2024-11-25
\item \textbf{Obiettivi:}
\begin{itemize}
\item [-] Studio delle tecnologie consigliate dal proponente per capire quale utilizzare nella realizzazione del progetto;
\item [-] Inizio stesura del Piano di Progetto;
\item [-] Inizio redazione dell'Analisi dei Requisiti;
\item [-] Inizio scrittura delle Norme di Progetto;
\end{itemize}
\item \textbf{Rischi incontrati:}
\item \textbf{Ruoli:}\\
Questi sono i ruoli assegnati per membro in questo primo sprint$_G$.\\
\vspace{2mm}
\begin{center}
\begin{tabular}{|c|c|}
\hline
\textbf{Ruolo} & \textbf{Membro}\\
\hline
Responsabile & Manuel Gusella\\
\hline
Amministratore & Uncas Peruzzi\\ 
\hline
Analista & Giovanni Cristellon\\
& Leonardo Trolese\\
& Alfredo Rubino\\
\hline
Progettista & \\
\hline
Programmatore & \\
\hline
Verificatore & Federico Pivetta\\
& Riccardo Piva\\
\hline
\end{tabular}
\end{center}
\item \textbf{Preventivo ore:}
\item \textbf{Consuntivo ore:}
\begin{figure}[ht]
	\centering
	\includegraphics[width=0.6\linewidth]{Consuntivo-table-membri-1.png}
	\caption{Consuntivo orario per membro - primo sprint$_G$}
	\label{fig:Consuntivo orario per membro - primo sprint$_G$}
\end{figure}
\begin{figure}[ht]
	\centering
	\includegraphics[width=0.4\linewidth]{Consuntivo-ore-ruoli-torta-1.png}
	\caption{Diagramma circolare della partizione delle ore per ruolo - primo sprint$_G$ }
	\label{fig:Diagramma circolare della partizione delle ore per ruolo - primo sprint$_G$}
\end{figure}
\begin{figure}[ht]
	\centering
	\includegraphics[width=0.6\linewidth]{Consuntivo-table-ruoli-1.png}
	\caption{Consuntivo orario e costi per ruolo - primo sprint$_G$}
	\label{fig:Consuntivo orario e costi per ruolo - primo sprint$_G$}
\end{figure}
\begin{figure}[ht]
	\centering
	\includegraphics[width=0.5\linewidth]{Consuntivo-ore-tot-torta-1.png}
	\caption{Diagramma circolare delle ore rimanenti - primo sprint$_G$ }
	\label{fig:Diagramma circolare delle ore rimanenti - primo sprint$_G$}
\end{figure}
\begin{figure}[ht]
	\centering
	\includegraphics[width=0.5\linewidth]{Consuntivo-costi-tot-torta-1.png}
	\caption{Diagramma circolare dei costi avvenuti - primo sprint$_G$ }
	\label{fig:Diagramma circolare dei costi avvenuti - primo sprint$_G$}
\end{figure}

\end{enumerate}

\end{document}
