\documentclass[12pt]{article}

\usepackage[utf8]{inputenc}
\usepackage{tabularx}
\usepackage{hyperref}
\usepackage{array}  
\usepackage{graphicx} % Per inserire immagini (loghi)
\usepackage{geometry} % Per personalizzare i margini
\usepackage{fancyhdr} % Per gestire intestazioni e piè di pagina
\usepackage{tikz}
\usepackage{anyfontsize}
\usepackage[table,xcdraw]{xcolor}
\usepackage{tabularx, etoolbox} % aggiungi etoolbox per condizioni% Load xcolor
\usepackage{eso-pic} % Per aggiungere elementi grafici su tutte le pagine

%\graphicspath{{images/}}
\graphicspath{{../images/}}




%cambio misure della pagina
\geometry{a4paper,left=20mm,right=20mm,top=20mm}
%ebdfc7
\definecolor{colorePie}{HTML}{ebdfc7}


\pagestyle{fancy}
\fancyhf{}
\renewcommand{\headrulewidth}{0.4pt}
\lhead{
    \parbox[c]{1cm}{\includegraphics[width=1.1cm]{Sevenbitslogo.png}}
}
\rhead{\textcolor[HTML]{9e978a}{ Piano di Progetto v0.1.0}
}
\setlength{\headheight}{25pt}
\cfoot{\thepage}




\renewcommand*\contentsname{Indice}

\begin{document}

% Pagina del titolo
\begin{titlepage}
    \setcounter{page}{0}
    \centering
    % Inserisci il logo del gruppo (modifica il percorso dell'immagine)
    \includegraphics[width=7.2cm]{Sevenbitslogo.png} \\[2cm] 
    
    % Titolo
     {\fontsize{40}{40}\bfseries Piano di Progetto}\selectfont \\[3.9em]
    
    % Sottotitolo
    % Email del gruppo
    {\large sevenbits.swe.unipd@gmail.com} \\[3em]
    
    % Spazio per il logo dell'università
    \hfill
    
        
    \AddToShipoutPictureBG{ % Imposta il triangolo con logo
        \ifnum\value{page}=0
        \begin{tikzpicture}[overlay]
        
            % Definisce un triangolo blu in basso a destra
            \fill[colorePie] 
                (current page.south east) -- ++(-9cm,0) -- ++(9cm,9cm);
            
            % Inserisce il logo all'interno del triangolo
            \node[anchor=south east, xshift=-0.3cm, yshift=0.3cm] at (current page.south east) {
                \includegraphics[width=4.5cm]{LogoUnipd.png}
            };
        \end{tikzpicture}
        \fi
    }
        
    

    \vfill % Aggiunge spazio verticale per centrare il contenuto
\end{titlepage}
\newpage
\clearpage

\setcounter{page}{1}
%registro modifiche
\begin{center}
\textbf{Registro modifiche}\\
\vspace{2mm}
\begin{tabular}{|l|l|l|l|l|l|}
\hline
\textbf{Versione} & \textbf{Data} & \textbf{Autore} & \textbf{Verificatore} & \textbf{Descrizione}\\
\hline
0.1.0 & 2024-11-14  & Manuel Gusella &   & Stesura sezione Introduzione\\
\hline

\end{tabular}   
\end{center}
\newpage
\tableofcontents
\newpage
\section{Introduzione}
\subsection{Scopo del documento}
Questo documento ha lo scopo di definire in modo chiaro le modalità con cui le attività saranno svolte dai membri del gruppo per la realizzazione del progetto.\\
Saranno trattati in dettagli i seguenti temi:
\begin{itemize}
    \item Analisi dei rischi;
    \item Organizzazione delle attività nei singoli periodi;
    \item Suddivisione dei ruoli tra i membri del gruppo;
    \item Stima dei costi e delle risorse nelle varie iterazioni.
\end{itemize}

\subsection{Scopo del capitolato}
Il capitolato C4 ha come scopo la realizzazione di una dashboard "amministrativa" in grado di proporre ad ogni utente degli annunci personalizzati tramite l'utilizzo di \textit{LLM}$_G$.\\ La dashboard deve mostrare una mappa con ipotetici utenti generati virtualmente, che poi verranno rappresentati come punti in movimento, e ogni volta che un utente passa per un'area di interesse appare un annuncio generato tramite \textit{IA}$_G$.
\subsection{Glossario}
Al fine di evitare ambiguità relative alla terminologia utilizzata all'interno del documento, è presente il \textit{Glossario v}, in cui vengono riportate tutte le definizione delle parole con un significato specifico. Queste parole veranno scritte in \textit{corsivo} e marcati con una$_G$ a pedice. 
\subsection{Riferimenti}
\subsubsection{Informativi}
Slide del corso di Ingegneria del Software:
\begin{itemize}
    \item Gestione di Progetto:\\ \url{https://www.math.unipd.it/~tullio/IS-1/2024/Dispense/T04.pdf}
\end{itemize}

\subsubsection{Normativi}
\begin{itemize}
\item Norme di Progetto v
    \item Documento e presentazione del capitolato C4 - NearYou - Smart custom advertising platform:
    \begin{itemize}
        \item \url{https://www.math.unipd.it/~tullio/IS-1/2024/Progetto/C4.pdf}
        \item \url{https://www.math.unipd.it/~tullio/IS-1/2024/Progetto/C4p.pdf}
    \end{itemize}
    \item Regolamento del progetto didattico\\ \url{https://www.math.unipd.it/~tullio/IS-1/2024/Dispense/PD1.pdf}
\end{itemize}
\section{Analisi dei Rischi}

\end{document}
