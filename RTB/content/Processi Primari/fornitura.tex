\subsection{Fornitura}

\subsubsection{Descrizione e Scopo}
Come stabilito dallo standard ISO/IEC 12207:1995, il processo di fornitura definisce un insieme di linee guida necessario per una buona comunicazione tra fornitore e proponente. Il processo di fornitura si occupa di controllare e coordinare tutte le attività svolte dal gruppo, dalla comprensione dei requisiti fino alla consegna, per garantire che il prodotto finale soddisfi le esigenze concordate con la propontente.\\

\subsubsection{Attività}
Il processo di fornitura, come stabilito dallo standard ISO/IEC 12207:1995, si compone delle seguenti attività :
\begin{enumerate}
    \item \textbf{Avvio}: Composto dall'identificazione e comprensione delle richieste della proponente, con successiva verifica della fattibilità tecnologica di quest'ultime;
    \item \textbf{Preparazione dell'offerta}: Composta dall'elaborazione della proposta in grado di soddisfare le richieste della proponente, che dettagli i requisiti, i tempi, i costi e le condizioni contrattuali;
    \item \textbf{Contrattazione}: Composta dalla collaborazione tra fornitore e proponente per finalizzare i punti cardine della proposta;
    \item \textbf{Pianificazione}: Composta dalla pianificazione delle attività necessarie per soddisfare i requisiti della proponente, seguita da una suddivisione delle ore produttive disponibili ed una stima dei costi;
    \item \textbf{Esecuzione}: Composta dalla pianificazione e dallo sviluppo del prodotto in conformità ai requisiti concordati, insieme ad un monitoraggio continuo delle attività;
    \item \textbf{Revisione}: Composta dalla verifica periodica del progresso rispetto ai criteri definiti dal contratto;
    \item \textbf{Consegna}: Composta dalla consegna del prodotto software alla proponente, accompagnato dalla documentazione finale.
\end{enumerate}

\subsubsection{Rapporti con l'azienda proponente}
L'azienda proponente \textit{SyncLab}$_G$ si è resa disponibile mediante diversi canali tra cui: e-mail, Discord e Google Meet ad una comunicazione frequente con il gruppo SevenBits, così da risolvere tempestivamente eventuali domande o dubbi che possono emergere durante lo svolgimento del progetto.\\
Durante la prima riunione organizzativa con l'azienda, è stata definita l'organizzazione dei periodi di \textit{sprint}$_G$, stabilendo una durata di due settimane per ciascun ciclo. Al termine di ogni \textit{sprint}$_G$ avviene un incontro \textit{SAL}$_G$ (Stato di Avanzamento Lavori), dove verranno analizzati i risultati del lavoro svolto e si procederà con una sprint \textit{review}$_G$. Inoltre, tra un \textit{SAL}$_G$ e l'altro, è stato concordato un incontro intermedio per monitorare i progressi raggiunti e rispondere ad eventuali quesiti emersi.\\
Ogni incontro con l'azienda viene formalizzato attraverso un verbale esterno. Tale verbale è successivamente sottoposto alla proponente per la validazione mediante firma, in modo da ottenere un'approvazione formale del resoconto delle discussioni svolte durante la riunione.\\

\subsubsection{Documentazione fornita}
Di seguito sono elencati i documenti che il gruppo si impegna a consegnare ai Committenti, Prof. Tullio Vardanega e Prof. Riccardo Cardin, nonché all'azienda proponente:\\

    \paragraph{Analisi dei Requisiti}
    \'E un documento essenziale per lo sviluppo del prodotto software, che include la descrizione degli attori coinvolti, dei casi d’uso e l’elenco dei requisiti, suddivisi in requisiti funzionali, di qualità, di vincolo e prestazionali.\\

    \paragraph{Piano di Progetto}
    Il Piano di Progetto è un documento che ha lo scopo di definire in modo chiaro le modalità con cui ogni membro del gruppo svolgerà le attività per la realizzazione del progetto. Include l'analisi dei rischi, la pianificazione delle attività, la suddivisione dei ruoli e la stima di costi e risorse.\\

    \paragraph{Piano di Qualifica}
    Il Piano di Qualifica è un documento che ha l'obiettivo di garantire la qualità del prodotto e dei processi durante l'intero ciclo di vita del progetto, per questo motivo sarà aggiornato nel tempo per riflettere eventuali modifiche e i risultati delle verifiche effettuate. Include le sezioni sulla qualità di processo, sulla qualità di prodotto, le modalità di testing e il cruscotto di valutazione delle qualità.\\

    \paragraph{Glossario}
    Il Glossario è un documento che raccoglie dei termini specifici e le loro definizioni chiare e concise. Il suo scopo è quello di facilitare la comprensione dei concetti chiave presenti nei vari documenti redatti.\\

    \paragraph{Lettera di Presentazione}
    La Lettera di Presentazione è un documento che accompagna la consegna del prodotto software e della relativa documentazione durante le fasi di revisione di progetto. Il contenuto di questo documento comprende un link alla pagina web che contiene tutta la documentazione fin'ora prodotta ed un preventivo aggiornato rispetto a quello presentato alla revisione precedente.\\

\subsubsection{Strumenti}
 Gli strumenti utilizzati per la gestione del processo di fornitura sono i seguenti :
 \begin{itemize}
    \item \textbf{Google Meet e Discord}: servizi che permettono di effettuare videochiamate, utilizzati dal team per le discussioni sincrone e asincrone con la proponente;
    \item \textbf{Google Sheets}: servizio che permette la creazione di fogli di calcolo, utilizzato dal gruppo per la rendicontazione delle ore produttive impiegate durante ogni \textit{sprint}$_G$;
    \item \textbf{Canva}: piattaforma che permette la creazione di presentazioni multimediali, utilizzata per la realizzazione dei diari di bordo;
    \item \textbf{Draw.io}: software utilizzato per creare diagrammi e grafici di vario tipo, in particolare è stato impiegato per realizzare diagrammi UML, come quelli dei casi d’uso;
 \end{itemize}