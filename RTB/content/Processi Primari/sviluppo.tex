\subsection{Sviluppo}

\subsubsection{Descrizione e Scopo}
Il processo di sviluppo rappresenta l’insieme delle attività necessarie per realizzare il prodotto software, garantendo il rispetto dei requisiti e delle scadenze concordate con il proponente. Questo processo si articola in diverse fasi fondamentali, quali l'analisi dei requisiti, la progettazione, la codifica, l'integrazione, e la verifica, assicurando che ogni fase contribuisca al raggiungimento degli obiettivi prefissati.
Le linee guida descritte in questa sezione sono volte a strutturare il lavoro in modo chiaro e uniforme, promuovendo la qualità del prodotto finale. Seguendo standard solidi e definiti, nel caso specifico l'ISO/IEC 12207:1995, si crea un ambiente di lavoro orientato a garantire la coerenza nei metodi utilizzati e il rispetto delle aspettative.
L'obiettivo è consegnare un prodotto software di alta qualità, che soddisfi le esigenze richieste, rispettando le tempistiche e garantendo il successo del progetto.\\

\subsubsection{Analisi dei Requisiti}
L’analisi dei requisiti è la prima fase cruciale del processo di sviluppo software.Lo scopo principale di questa fase è definire con chiarezza le funzionalità e le caratteristiche che il sistema dovrà offrire, in base alle necessità degli utenti. Per raggiungere tale obiettivo, è essenziale stabilire una comunicazione efficace con il proponente, assicurandosi che tutte le esigenze siano documentate e validate. Questo processo permette di chiarire gli obiettivi del prodotto, identificare i vincoli operativi e fornire ai Progettisti le informazioni necessarie per sviluppare un’architettura coerente e un design adeguato.
Il risultato di questa attività è formalizzato nel documento \textit{Analisi\_dei\_requisiti.pdf}, redatto dagli Analisti, che contiene una descrizione dettagliata degli obiettivi del prodotto, delle funzionalità previste, delle caratteristiche degli utenti e delle tecnologie coinvolte. Include inoltre una sezione dedicata ai casi d’uso, che descrivono le interazioni tra gli attori esterni e il sistema.
Infine, l’analisi dei requisiti contribuisce a migliorare la comunicazione tra tutti gli stakeholder, agevola la pianificazione del progetto in termini di tempistiche e costi e fornisce un riferimento chiaro per le attività di verifica e test.\\

\subsubsection{Progettazione}

\subsubsection{Codifica}

\subsubsection{Integrazione}

\subsubsection{ Verifica}
