\subsection{Documentazione}

\subsubsection{Descrizione e Scopo}
La documentazione è l'insieme delle informazioni che accompagna lo sviluppo di un prodotto software, svolge un ruolo essenziale nella descrizione del prodotto per coloro che lo realizzano, lo distribuiscono e lo utilizzano.
Il suo scopo principale è quello di semplificare il lavoro dei membri del team durante l'intero ciclo di vita del progetto, monitorando tutti i processi e le attività coinvolte. Questo permette di migliorare il risultato finale e semplifica notevolmente la manutenzione.\\

\subsubsection{Lista documenti}
I documenti prodotti nel contesto della realizzazione del progetto sono:
\begin{itemize}
    \item [-] \textit{Analisi\_dei\_requisiti.pdf}
    \item [-] \textit{Glossario.pdf}
    \item [-] \textit{Norme\_di\_progetto.pdf}
    \item [-] \textit{Piano\_di\_progetto.pdf}
    \item [-] \textit{Piano\_di\_qualifica.pdf}
    \item [-] \textit{Verbali esterni}
    \item [-] \textit{Verbali interni}
\end{itemize}

\subsubsection{Ciclo di vita documenti e Versionamento}
I documenti seguono due workflow distinti a seconda che si tratti di verbali (interni o esterni) oppure di documenti più corposi, e le versioni delle modifiche successive apportate ai documenti seguono il sistema di versionamento indicato di seguito.\\

    \paragraph{Versionamento dei documenti}
    Il sistema adottato dal team per il versionamento dei documenti è il sistema di \textbf{versionamento semantico}: \textbf{x.y.z}, in cui ogni numero (x, y, z) ha un significato specifico:\\
    \begin{itemize}
    \item "x" indica la versione maggiore, incrementata per cambiamenti incompatibili con versioni precedenti;
    \item "y" rappresenta la versione minore, usata per aggiungere informazioni compatibili;
    \item "z" è la versione di patch, aggiornata per correzioni di errori poco significativi e retrocompatibili.
    \end{itemize} 
    Questo sistema aiuta il team a comprendere velocemente l'impatto di un aggiornamento fatto ad un documento prodotto.\\

    \paragraph{Workflow verbali}
    I verbali seguono il seguente workflow:
    \begin{enumerate}
        \item Creazione del documento a partire da un template diverso a seconda che si tratti di un verbale interno o esterno (la versione iniziale del documento corrisponde a \textit{0.1.0});
        \item Compilazione dei campi della sezione \textit{Registro modifiche};
        \item Redazione del documento indicando la durata dell'incontro, i partecipanti (interni ed esterni) e la piattaforma utilizzata. Seguono la sintesi di quanto fatto e una descrizione di ciascuna 
        delle considerazioni fatte e successive decisioni prese;
        \item Nella sezione \textit{Decisioni prese} si compila una 
        tabella in cui ciascuna azione da intraprendere viene associata 
        ad una issue corrispondente;
        \item Creazione di una pull-requestdal branch \framebox{Verbali} al branch \framebox{Main};
        \item Verifica del verbale prodotto da parte del verificatore 
        indicato nel \textit{Registro modifiche} del documento stesso;
        \item Se ci sono correzioni o ulteriori modifiche da fare, queste 
        devono essere indicate a loro volta nella sezione \textit{Registro modifiche} per poi essere verificate;
        \item Solo se la verifica dà esito positivo si può passare alla fase di approvazione, anch'essa da verificare. \newline Da questo punto se il verbale è interno si passa all'ultimo passaggio mentre se il verbale è esterno va mandato all'azienda SyncLab perché venga firmato;
        \item Quando il documento è completo l'ultimo verificatore 
        chiude la pull-request ed esegue il merge nel branch \framebox{Main};
    \end{enumerate}

    \paragraph{Workflow altri documenti}
    Gli altri documenti seguono il seguente workflow:
    \begin{enumerate}
        \item Creazione di un branch dedicato esclusivamente alla redazione di un documento specifico;
        \item Creazione del documento a partire da un template comune a tutti i documenti (esclusi i verbali). La versione iniziale del documento corrisponde a \textit{0.1.0};
        \item Creazione di una draft pull request dal branch corrispondente a tale documento al branch \framebox{Main};
        \item Compilazione dei campi della sezione \textit{Registro modifiche};
        \item Redazione del documento o di una sua sezione;
        \item Verifica della documentazione prodotta da parte del verificatore indicato nel \textit{Registro modifiche} e associato alle modifiche effettuate in quella seduta di lavoro;
        \item Se ci sono correzioni o ulteriori modifiche da fare, queste 
        devono essere indicate a loro volta nella sezione \textit{Registro modifiche} per poi essere verificate a loro volta;
        \item Si ripetono le operazioni indicate dal punto \textit{3} al punto \textit{6} fino a quando il documento non è stato completato;
        \item Quando il documento è completo l'ultimo verificatore 
        chiude la draft pull request ed esegue il merge nel branch \framebox{Main}, dopodiché si procede ad eliminare il branch$_G$ dedicato;
    \end{enumerate}

\subsubsection{Template in \LaTeX}
Per la stesura dei documenti, viene utilizzato un template in formato \LaTeX. Questo template ha lo scopo di semplificare la redazione dei documenti, garantire la coerenza e risparmiare del tempo, in modo da rendere la produzione dei documenti più efficiente e professionale. Sono stati sviluppati tre diversi modelli di template:
\begin{itemize}
    \item Documenti ufficiali
    \item Verbale Interno
    \item Verbale Esterno
\end{itemize}

\subsubsection{Nomenclatura}
La nomenclatura dei documenti prevede l'unione del nome del file, utilizzando degli underscore (\_), ad esempio Piano\_di\_Qualifica.pdf.
Nel caso dei verbali, la nomenclatura prevede l'uso del nome "VerbaleInterno" o "VerbaleEsterno", seguito dalla data nel formato "YYYY-MM-DD", con un trattino (-) che li unisce, come nell'esempio "VerbaleEsterno-2024-11-10.pdf".\\

\subsubsection{Struttura documenti}

    \paragraph{Prima Pagina}
    \begin{itemize}
    \item \textbf{Logo Team}: situato in alto al centro
    \item \textbf{Titolo}: \begin{itemize}
            \item Nome del documento, qualora non sia un verbale
            \item Verbale Interno
            \item Verbale Esterno
            \end{itemize}
    \item\textbf{Sottotitolo}: nome del capitolato
    \item\textbf{Contatti}: l'email del team
    \item\textbf{Logo Università}: situato in basso a destra
    \end{itemize}

    \paragraph{Intestazione}
    Su ogni pagina del documento, eccetto la prima, si trova il logo del gruppo seguito dal titolo del documento e dalla sua versione.\\

    \paragraph{Registro modifiche}
    Il registro delle modifiche è una tabella dettagliata che tiene traccia di ogni modifica avvenuta al documento nel corso del tempo. È utile per tenere traccia dell’evoluzione del documento e per consentire a chiunque stia lavorando sul progetto di comprendere quali modifiche sono state apportate e quando.
    L’intestazione comprende:
    \begin{itemize}
    \item \textbf{Versione}: versione del documento;
    \item \textbf{Data}: data della modifica apportata;
    \item \textbf{Autore}: l’autore della modifica;
    \item \textbf{Verificatore}: l’autore della verifica;
    \item \textbf{Descrizione}: cosa è stato modificato o aggiunto al file;
    \end{itemize}

    \paragraph{Indice}
    Nella pagina successiva al registro delle modifiche è presente l’indice, che permette di facilitare la ricerca e la navigazione all’interno del documento. \\

    \paragraph{Verbali}
    I verbali sono dei documenti di sintesi di un incontro che sia interno al team o esterno con l'azienda, per questo motivo la loro struttura è diversa rispetto agli altri documenti ufficiali.\\
    I verbali hanno lo scopo di tenere traccia di chi ha partecipato agli incontri ed in particolare quali decisioni sono state prese.\\
    Sono composti da 2 macrosezioni:
    \begin{itemize}
    \item  \textbf{Durata e Partecipanti}: viene indicata la data di inizio e fine incontro e il luogo in cui si è svolto. A seguire, i nomi dei partecipanti del gruppo. Se il verbale è esterno si indicano anche i partecipanti dell'azienda SyncLab;
    \item  \textbf{Sintesi e Decisioni Prese}: riassunto degli argomenti trattati ed elenco delle decisioni prese durante il meeting, collegate alle issue corrispondenti tramite una tabella;
    \end{itemize}
    Per i verbali esterni è presente una sezione per la convalida del documento mediante una firma.\\

\subsubsection{Convenzioni stilistiche}
    \paragraph{Stile del testo}
    \begin{itemize}
        \item \textbf{Grassetto}: viene utilizzato per i titoli di sezioni/sottosezioni/paragrafi di un documento e per le definizioni di termini negli elenchi puntati.
        \item \textbf{Corsivo}: viene utilizzato per i nomi dei file e per i titoli delle sezioni.
        \item \textbf{Link}: sono collegamenti ipertestuali, consentono di accedere a risorse esterne o interne, come altre pagine, sezioni, immagini o file, con un semplice click.
        \item \textbf{Glossario}: i termini che si possono ritrovare nel glossario sono seguiti dall'apice.
    \end{itemize}

    \paragraph{Formato delle date}
    \'E stato adottato il formato "YYYY-MM-DD", ovvero:
    \begin{itemize}
        \item YYYY: anno con 4 cifre;
        \item MM: mese con 2 cifre;
        \item DD: giorno con 2 cifre;
    \end{itemize}

\subsubsection{Strumenti}
Il gruppo ha deciso di utilizzare i seguenti strumenti:
\begin{itemize}
    \item \textbf{Git}: strumento per il controllo di versione distribuito impiegato dal gruppo per gestire i documenti e le versioni successive di essi prodotte in maniera asincrona dai membri;
    \item \textbf{GitHub}: piattaforma di versionamento e repository per la documentazione. Permette la gestione del codice sorgente, il controllo delle modifiche tramite funzionalità come le issue e le pull request, facilitando la collaborazione all’interno di un team di sviluppo;
    \item \textbf{\LaTeX}: linguaggio scelto per la redazione dei documenti, spesso utilizzato in ambito accademico, scientifico e tecnico;
    \item \textbf{Overleaf}: servizio utilizzato dal gruppo per la creazione e modifica sincrona da parte di due o più componenti di file .tex usati per produrre documentazione;
\end{itemize}