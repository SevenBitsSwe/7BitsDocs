
\documentclass[10pt]{article}

\usepackage[utf8]{inputenc}
\usepackage{geometry}
\usepackage{tabularx}
\usepackage{graphicx}
\usepackage{hyperref}
\usepackage{array}

%path per conversione in locale
%\graphicspath{{../../../images/}}

%path per quando caricare in repo
\graphicspath{{images/}}

%cambio misure della pagina
\geometry{a4paper,left=20mm,right=20mm,top=20mm}

\title{Verbale Interno del meeting in data 2025-02-20}
\date{A.A 2024/2025}

\renewcommand*\contentsname{Indice}
\begin{document}
%contenuti principali
\maketitle
\begin{center}
\includegraphics[width=0.25\textwidth]{LogoUnipd}\\
\includegraphics[width=0.25\textwidth]{Sevenbitslogo}\\
sevenbits.swe.unipd@gmail.com\\
\vspace{2mm}

\textbf{Registro modifiche}\\
\vspace{2mm}
\begin{tabularx}{\textwidth}{|l|l|l|l|X|}
\hline
\textbf{Versione} & \textbf{Data} & \textbf{Autore} & \textbf{Verificatore} & \textbf{Descrizione} \\
\hline
1.0.0 & 2025-02-21 & Federico Pivetta & Giovanni Cristellon & Approvazione del verbale \\
\hline
0.1.0 & 2025-02-20 & Alfredo Rubino & Federico Pivetta & Redazione del verbale \\
\hline
\end{tabularx}
\end{center}

\newpage
\tableofcontents
\newpage
\section{2025-02-20}
\subsection{Durata e partecipanti}
\begin{itemize}
\item Ora: 14:00 - 15:30;
\item Partecipanti:
	\begin{itemize}
    	\item Giovanni Cristellon;
		\item Gusella Manuel;
		\item Peruzzi Uncas;
		\item Piva Riccardo;
		\item Pivetta Federico;
		\item Rubino Alfredo;
		\item Trolese Leonardo.
	\end{itemize}
\item Piattaforma: Discord (online)
\end{itemize}

\subsection{Sintesi}
Il gruppo si è riunito per concludere lo sprint. Durante l'incontro è stato prodotto l'elaborato da presentare nella seconda parte dell'RTB, durante l'incontro con il prof. Vardanega.\\
Nel corso della riunione è stato analizzato lo sprint appena concluso e redatto il consuntivo del periodo. Successivamente, sono stati ruotati i ruoli per la prossima iterazione e stimato il relativo preventivo.

\subsubsection{Assegnazione ruoli}
Il gruppo ha deciso di assegnare i nuovi ruoli in modo che ciascun membro ricopra un ruolo che non ha ancora svolto dove possibile. Poiché questo sprint è il primo dopo la revisione RTB, si è deciso di introdurre il ruolo di Progettista.

\subsubsection{Preventivo e Consuntivo delle ore}
Durante la riunione il gruppo ha redatto il consuntivo delle ore relative allo svolgimento del sesto sprint e compilato il preventivo del settimo sprint. Entrambi consultabili nel documento Piano di Progetto nelle sezioni corrispondenti.\\
A seguito del completamento della sessione degli esami per tutti i membri del team, sono aumentate il numero di ore di orologio e di conseguenza produttive a disposizione per i prossimi sprint. Ne sono state preventivate quindi un numero maggiore, anche per rispettare quanto più possibile la nuova data di consegna del progetto prevista e scelta dal gruppo, ossia il 04/04/2025.

\subsubsection{Sprint Planning}
Dopo aver confermato il consuntivo orario della sesta iterazione, il gruppo si è confrontato sulle attività svolte e rimanenti. Avendo più tempo a disposizione, il gruppo ha deciso di rendere i prossimi sprint più intensi, riducendone la durata ad una sola settimana (dall'ottavo in poi) rispetto alle due attuali, per massimizzare la produttività e la comunicazione con la proponente, al fine di rispettare la nuova scadenza del progetto.

\subsubsection{Issue}
Durante l'incontro non è stato necessario creare nuove issue in quanto erano state già create e assegnate in seguito al primo incontro per la revisione RTB avuto con il prof. Cardin il 17/02/2025.

\subsection{Decisioni Prese}
Segue un elenco delle decisioni prese durante la riunione:
\begin{itemize}
    \item   Assegnazione dei ruoli per il sesto sprint:\\
            Responsabile: Giovanni Cristellon;\\
            Progettisti: Alfredo Rubino, Uncas Peruzzi, Leonardo Trolese;\\
            Programmatore: Federico Pivetta;\\
            Verificatori: Manuel Gusella, Riccardo Piva.
    \item   Riduzione della durata dei prossimi sprint.
\end{itemize}

\end{document}