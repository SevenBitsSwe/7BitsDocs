\documentclass[12pt]{article}

\usepackage[utf8]{inputenc}
\usepackage{geometry}
\usepackage{tabularx}
\usepackage{graphicx}
\usepackage[table,xcdraw]{xcolor}
\usepackage{tabularx, xcolor, etoolbox} % aggiungi etoolbox per condizioni

%path per conversione in locale
%\graphicspath{{../../../images/}}

%path per quando caricare in repo
\graphicspath{{images/}}

%cambio misure della pagina
\geometry{a4paper,left=20mm,right=20mm,top=20mm}

\title{Glossario}
\date{A.A 2024/2025}

\renewcommand*\contentsname{Indice}
\begin{document}
%contenuti principali
\maketitle
\begin{center}
\includegraphics[width=0.25\textwidth]{LogoUnipd}\\
\includegraphics[width=0.25\textwidth]{Sevenbitslogo}\\
sevenbits.swe.unipd@gmail.com\\
\vspace{2mm}


\textbf{Registro modifiche}\\
\vspace{2mm}
\begin{tabular}{|l|l|l|l|l|l|}
\hline
\textbf{Versione} & \textbf{Data} & \textbf{Autore} & \textbf{Verificatore} & \textbf{Descrizione}\\
\hline
0.2.0 & 2024-11-13  & Federico Pivetta & Riccardo Piva  & Modifica alla struttura ed\\
& & & & inserimento dei primi termini\\
\hline
0.1.0 & 2024-10-23  & Federico Pivetta & Riccardo Piva  & Creazione del documento\\
\hline

\end{tabular}
\end{center}
\newpage

\tableofcontents
\newpage

\section{Introduzione}
Questo documento ha lo scopo di fornire una raccolta di termini specifici e delle loro definizioni. Il glossario è stato pensato per facilitare la comprensione dei concetti chiave utilizzati nei vari documenti redatti. Per questa ragione ogni termine elencato viene accompagnato da una definizione chiara e concisa.\\

\section{Struttura del glossario}
Questo glossario è strutturato come una raccolta di semplici tabelle caratterizzate da due colonne: Termini e Definizioni. I termini sono elencati in ordine alfabetico nella prima colonna, ciascuno accompagnato da una definizione dettagliata nella colonna accanto. \newline
Per migliorare la navigazione, ad ogni lettera dell'alfabeto corrisponde una tabella. Questo sistema rende il glossario più ordinato e facilmente consultabile, aiutando il fruitore a trovare velocemente le definizioni di cui ha bisogno.\\

\newpage
\section{Tabelle Glossario}

\vspace{2mm}
\renewcommand{\arraystretch}{1.5} % gestisce altezza delle righe
\setlength{\tabcolsep}{10pt} % padding orizzontale alle celle

\subsection{A} % LETTERA A
\begin{tabularx}{\textwidth}{|>{\centering\arraybackslash}l|X|}
\hline
\rowcolor[gray]{0.8}
\textbf{Termine} & \textbf{Definizione}\\
\hline
Analisi dei Requisiti & Processo fondamentale dello sviluppo di un prodotto software, durante il quale vengono identificati e definiti in dettaglio i requisiti che il sistema deve soddisfare.\newline È molto importante questa analisi per assicurarsi che i requisiti individuatisiano completi e corretti, riducendo così possibili errori nelle fasi successive del progetto.\\
\hline
 & \\
\hline
\end{tabularx}

\subsection{B} % LETTERA B
\begin{tabularx}{\textwidth}{|>{\centering\arraybackslash}l|X|}
\hline
\rowcolor[gray]{0.8}
\textbf{Termine} & \textbf{Definizione}\\
\hline
 & \\
\hline
\end{tabularx}

\subsection{C} % LETTERA C
\begin{tabularx}{\textwidth}{|>{\centering\arraybackslash}l|X|}
\hline
\rowcolor[gray]{0.8}
\textbf{Termine} & \textbf{Definizione}\\
\hline
 & \\
\hline
\end{tabularx}

\subsection{D} % LETTERA D
\begin{tabularx}{\textwidth}{|>{\centering\arraybackslash}l|X|}
\hline
\rowcolor[gray]{0.8}
\textbf{Termine} & \textbf{Definizione}\\
\hline
Docker & Piattaforma che consente di creare, distribuire e eseguire applicazioni all'interno di container, ovvero ambienti isolati contenenti tutto il necessario per far funzionare una applicazione. I container garantiscono che l'applicazione funzioni in modo coerente su diversi sistemi, migliorando portabilità e gestione.\\
\hline
 & \\
\hline
\end{tabularx}

\subsection{E} % LETTERA E
\begin{tabularx}{\textwidth}{|>{\centering\arraybackslash}l|X|}
\hline
\rowcolor[gray]{0.8}
\textbf{Termine} & \textbf{Definizione}\\
\hline
 & \\
\hline
\end{tabularx}

\subsection{F} % LETTERA F
\begin{tabularx}{\textwidth}{|>{\centering\arraybackslash}l|X|}
\hline
\rowcolor[gray]{0.8}
\textbf{Termine} & \textbf{Definizione}\\
\hline
 & \\
\hline
\end{tabularx}

\subsection{G} % LETTERA G
\begin{tabularx}{\textwidth}{|>{\centering\arraybackslash}l|X|}
\hline
\rowcolor[gray]{0.8}
\textbf{Termine} & \textbf{Definizione}\\
\hline
 & \\
\hline
\end{tabularx}

\subsection{H} % LETTERA H
\begin{tabularx}{\textwidth}{|>{\centering\arraybackslash}l|X|}
\hline
\rowcolor[gray]{0.8}
\textbf{Termine} & \textbf{Definizione}\\
\hline
 & \\
\hline
\end{tabularx}

\subsection{I} % LETTERA I
\begin{tabularx}{\textwidth}{|>{\centering\arraybackslash}l|X|}
\hline
\rowcolor[gray]{0.8}
\textbf{Termine} & \textbf{Definizione}\\
\hline
 & \\
\hline
\end{tabularx}

\subsection{J} % LETTERA J
\begin{tabularx}{\textwidth}{|>{\centering\arraybackslash}l|X|}
\hline
\rowcolor[gray]{0.8}
\textbf{Termine} & \textbf{Definizione}\\
\hline
 & \\
\hline
\end{tabularx}

\subsection{K} % LETTERA K
\begin{tabularx}{\textwidth}{|>{\centering\arraybackslash}l|X|}
\hline
\rowcolor[gray]{0.8}
\textbf{Termine} & \textbf{Definizione}\\
\hline
 & \\
\hline
\end{tabularx}

\subsection{L} % LETTERA L
\begin{tabularx}{\textwidth}{|>{\centering\arraybackslash}l|X|}
\hline
\rowcolor[gray]{0.8}
\textbf{Termine} & \textbf{Definizione}\\
\hline
LaTeX & Linguaggio di marcatura basato su comandi che permettono di formattare il testo e produrre documenti. Viene spesso utilizzato in ambito accademico, scientifico e tecnico per la scrittura di articoli, libri, tesi e presentazioni.\\
\hline
 & \\
\hline
\end{tabularx}
 
\subsection{M} % LETTERA M
\begin{tabularx}{\textwidth}{|>{\centering\arraybackslash}l|X|}
\hline
\rowcolor[gray]{0.8}
\textbf{Termine} & \textbf{Definizione}\\
\hline
 & \\
\hline
\end{tabularx}

\subsection{N} % LETTERA N
\begin{tabularx}{\textwidth}{|>{\centering\arraybackslash}l|X|}
\hline
\rowcolor[gray]{0.8}
\textbf{Termine} & \textbf{Definizione}\\
\hline
 & \\
\hline
\end{tabularx}

\subsection{O} % LETTERA O
\begin{tabularx}{\textwidth}{|>{\centering\arraybackslash}l|X|}
\hline
\rowcolor[gray]{0.8}
\textbf{Termine} & \textbf{Definizione}\\
\hline
 & \\
\hline
\end{tabularx}

\subsection{P} % LETTERA P
\begin{tabularx}{\textwidth}{|>{\centering\arraybackslash}l|X|}
\hline
\rowcolor[gray]{0.8}
\textbf{Termine} & \textbf{Definizione}\\
\hline
 & \\
\hline
\end{tabularx}

\subsection{Q} % LETTERA Q
\begin{tabularx}{\textwidth}{|>{\centering\arraybackslash}l|X|}
\hline
\rowcolor[gray]{0.8}
\textbf{Termine} & \textbf{Definizione}\\
\hline
 & \\
\hline
\end{tabularx}

\subsection{R} % LETTERA R
\begin{tabularx}{\textwidth}{|>{\centering\arraybackslash}l|X|}
\hline
\rowcolor[gray]{0.8}
\textbf{Termine} & \textbf{Definizione}\\
\hline
Retrospettiva & Incontro che si tiene alla fine di ogni sprint, subito dopo la review. Durante questa riunione, il team di sviluppo riflette sullo sprint appena concluso, discutendo cosa è andato bene, cosa potrebbe essere migliorato e quali azioni concrete debbano essere adottate per migliorare i successivi sprint.\\
\hline
Review & Incontro che si tiene alla fine di uno sprint, in cui il team di sviluppo presenta il lavoro completato. Durante questa sessione, il team mostra le funzionalità sviluppate e raccoglie feedback per verificare se i requisiti sono stati soddisfatti e se ci sono modifiche da apportare.\\
\hline
RTB & Acronimo per Requirement and Technology Retrospective, prima revisione di avanzamento del progetto didattico. Fissa i requisiti da soddisfare in accordo con il proponente; motiva le tecnologie, i framework, le librerie adottate dimostrandone sia l'adeguatezza sia la compatibilità tramite il Proof of Concept (PoC).\\
\hline
& \\
\hline
\end{tabularx}

\subsection{S} % LETTERA S
\begin{tabularx}{\textwidth}{|>{\centering\arraybackslash}l|X|}
\hline
\rowcolor[gray]{0.8}
\textbf{Termine} & \textbf{Definizione}\\
\hline
SAL & Acronimo per "Stato Avanzamento Lavori", è un incontro in cui il team si riunisce per controllare il progresso degli obiettivi pianificati.\newline Durante questo incontro, si discute cosa è stato completato, cosa è ancora in corso ed cosa potrebbe ostacolare il progetto.\\
\hline
Sprint & Periodo di tempo dalle 2 alle 4 settimane, in cui un team di sviluppo si concentra sul completamento di un insieme ristretto di attività. Ogni sprint è seguito da una review per verificare il progresso raggiunto e da una retrospettiva per adottare ulteriori miglioramenti.\\
\hline
SyncLab & Azienda italiana attenta ai paradigmi della trasformazione digitale che realizza prodotti e soluzioni per diversi mercati quali: Sanità, Industria, Energia, Telco, Finanza e Trasporti \& Logistica. Offre anche consulenze per diversi temi come: GDPR, Big Data, Cloud Computing, IoT, Mobile e Cyber Security.\\
\hline
 & \\
\hline
\end{tabularx}

\subsection{T} % LETTERA T
\begin{tabularx}{\textwidth}{|>{\centering\arraybackslash}l|X|}
\hline
\rowcolor[gray]{0.8}
\textbf{Termine} & \textbf{Definizione}\\
\hline
 & \\
\hline
\end{tabularx}

\subsection{U} % LETTERA U
\begin{tabularx}{\textwidth}{|>{\centering\arraybackslash}l|X|}
\hline
\rowcolor[gray]{0.8}
\textbf{Termine} & \textbf{Definizione}\\
\hline
 & \\
\hline
\end{tabularx}

\subsection{V} % LETTERA V
\begin{tabularx}{\textwidth}{|>{\centering\arraybackslash}l|X|}
\hline
\rowcolor[gray]{0.8}
\textbf{Termine} & \textbf{Definizione}\\
\hline
 & \\
\hline
\end{tabularx}

\subsection{W} % LETTERA W
\begin{tabularx}{\textwidth}{|>{\centering\arraybackslash}l|X|}
\hline
\rowcolor[gray]{0.8}
\textbf{Termine} & \textbf{Definizione}\\
\hline
 & \\
\hline
\end{tabularx}

\subsection{X} % LETTERA X
\begin{tabularx}{\textwidth}{|>{\centering\arraybackslash}l|X|}
\hline
\rowcolor[gray]{0.8}
\textbf{Termine} & \textbf{Definizione}\\
\hline
 & \\
\hline
\end{tabularx}

\subsection{Y} % LETTERA Y
\begin{tabularx}{\textwidth}{|>{\centering\arraybackslash}l|X|}
\hline
\rowcolor[gray]{0.8}
\textbf{Termine} & \textbf{Definizione}\\
\hline
 & \\
\hline
\end{tabularx}

\subsection{Z} % LETTERA Z
\begin{tabularx}{\textwidth}{|>{\centering\arraybackslash}l|X|}
\hline
\rowcolor[gray]{0.8}
\textbf{Termine} & \textbf{Definizione}\\
\hline
 & \\
\hline
\end{tabularx}


\end{document}