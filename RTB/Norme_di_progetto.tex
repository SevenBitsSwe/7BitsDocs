\documentclass[10pt]{article}

\usepackage[utf8]{inputenc}
\usepackage{tabularx}
\usepackage{hyperref}
\usepackage{array}  
\usepackage{graphicx} % Per inserire immagini (loghi)
\usepackage{geometry} % Per personalizzare i margini
\usepackage{fancyhdr} % Per gestire intestazioni e piè di pagina
\usepackage{tikz} % Per disegni grafici
%\usepackage{xcolor} % Per definire e usare colori
\usepackage{anyfontsize}
\usepackage[table,xcdraw]{xcolor}
\usepackage{tabularx, etoolbox} % Aggiungi etoolbox per condizioni % Load xcolor
\usepackage{eso-pic} % Per aggiungere elementi grafici su tutte le pagine
\usepackage{titlesec} % Per avere ulteriore profondità nell'albero delle sottosezioni \subsubsubsection=\paragraph
\usepackage{float} % Per posizionare le immagini in un elenco puntato

\graphicspath{{../images/}}


% Imposta il livello di profondità dell'indice
\setcounter{tocdepth}{4} % Aggiunge paragraph all'indice

% Abilita la numerazione fino a paragraph
\setcounter{secnumdepth}{5} % Estende la profondità della numerazione

% Ridefinizione dello stile di paragraph per includere la numerazione
\renewcommand\theparagraph{\thesubsubsection.\arabic{paragraph}}
\titleformat{\paragraph}
  {\normalfont\normalsize\bfseries}{\theparagraph}{1em}{}
\titlespacing*{\paragraph}
  {0pt}{1ex plus .2ex}{1ex plus .2ex}

% Ridefinizione dello stile di subparagraph per includere la numerazione
\renewcommand\thesubparagraph{\theparagraph.\arabic{subparagraph}}
\titleformat{\subparagraph}
  {\normalfont\normalsize\itshape}{\thesubparagraph}{1em}{}
\titlespacing*{\subparagraph}
  {0pt}{1ex plus .2ex}{1ex plus .2ex}


% Cambio misure della pagina
\geometry{a4paper,left=20mm,right=20mm,top=20mm}
% Colore:ebdfc7
\definecolor{colorePie}{HTML}{ebdfc7}


% Header per ogni pagina
\pagestyle{fancy}
\fancyhf{}
\renewcommand{\headrulewidth}{0.4pt}
\lhead{
    \parbox[c]{1cm}{\includegraphics[width=1.1cm]{Sevenbitslogo.png}}
}
\rhead{\textcolor[HTML]{9e978a}{ NORME DI PROGETTO v0.3.1}
}
\setlength{\headheight}{25pt}
\cfoot{\thepage}


\renewcommand*\contentsname{Indice}


\begin{document}

\include{assets/title_page.tex}


%\clearpage
\setcounter{page}{1}


% Registro delle modifiche

\begin{center}
\textbf{Registro modifiche}\\
\vspace{2mm}

\begin{tabularx}{\textwidth}{|l|l|l|l|X|}
\hline
\textbf{Versione} & \textbf{Data} & \textbf{Autore} & \textbf{Verificatore} & \textbf{Descrizione}\\
    \hline
    0.3.6 & 2024-12-04  & Federico Pivetta & Leonardo Trolese & Ulteriore aggiornamento sottosezione "Sviluppo"\\
    \hline
    0.3.5 & 2024-12-02  & Alfredo Rubino & Giovanni Cristellon & Aggiornata sottosezione "Sviluppo"\\
    \hline
    0.3.4 & 2024-11-30  & Leonardo Trolese & Giovanni Cristellon & Correzioni grammaticali e di contenuti non più aderenti al Way of Working stabilito\\
    \hline
    0.3.3 & 2024-11-30  & Alfredo Rubino & Giovanni Cristellon & Aggiornata sottosezione "Sviluppo" e aggiunti livelli di profondità nell'indice\\
    \hline
    0.3.2 & 2024-11-28  & Alfredo Rubino & Giovanni Cristellon & Modifica ai termini del Glossario\\
    \hline
    0.3.1 & 2024-11-25  & Federico Pivetta & Riccardo Piva & Completata sottosezione "Fornitura" e aggiunta sottosezione "Sviluppo"\\
    \hline
    0.3.0 & 2024-11-24  & Federico Pivetta & Riccardo Piva & Aggiunta sezione "Processi primari"\\
    \hline
     0.2.1 & 2024-11-21  & Federico Pivetta  & Riccardo Piva& Completata sottosezione "Introduzione" e sottosezione "Documentazione", inclusa modifica alla tabella Registro modifiche\\
    \hline
    0.2.0 & 2024-11-20  & Leonardo Trolese & Federico Pivetta  & Aggiunta sezione "Supporto allo sviluppo software" e impostazione della divisione in sottodocumenti\\
    \hline
    0.1.0 & 2024-11-11  & Leonardo Trolese & Federico Pivetta  & Creazione del documento secondo la struttura definita dal gruppo\\
    \hline
\end{tabularx}
\end{center}


\include{assets/indice_contenuti.tex}


\section{Introduzione}
    \subsection{Scopo del documento}
    Il presente documento è stato realizzato ai fini di definire e raccogliere le best practices e il way of working a cui ogni componente del gruppo Seven Bits dovrà aderire per l'intera realizzazione del progetto, al fine di garantire l'adozione di un metodo di lavoro completamente omogeno.\\
    La formulaziome delle norme di progetto avviene in maniera progressiva, permettendo al gruppo di apportare continui aggiornamenti ad esse in risposta alle esigenze che il team dovrà affrontare durante lo svolgimento del progetto stesso.\\ 

    \subsection{Scopo del prodotto}
    Ogni giorno, le persone vengono sommerse da una miriade di annunci generici che spesso non rispecchiano i loro reali interessi o il contesto in cui si trovano. Questa separazione tra il messaggio e il destinatario porta ad una bassa interazione con gli utenti e una riduzione delle conversioni per i brand.\\
    Il progetto “Near You” si concentra sulla creazione di una dashboard composta principalmente da una mappa, sulla quale verranno visualizzate in tempo reale le posizioni degli utenti. Mediante un popup o una finestra a parte, verranno visualizzati messaggi personalizzati solo in prossimità dei punti di interesse.\\
    L'obiettivo finale è generare annunci pubblicitari in base agli interessi del cliente e alla sua posizione in quel momento.\\
    
    \subsection{Glossario}
    Ai fini di garantire l'adesione dei membri del gruppo ad un vocabolario comune e condiviso, che non lasci spazio ad ambiguità, dubbi o imprecisioni; il gruppo ha definito un documento denominato Glossario, nel quale sono presenti tutti i termini tecnici adottati dal gruppo per l'intera durata della realizzazione del progetto. Tali termini saranno contrassegnati con una $_G$ a pedice. Nel caso di termini composti, essi saranno uniti da un trattino (es. Analisi-dei-requisiti$_G$)

    \subsection{Riferimenti}
        \subsubsection{Riferimenti normativi}
        \begin{itemize}
            \item Capitolato di progetto C4 - Near You: \href{https://www.math.unipd.it/~tullio/IS-1/2024/Progetto/C4.pdf}{https://www.math.unipd.it/~tullio/IS-1/2024/Progetto/C4.pdf}
            \item Standard ISO/IEC 12207:1995: \href{https://www.math.unipd.it/~tullio/IS-1/2009/Approfondimenti/ISO_12207-1995.pdf}{https://www.math.unipd.it/~tullio/IS-1/2009/Approfondimenti/ISO_12207-1995.pdf}
        \end{itemize}
        \subsubsection{Riferimenti tecnologici}
        \begin{itemize}
            \item Documentazione Git: \href{https://git-scm.com/docs}{https://git-scm.com/docs}
            \item Documentazione GitHub: \href{https://docs.github.com/en}{https://docs.github.com/en}
            \item Documentazione \LaTeX: \href{https://www.latex-project.org/help/documentation/}{https://www.latex-project.org/help/documentation/}
            \item Documentazione Python: \href{https://www.python.org/doc/}{https://www.python.org/doc/}
        \end{itemize}
        
        


\section{Processi Primari}
\subsection{Fornitura}

\subsubsection{Descrizione e Scopo}
Come stabilito dallo standard ISO/IEC 12207:1995, il processo di fornitura definisce un insieme di linee guida necessario per una buona comunicazione tra fornitore e proponente. Il processo di fornitura si occupa di controllare e coordinare tutte le attività svolte dal gruppo, dalla comprensione dei requisiti fino alla consegna, per garantire che il prodotto finale soddisfi le esigenze concordate con la propontente.\\

\subsubsection{Attività}
Il processo di fornitura, come stabilito dallo standard ISO/IEC 12207:1995, si compone delle seguenti attività :
\begin{enumerate}
    \item \textbf{Avvio}: Composto dall'identificazione e comprensione delle richieste della proponente, con successiva verifica della fattibilità tecnologica di quest'ultime;
    \item \textbf{Preparazione dell'offerta}: Composta dall'elaborazione della proposta in grado di soddisfare le richieste della proponente, che dettagli i requisiti, i tempi, i costi e le condizioni contrattuali;
    \item \textbf{Contrattazione}: Composta dalla collaborazione tra fornitore e proponente per finalizzare i punti cardine della proposta;
    \item \textbf{Pianificazione}: Composta dalla pianificazione delle attività necessarie per soddisfare i requisiti della proponente, seguita da una suddivisione delle ore produttive disponibili ed una stima dei costi;
    \item \textbf{Esecuzione}: Composta dalla pianificazione e dallo sviluppo del prodotto in conformità ai requisiti concordati, insieme ad un monitoraggio continuo delle attività;
    \item \textbf{Revisione}: Composta dalla verifica periodica del progresso rispetto ai criteri definiti dal contratto;
    \item \textbf{Consegna}: Composta dalla consegna del prodotto software alla proponente, accompagnato dalla documentazione finale.
\end{enumerate}

\subsubsection{Rapporti con l'azienda proponente}
L'azienda proponente SyncLab$_G$ si è resa disponibile mediante diversi canali tra cui: e-mail, Discord e Google Meet ad una comunicazione frequente con il gruppo SevenBits, così da risolvere tempestivamente eventuali domande o dubbi che possono emergere durante lo svolgimento del progetto.\\
Durante la prima riunione organizzativa con l'azienda, è stata definita l'organizzazione dei periodi di sprint$_G$, stabilendo una durata di due settimane per ciascun ciclo. Al termine di ogni sprint$_G$ avviene un incontro SAL$_G$ (Stato di Avanzamento Lavori), dove verranno analizzati i risultati del lavoro svolto e si procederà con una sprint review$_G$. Inoltre, tra un SAL$_G$ e l'altro, è stato concordato un incontro intermedio per monitorare i progressi raggiunti e rispondere ad eventuali quesiti emersi.\\
Ogni incontro con l'azienda viene formalizzato attraverso un verbale esterno. Tale verbale è successivamente sottoposto alla proponente per la validazione mediante firma, in modo da ottenere un'approvazione formale del resoconto delle discussioni svolte durante la riunione.\\

\subsubsection{Documentazione fornita}
Di seguito sono elencati i documenti che il gruppo si impegna a consegnare ai Committenti, Prof. Tullio Vardanega e Prof. Riccardo Cardin, nonché all'azienda proponente:\\

    \paragraph{Analisi dei Requisiti}
    \'E un documento essenziale per lo sviluppo del prodotto software, che include la descrizione degli attori coinvolti, dei casi d’uso e l’elenco dei requisiti, suddivisi in requisiti funzionali, di qualità, di vincolo e prestazionali.\\

    \paragraph{Piano di Progetto}
    Il Piano di Progetto è un documento che ha lo scopo di definire in modo chiaro le modalità con cui ogni membro del gruppo svolgerà le attività per la realizzazione del progetto. Include l'analisi dei rischi, la pianificazione delle attività, la suddivisione dei ruoli e la stima di costi e risorse.\\

    \paragraph{Piano di Qualifica}
    Il Piano di Qualifica è un documento che ha l'obiettivo di garantire la qualità del prodotto e dei processi durante l'intero ciclo di vita del progetto, per questo motivo sarà aggiornato nel tempo per riflettere eventuali modifiche e i risultati delle verifiche effettuate. Include le sezioni sulla qualità di processo, sulla qualità di prodotto, le modalità di testing e il cruscotto di valutazione delle qualità.\\

    \paragraph{Glossario}
    Il Glossario è un documento che raccoglie dei termini specifici e le loro definizioni chiare e concise. Il suo scopo è quello di facilitare la comprensione dei concetti chiave presenti nei vari documenti redatti.\\

    \paragraph{Lettera di Presentazione}
    La Lettera di Presentazione è un documento che accompagna la consegna del prodotto software e della relativa documentazione durante le fasi di revisione di progetto. Il contenuto di questo documento comprende un link alla pagina web che contiene tutta la documentazione fin'ora prodotta ed un preventivo aggiornato rispetto a quello presentato alla revisione precedente.\\

\subsubsection{Strumenti}
 Gli strumenti utilizzati per la gestione del processo di fornitura sono i seguenti :
 \begin{itemize}
    \item \textbf{Google Meet e Discord}: servizi che permettono di effettuare videochiamate, utilizzati dal team per le discussioni sincrone e asincrone con la proponente;
    \item \textbf{Google Sheets}: servizio che permette la creazione di fogli di calcolo, utilizzato dal gruppo per la rendicontazione delle ore produttive impiegate durante ogni sprint$_G$;
    \item \textbf{Canva}: piattaforma che permette la creazione di presentazioni multimediali, utilizzata per la realizzazione dei diari di bordo;
    \item \textbf{Draw.io}: software utilizzato per creare diagrammi e grafici di vario tipo, in particolare è stato impiegato per realizzare diagrammi UML, come quelli dei casi d’uso;
 \end{itemize}

\subsection{Sviluppo}

\subsubsection{Descrizione e Scopo}
Il processo di sviluppo definisce tutte le attività necessarie allo sviluppo del prodotto software, al fine di garantire i requisiti e le scadenze concordate con la proponente. Comprende l'analisi dei requisiti, la progettazione e la codifica con relativa verifica.\\

\subsubsection{Analisi dei Requisiti}

\subsubsection{Progettazione}

\subsubsection{Codifica e Verifica}



\include{content/Supporto allo sviluppo software/supporto_sviluppo.tex}


\end{document}
