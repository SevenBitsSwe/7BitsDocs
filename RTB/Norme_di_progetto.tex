\documentclass[10pt]{article}

\usepackage[utf8]{inputenc}
\usepackage{tabularx}
\usepackage{hyperref}
\usepackage{array}  
\usepackage{graphicx} % Per inserire immagini (loghi)
\usepackage{geometry} % Per personalizzare i margini
\usepackage{fancyhdr} % Per gestire intestazioni e piè di pagina
\usepackage{tikz}
\usepackage{anyfontsize}
\usepackage[table,xcdraw]{xcolor}
\usepackage{tabularx, etoolbox} % aggiungi etoolbox per condizioni% Load xcolor
\usepackage{eso-pic} % Per aggiungere elementi grafici su tutte le pagine

\graphicspath{{../images/}}





%cambio misure della pagina
\geometry{a4paper,left=20mm,right=20mm,top=20mm}
%ebdfc7
\definecolor{colorePie}{HTML}{ebdfc7}


\pagestyle{fancy}
\fancyhf{}
\renewcommand{\headrulewidth}{0.4pt}
\lhead{
    \parbox[c]{1cm}{\includegraphics[width=1.1cm]{Sevenbitslogo.png}}
}
\rhead{\textcolor[HTML]{9e978a}{ GLOSSARIO v0.2.1}
}
\setlength{\headheight}{25pt}
\cfoot{\thepage}




\renewcommand*\contentsname{Indice}

\begin{document}

% Pagina del titolo
\begin{titlepage}
    \setcounter{page}{0}
    \centering
    % Inserisci il logo del gruppo (modifica il percorso dell'immagine)
    \includegraphics[width=7.2cm]{Sevenbitslogo.png} \\[2cm] 
    
    % Titolo
     {\fontsize{40}{40}\bfseries Norme di Progetto}\selectfont \\[3.9em]
    
    % Sottotitolo
    % Email del gruppo
    {\large sevenbits.swe.unipd@gmail.com} \\[3em]
    
    % Spazio per il logo dell'università
    \hfill
    
        
    \AddToShipoutPictureBG{ % Imposta il triangolo con logo
        \ifnum\value{page}=0
        \begin{tikzpicture}[overlay]
        
            % Definisce un triangolo blu in basso a destra
            \fill[colorePie] 
                (current page.south east) -- ++(-9cm,0) -- ++(9cm,9cm);
            
            % Inserisce il logo all'interno del triangolo
            \node[anchor=south east, xshift=-0.3cm, yshift=0.3cm] at (current page.south east) {
                \includegraphics[width=4.5cm]{LogoUnipd.png}
            };
        \end{tikzpicture}
        \fi
    }
        
    

    \vfill % Aggiunge spazio verticale per centrare il contenuto
\end{titlepage}
\newpage
\clearpage
\setcounter{page}{1}


\centering\textbf{Registro modifiche}\\
\vspace{2mm}
\begin{tabular}{|l|l|l|l|l|l|}
\hline
\textbf{Versione} & \textbf{Data} & \textbf{Autore} & \textbf{Verificatore} & \textbf{Descrizione}\\
\hline
0.1.0 & 2024-10-23  & Leonardo Trolese & Federico Pivetta  & Creazione del documento secondo \\ & & & & la struttura definita dal gruppo\\
\hline

\end{tabular}
\newpage
\raggedright
\tableofcontents
\newpage

\section{Introduzione}
    \subsection{Funzione del documento}
    Il presente documeto è stato realizzato ai fini di definire e reccagliore le best 
    practices e il way of working a cui ogni componente del gruppo Seven Bits dovrà
    aderire per l'intera realizzazione del progetto, al fine di garantire l'adozione 
    di un metodo di lavoro completamente omogeno\\
    La formulaziome delle norme di porgetto avviene in maniera progressiva, permettendo
    al gruppo di apportare continui aggiornamenti ad esse in risposta alle esigenze che
    il team deve affrontare durante lo svolgimento del progetto stesso.\\ 

    \subsection{Glossario}
    Ai fini di garantire l'adesione dei membri del gruppo a un vocabolario comune 
    e condiviso, che non lasci spazio a ambiguità, dubbi o imprecisioni; il team ha definito
    un documento denominato Glossario, nel quale sono presenti tutti i termini tecnici adottati
    dal gruppo per l'intera durata della realizzazione del progetto.

    \subsection{Riferimenti}
        \subsubsection{Riferimenti progettuali}
        \begin{itemize}
            \item \href{https://www.math.unipd.it/~tullio/IS-1/2024/Progetto/C4.pdf}{Descrizione capitolato di porgetto}
            \item Presentazione capitolato di progetto: 
        \end{itemize}
        \subsubsection{Riferimenti tecnologici}
        \begin{itemize}
            \item \href{https://git-scm.com/docs}{Documentazione git}
            \item \href{https://docs.github.com/en}{Documentazione GitHub}
            \item \href{https://www.latex-project.org/help/documentation/}{Documentazione LATEX}
            \item \href{https://www.python.org/doc/}{Documentazione Python:}
        \end{itemize}


\end{document}