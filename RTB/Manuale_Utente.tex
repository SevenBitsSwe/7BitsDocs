\documentclass[10pt]{article}

\usepackage[utf8]{inputenc}
\usepackage{tabularx}
\usepackage{hyperref}
\usepackage{array}  
\usepackage{graphicx}
\usepackage{geometry} 
\usepackage{fancyhdr} 
\usepackage{tikz}
\usepackage{ragged2e}
\usepackage{anyfontsize}
\usepackage[table,xcdraw]{xcolor}
\usepackage{tabularx, etoolbox} 
\usepackage{eso-pic}
\usepackage{float}
\usepackage{longtable}

\newcommand\version{0.1.0} % Versione

\graphicspath{{images/}}
%\graphicspath{{../images/}}

\setcounter{secnumdepth}{4}

%cambio misure della pagina
\geometry{a4paper,left=25mm,right=25mm,top=25mm,bottom=25mm}
%ebdfc7
\definecolor{colorePie}{HTML}{ebdfc7}

% Imposta il livello di profondità dell'indice
\setcounter{tocdepth}{4} % Aggiunge paragraph all'indice

% Abilita la numerazione fino a paragraph
\setcounter{secnumdepth}{5} % Estende la profondità della numerazione

\pagestyle{fancy}
\fancyhf{}
\renewcommand{\headrulewidth}{0.4pt}
\lhead{
    \parbox[c]{1cm}{\includegraphics[width=1.1cm]{Sevenbitslogo.png}}
}
\rhead{\textcolor[HTML]{9e978a}{ MANUALE UTENTE v\version}
}
\setlength{\headheight}{25pt}
\cfoot{\thepage}

\renewcommand*\contentsname{Indice}
\renewcommand{\listfigurename}{Elenco delle figure}
\renewcommand{\listtablename}{Elenco delle tabelle}

\begin{document}

% Pagina del titolo
\begin{titlepage}
    \setcounter{page}{0}
    \centering
    % Inserisci il logo del gruppo (modifica il percorso dell'immagine)
    \includegraphics[width=7.2cm]{Sevenbitslogo.png} \\[2cm] 
    
    % Titolo
     {\fontsize{40}{40}\bfseries Manuale Utente}\selectfont \\[3.9em]
    
 % Email del gruppo
    {\large sevenbits.swe.unipd@gmail.com} \\[3em]
    
    % Spazio per il logo dell'università
    \hfill
      
    \AddToShipoutPictureBG{ % Imposta il triangolo con logo
        \ifnum\value{page}=0
        \begin{tikzpicture}[overlay]
        
            % Definisce un triangolo blu in basso a destra
            \fill[colorePie] 
                (current page.south east) -- ++(-9cm,0) -- ++(9cm,9cm);
            
            % Inserisce il logo all'interno del triangolo
            \node[anchor=south east, xshift=-0.3cm, yshift=0.3cm] at (current page.south east) {
                \includegraphics[width=4.5cm]{LogoUnipd.png}
            };
        \end{tikzpicture}
        \fi
    }
        
\vfill % Aggiunge spazio verticale per centrare il contenuto
\end{titlepage}
\newpage
\clearpage
\setcounter{page}{1}

% Registro Modifiche
\begin{center}
 \textbf{Registro modifiche}\\   
\end{center}

\renewcommand{\arraystretch}{1.5}
\rowcolors{0}{gray!11}{white} % Aggiunge colore alternato alle righe

\begin{longtable}{|>{\centering\arraybackslash}m{1.5cm}|>{\centering\arraybackslash}m{2cm}|>{\centering\arraybackslash}m{2.5cm}|>{\centering\arraybackslash}m{2.5cm}|>{\centering\arraybackslash}m{5cm}|}
\hline
\textbf{Versione} & \textbf{Data} & \textbf{Autore} & \textbf{Verificatore} & \textbf{Descrizione}\\
\endhead
\hline
0.1.0 & 2025-03-03  & Alfredo Rubino & Riccardo Piva & Stesura sezione \ref{introduzione}\\
\hline
\end{longtable}
\rowcolors{0}{}{} % Riporta le righe alla colorazione originale

\newpage
\tableofcontents
\newpage
\listoffigures
\newpage
\listoftables

\newpage
\begin{justify}

\section{Introduzione}
\label{introduzione}

\subsection{Scopo del documento}
Il presente documento ha l'obiettivo di fornire una guida dettagliata sull'utilizzo del prodotto software \textit{"NearYou - Smart custom advertising platform"}, concernente il Capitolato C4 proposto dall'azienda Synclab e aggiudicato al gruppo dal Committente. Vengono illustrate le funzionalità principali, le modalità di interazione con il sistema e le opzioni di personalizzazione disponibili. Il manuale intende informare l'utente sui requisiti minimi necessari per l'utilizzo della piattaforma, sulle procedure di configurazione e su come sfruttare al meglio tutte le potenzialità offerte dal sistema.

\subsection{Scopo del prodotto}
Ogni giorno, le persone vengono sommerse da una miriade di annunci generici che spesso non rispecchiano i loro reali interessi o il contesto in cui si trovano. Questa separazione tra il messaggio e il destinatario porta ad una bassa interazione con gli utenti e una riduzione delle conversioni per i brand.\\
Il progetto \textit{"NearYou"} è stato sviluppato per affrontare questo problema, concentrandosi sulla creazione di una dashboard composta principalmente da una mappa, sulla quale vengono visualizzate in tempo reale le posizioni degli utenti. Mediante un pop-up o una finestra a parte, vengono visualizzati messaggi personalizzati solo in prossimità dei punti di interesse.\\
L'obiettivo finale è generare annunci pubblicitari in base agli interessi del cliente e alla sua posizione in quel momento, sfruttando la potenza dell'intelligenza artificiale con modelli linguistici (LLM) per creare messaggi pubblicitari dinamici e personalizzati che si adattano perfettamente alle esigenze degli utenti, migliorando significativamente l'efficacia degli annunci e il ritorno sull'investimento (ROI) per gli inserzionisti.

\subsection{Accesso alla piattaforma}
La piattaforma \textit{"NearYou"} è presentata come una web-application accessibile agli utenti amministratori.\\
L'accesso alla dashboard è garantito tramite un browser web, senza la necessità di installare software aggiuntivo sul dispositivo. L'accesso al servizio è protetto da credenziali fornite dal team amministrativo. Una volta effettuato l'accesso, l'amministratore può visualizzare una mappa con le posizioni in tempo reale degli utenti, monitorare l'efficacia degli annunci pubblicitari generati e configurare i parametri del sistema.\\
L'accesso alla piattaforma è garantito da qualsiasi dispositivo con connessione Internet, offrendo un'esperienza di utilizzo flessibile e accessibile.

\subsection{Glossario}
Al fine di evitare ambiguità relative alla terminologia utilizzata all'interno del documento, è presente il \textit{Glossario.pdf}, in cui vengono riportate tutte le definizione delle parole con un significato specifico. Questi termini veranno marcati con una $_G$ a pedice, mentre i termini composti, oltre alla $_G$ a pedice, saranno uniti da un "-" come segue: termine-composto$_G$.\\
Le definizioni sono presenti nell'apposito documento \textit{Glossario\_v1.0.0.pdf}.

\subsection{Riferimenti}

\subsubsection{Riferimenti normativi}
\begin{itemize}
    \item[-] Capitolatodi progetto C4 - NearYou - Smart custom advertising platform \\ 
    \textcolor{blue}{\texttt{\url{https://www.math.unipd.it/~tullio/IS-1/2024/Progetto/C4p.pdf}}} \\ 
    (Consultato: 2025-03-03).
    
    \item[-] Standard ISO/IEC 12207:1995 \\ 
    \textcolor{blue}{\texttt{\url{https://www.math.unipd.it/~tullio/IS-1/2009/Approfondimenti/ISO_12207-1995.pdf}}} \\
    (Consultato: 2025-03-03).
    
    \item[-] \textit{Norme\_di\_Progetto\_v1.0.0.pdf}
\end{itemize}

\subsubsection{Riferimenti tecnologici}
\begin{itemize}
    \item[-] Documentazione Git: \textcolor{blue}{\texttt{\url{https://git-scm.com/docs}}} \\
    (Consultato: 2025-03-03).
    
    \item[-] Documentazione GitHub$_G$: \textcolor{blue}{\texttt{\url{https://docs.github.com/en}}} \\
    (Consultato: 2025-03-03).
    
    \item[-] Documentazione \LaTeX: \textcolor{blue}{\texttt{\url{https://www.latex-project.org/help/documentation/}}} \\
    (Consultato: 2025-03-03).
    
    \item[-] Documentazione Python$_G$: \textcolor{blue}{\texttt{\url{https://www.python.org/doc/}}} \\
    (Consultato: 2025-03-03).
    
    \item[-] Documentazione LangChain: \textcolor{blue}{\texttt{\url{https://www.langchain.com}}} \\
    (Consultato: 2025-03-03).
\end{itemize}

\subsubsection{Riferimenti informativi}
\begin{itemize}
    \item[-] Analisi dei Requisiti$_G$ v1.0.0
    \item[-] Specifica Tecnica$_G$ v1.0.0
\end{itemize}


\section{Requisiti}
\label{requisiti}

\subsection{Requisiti hardware}

\subsection{Requisiti di sistema operativo}

\subsection{Requisiti software}

\subsection{Requisiti browser}

%% Codice per Immagini %%

%\begin{figure}[ht]
%    \centering
%    \includegraphics[width=0.5\linewidth]{NomeImmagine.png}
%    \caption{TestoCaption}
%    \label{LabelImmagine}
%\end{figure}


%% Codice per Tabelle %%

%\begin{center}
%\renewcommand{\arraystretch}{1.5}
%\begin{longtable}
%{|>{\centering\arraybackslash}m{2.7cm}|>       {\centering\arraybackslash}m{2.7cm}|>{\centering\arraybackslash}m{2.1cm}|}
%\hline
%\textbf{} & \textbf{} & \textbf{}\\
%\endhead
%\hline
%\hline
%\caption{Requisiti$_G$ funzionali}
%\end{longtable}
%\end{center}

\end{justify}
\end{document}