\documentclass[10pt]{article}

\usepackage[utf8]{inputenc}
\usepackage{tabularx}
\usepackage{hyperref}
\usepackage{array}  
\usepackage{graphicx}
\usepackage{geometry}
\usepackage{fancyhdr} 
\usepackage{tikz}
\usepackage{anyfontsize}
\usepackage[table,xcdraw]{xcolor}
\usepackage{tabularx, etoolbox}
\usepackage{eso-pic}
\usepackage{float}

\newcommand\version{0.1.0} % Versione

\graphicspath{{images/}}
%\graphicspath{{../images/}}

\geometry{a4paper,left=20mm,right=20mm,top=20mm} % cambio misure pagina
%ebdfc7
\definecolor{colorePie}{HTML}{ebdfc7}

\pagestyle{fancy}
\fancyhf{}
\renewcommand{\headrulewidth}{0.4pt}
\lhead{
    \parbox[c]{1cm}{\includegraphics[width=1.1cm]{Sevenbitslogo.png}}
}
\rhead{\textcolor[HTML]{9e978a}{ Nome\_Documento v\version}
}
\setlength{\headheight}{25pt}
\cfoot{\thepage}

\renewcommand*\contentsname{Indice}

\begin{document}

\begin{titlepage} % Pagina del titolo
    \setcounter{page}{0}
    \centering
    % Logo del gruppo (modifica il percorso dell'immagine)
    \includegraphics[width=7.2cm]{Sevenbitslogo.png} \\[2cm] 
    
    % Titolo
     {\fontsize{40}{40}\bfseries Nome\_Documento}\selectfont \\[3.9em]
    
    % Sottotitolo : Email del gruppo
    {\large sevenbits.swe.unipd@gmail.com} \\[3em]
    
    % Spazio per il logo dell'università
    \hfill
    
        
    \AddToShipoutPictureBG{ % Triangolo con logo
        \ifnum\value{page}=0
        \begin{tikzpicture}[overlay]
        
            % Definisce un triangolo in basso a destra
            \fill[colorePie] 
                (current page.south east) -- ++(-9cm,0) -- ++(9cm,9cm);
            
            % Inserisce il logo all'interno del triangolo
            \node[anchor=south east, xshift=-0.3cm, yshift=0.3cm] at (current page.south east) {
                \includegraphics[width=4.5cm]{LogoUnipd.png}
            };
        \end{tikzpicture}
        \fi
    }

\vfill
\end{titlepage}
\newpage
\clearpage

\setcounter{page}{1}

\begin{center} % Registro modifiche
\textbf{Registro modifiche}\\
\vspace{2mm}
\begin{tabularx}{\textwidth}{|l|l|l|l|X|}
\hline
\textbf{Versione} & \textbf{Data} & \textbf{Autore} & \textbf{Verificatore} & \textbf{Descrizione}\\
\hline
\end{tabularx}   
\end{center}

\newpage
\tableofcontents % Indice

%% Codice per Immagini %%

%\begin{figure}[ht]
%    \centering
%    \includegraphics[width=0.5\linewidth]{NomeImmagine.png}
%    \caption{TestoCaption}
%    \label{LabelImmagine}
%\end{figure}


%% Codice per Tabelle %%

%\begin{table}[H]
%\centering
%\renewcommand{\arraystretch}{1.5}
%\begin{tabular}{|>{\centering\arraybackslash}m{2.7cm}|>{\centering\arraybackslash}m{2.7cm}|>{\centering\arraybackslash}m{6cm}|>{\centering\arraybackslash}m{2.1cm}|}
%\hline
%\textbf{Id. Requisito} & \textbf{Importanza} & \textbf{Descrizione} & \textbf{Fonti}\\
%\hline
%\end{tabular}
%\caption{TestoCaption}
%\end{table}


\end{document}
