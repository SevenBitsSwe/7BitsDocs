\documentclass[10pt]{article}

\usepackage[utf8]{inputenc}
\usepackage{tabularx}
\usepackage{hyperref}
\usepackage{array}  
\usepackage{graphicx}
\usepackage{geometry} 
\usepackage{fancyhdr} 
\usepackage{tikz}
\usepackage{ragged2e}
\usepackage{anyfontsize}
\usepackage[table,xcdraw]{xcolor}
\usepackage{tabularx, etoolbox} 
\usepackage{eso-pic}
\usepackage{float}
\usepackage{longtable}

\newcommand\version{0.1.0} % Versione

\graphicspath{{images/}}
%\graphicspath{{../images/}}

\setcounter{secnumdepth}{4}

%cambio misure della pagina
\geometry{a4paper,left=25mm,right=25mm,top=25mm,bottom=25mm}
%ebdfc7
\definecolor{colorePie}{HTML}{ebdfc7}
\pagestyle{fancy}
\fancyhf{}
\renewcommand{\headrulewidth}{0.4pt}
\lhead{
    \parbox[c]{1cm}{\includegraphics[width=1.1cm]{Sevenbitslogo.png}}
}
\rhead{\textcolor[HTML]{9e978a}{ Nome Documento v\version}
}
\setlength{\headheight}{25pt}
\cfoot{\thepage}

\renewcommand*\contentsname{Indice}
\renewcommand{\listfigurename}{Elenco delle figure}
\renewcommand{\listtablename}{Elenco delle tabelle}

\begin{document}

% Pagina del titolo
\begin{titlepage}
    \setcounter{page}{0}
    \centering
    % Inserisci il logo del gruppo (modifica il percorso dell'immagine)
    \includegraphics[width=7.2cm]{Sevenbitslogo.png} \\[2cm] 
    
    % Titolo
     {\fontsize{40}{40}\bfseries Nome\_Documento}\selectfont \\[3.9em]
    
 % Email del gruppo
    {\large sevenbits.swe.unipd@gmail.com} \\[3em]
    
    % Spazio per il logo dell'università
    \hfill
      
    \AddToShipoutPictureBG{ % Imposta il triangolo con logo
        \ifnum\value{page}=0
        \begin{tikzpicture}[overlay]
        
            % Definisce un triangolo blu in basso a destra
            \fill[colorePie] 
                (current page.south east) -- ++(-9cm,0) -- ++(9cm,9cm);
            
            % Inserisce il logo all'interno del triangolo
            \node[anchor=south east, xshift=-0.3cm, yshift=0.3cm] at (current page.south east) {
                \includegraphics[width=4.5cm]{LogoUnipd.png}
            };
        \end{tikzpicture}
        \fi
    }
        
\vfill % Aggiunge spazio verticale per centrare il contenuto
\end{titlepage}
\newpage
\clearpage
\setcounter{page}{1}

% Registro Modifiche
\begin{center}
 \textbf{Registro modifiche}\\   
\end{center}

\renewcommand{\arraystretch}{1.5}
\rowcolors{0}{gray!11}{white} % Aggiunge colore alternato alle righe

\begin{longtable}{|>{\centering\arraybackslash}m{1.5cm}|>{\centering\arraybackslash}m{2cm}|>{\centering\arraybackslash}m{2.5cm}|>{\centering\arraybackslash}m{2.5cm}|>{\centering\arraybackslash}m{5cm}|}
\hline
\textbf{Versione} & \textbf{Data} & \textbf{Autore} & \textbf{Verificatore} & \textbf{Descrizione}\\
\endhead
\hline
\end{longtable}
\rowcolors{0}{}{} % Riporta le righe alla colorazione originale

\newpage
\tableofcontents
\newpage
\listoffigures
\newpage
\listoftables

\newpage
\begin{justify}

\section{Introduzione}

%% Codice per Immagini %%

%\begin{figure}[ht]
%    \centering
%    \includegraphics[width=0.5\linewidth]{NomeImmagine.png}
%    \caption{TestoCaption}
%    \label{LabelImmagine}
%\end{figure}


%% Codice per Tabelle %%

%\begin{center}
%\renewcommand{\arraystretch}{1.5}
%\begin{longtable}
%{|>{\centering\arraybackslash}m{2.7cm}|>       {\centering\arraybackslash}m{2.7cm}|>{\centering\arraybackslash}m{2.1cm}|}
%\hline
%\textbf{} & \textbf{} & \textbf{}\\
%\endhead
%\hline
%\hline
%\caption{Requisiti$_G$ funzionali}
%\end{longtable}
%\end{center}

\end{justify}
\end{document}
