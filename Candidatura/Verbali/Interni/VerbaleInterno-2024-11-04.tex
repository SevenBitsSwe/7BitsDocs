\documentclass[12pt]{article}

\usepackage[utf8]{inputenc}
\usepackage{geometry}
\usepackage{tabularx}
\usepackage{graphicx}

\graphicspath{{images/}}

%cambio misure della pagina
\geometry{a4paper,left=20mm,right=20mm,top=20mm}

\title{Verbale Interno del meeting in data 04/11/2024}
\date{A.A 2024/2025}

\renewcommand*\contentsname{Indice}
\begin{document}
%contenuti principali
\maketitle
\begin{center}
\includegraphics[width=0.25\textwidth]{LogoUnipd}\\
\includegraphics[width=0.25\textwidth]{Sevenbitslogo}\\
sevenbits.swe.unipd@gmail.com\\
\vspace{2mm}

\textbf{Registro modifiche}\\
\vspace{2mm}
\begin{tabular}{|l|l|l|l|l|l|}
\hline
\textbf{Versione} & \textbf{Data} & \textbf{Descrizione} & \textbf{Ruolo} & \textbf{Componente} \\
\hline
1.0 & 05/11/2024 & Verifica del documento & Verificatore & Pivetta Federico\\
\hline
0.3 & 04/11/2024 & Controllo generale del documento & Relatore & Gusella Manuel\\
& & Modifica delle Conclusioni & &\\
\hline
0.2 & 04/11/2024 & Redazione Verbale & Relatore & Pivetta Federico\\
& & Modifica di Sintesi e Conclusioni & & \\
\hline
0.1 & 04/11/2024 & Stesura del verbale & Scrittore & Cristellon Giovanni\\
\hline
\end{tabular}
\end{center}
\newpage
\tableofcontents
\newpage
\section{04/11/2024}
\subsection{Durata e partecipanti}
\begin{itemize}
\item Ora: 15:30 - 18:00;
\item Partecipanti: 	
	\begin{itemize}
	\item Gusella Manuel;
	\item Cristellon Giovanni;
	\item Peruzzi Uncas;
	\item Piva Riccardo;
	\item Pivetta Federico;
	\item Rubino Alfredo;
	\item Trolese Leonardo.
	\end{itemize}
\item Piattaforma: Discord (online)
\end{itemize}
\subsection{Sintesi}
A seguito della lezione in data 04/11/2024, durante la quale sono state presentate le decisioni di aggiudicazione degli appalti, il gruppo si è riunito in modo sincrono per discutere in merito al feedback ricevuto. Sono state analizzate tutte le mancanze segnalate ed intraprese le azioni necessarie per poter inviare una nuova candidatura quanto più curata possibile.
\subsection{Conclusioni}
Sono state prese dal gruppo diverse decisioni tra cui:
\begin{itemize}
\item l'uso di GitHub Pages e GitHub Actions per quanto riguarda la presentazione dei documenti, così da evitare la compilazione e l'inserimento manuale di ogni documento (vedasi Issue \#5 e \#9);
\item l'adozione di una procedura di ticketing utilizzando il sistema delle GitHub Issues per svolgere tutte le task in modo efficiente con l'ausilio di una project board;
\item l'aggiornamento della dichiarazione di impegni tenendo più in considerazione il ruolo dell'Analista, con la conseguente modifica della lettera di presentazione, e aggiungendo indicazioni sui criteri di rotazione dei ruoli che in precedenza mancavano (vedasi Issue \#6 e \#7);
\item l'uso corretto del numero di versione per ogni artefatto;
\item la correzione dei futuri verbali che dovranno tenere più in considerazione le decisioni prese e le ragioni che hanno spinto il gruppo verso quest'ultime (vedasi Issue \#3 e \#4); 
\item l'aggiornamento del registro delle modifiche, segnalando più chiaramente le modifiche apportate;
\item l'uso delle pull requests eseguendo così verifiche più corrette, per salvaguardare l'integrità del repository GitHub;
\end{itemize}

\end{document}
