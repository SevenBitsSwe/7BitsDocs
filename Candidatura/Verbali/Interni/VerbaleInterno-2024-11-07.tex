\documentclass[10pt]{article}

\usepackage[utf8]{inputenc}
\usepackage{geometry}
\usepackage{tabularx}
\usepackage{graphicx}
\usepackage{hyperref}
\usepackage{array}  

\graphicspath{{images/}}

%cambio misure della pagina
\geometry{a4paper,left=20mm,right=20mm,top=20mm}

\title{Verbale Interno del meeting in data}
\date{A.A 2024/2025}

\renewcommand*\contentsname{Indice}
\begin{document}
%contenuti principali
\maketitle
\center 
\includegraphics[width=0.25\textwidth]{LogoUnipd}\\
\includegraphics[width=0.25\textwidth]{Sevenbitslogo}\\
sevenbits.swe.unipd@gmail.com\\
\vspace{2mm}

\textbf{Registro modifiche}\\
\vspace{2mm}
\begin{tabular}{|c|c|c|c|c|c|}
\hline
\textbf{Versione} & \textbf{Data} & \textbf{Autore} & \textbf{Verificatore} & \textbf{Descrizione} \\
\hline
0.1.0 & 2024-11-08 & Federico Pivetta & Riccardo Piva & Approvazione del documento \\
\hline
0.0.1 & 2024-11-08 & Uncas Peruzzi & Federico Pivetta & Redazione del documento \\
\hline
\end{tabular}

\raggedright
\newpage
\tableofcontents
\newpage
\section{2024-11-07}
\subsection{Durata e partecipanti}
\begin{itemize}
\item Ora: 15:30 - 16:30 ;
\item Partecipanti: 	
	\begin{itemize}
	\item Gusella Manuel;
	\item Cristellon Giovanni;
	\item Peruzzi Uncas;
	\item Piva Riccardo;
	\item Pivetta Federico;
	\item Trolese Leonardo.
	\end{itemize}
\item Piattaforma: Discord (online)
\end{itemize}
\subsection{Sintesi}
\subsubsection{Motivo Riunione}
A seguito della ricezione dell'avvenuta aggiudicazione d'appalto per il \textbf{Capitolato C4} il gruppo ha deciso di riunirsi in maniera sincrona per riflettere sui nuovi feedback ricevuti dal professore nel file \textit{Avanzamento Revisioni.xlsx} e per riflettere sulle decisioni da prendere nel periodo che segue.
\subsubsection{Strumento di Redazione Verbali}
Si è discusso su un eventuale passaggio da LaTex a Typst per la redazione e verifica dei verbali, con la veloce conclusione che disponendo ora di una Github Action funzionante con LaTex non è poi così vantaggioso questo cambio di tecnologia.
\subsubsection{Rotazione dei Ruoli}
Il corpo centrale della riunione ha riguardato la rotazione e assegnazione dei vari ruoli. Si è deciso che essendo ancora in una fase pre-iniziale del progetto , di assegnare dei ruoli fino al giorno \textbf{2024-11-12} cioè il giorno della presentazione del diario di bordo in attesa di eventuali indicazioni e di un colloquio con la proponente esterna. A questo proposito si è deciso di inviare una mail per contattare l'azienda \textit{SyncLab} per avere un colloquio e discutere sulle fasi iniziali. É stato sottolineato inoltre,che durante lo svolgimento effettivo del progetto i ruoli resteranno i medesimi per almeno 14 giorni, cioè per tutta la durata di uno \textit{sprint}.
\subsubsection{Analisi dei Feedback ricevuti}
Nell'ultima frazione della riunione sono stati analizzati i feedback ricevuti dal docente. In particolare ci si è soffermati su \textit{"Non avete ancora ben compreso che
solo la verifica andata a buon fine
causa (abilita) l'attuazione di una
modifica di baseline e quindi di un
cambio di versione"}. \\
Dopo un analisi e un confronto di alcuni componenti con il docente in aula per avere delle esplicazioni si è deciso di modificare leggermente il \href{https://github.com/SevenBitsSwe/7BitsDocs/blob/main/ModelloVerbale.tex}{Modello del Verbale}, e il processo di stesura.
\begin{itemize}
    \item Aggiunta colonna della colonna verificatore ad ogni "modifica" svolta sul documento
    \item Il cambio di versione avviene solamente una volta che è stata effettuala una verifica andata a buon fine.
    \item Il redattore del documento, che lavora sul branch approval, una volta scritto il documento, dovrà aprire una Pull Request su github e il verificatore tramite commenti e tag dovrà informare i componenti di eventuali modifiche o verifica a buon fine
    \item Il documento ora dovrà essere infine approvato da un componente, una volta effettuata l'approvazione sarà chiamato il verificatore di questo procedimento che chiuderà la procedura con la \textbf{chiusura della pull request} e il merge del branch approval in main. 
\end{itemize}
\subsection{Decisioni Prese}
Queste le decisioni finali dei componenti:
\begin{itemize}
    \item Decisione dei ruoli fino a 2024-11-12: \\ Responsabile: \textbf{Manuel Gusella} \\ Verificatori: \textbf{Federico Pivetta, Riccardo Piva}
    \item \textbf{Uncas Peruzzi} si occuperà della stesura del Verbale Interno
    \item \textbf{Manuel Gusella} si occuperà della redazione del Diario di Bordo da presentare il giorno 2024-11-12.
    \item \textbf{Leonardo Trolese} invierà una mail all'azienda per informare dell'aggiudicazione d'appalto e modifica il file README.md aggiungendo il link alla Github Pages contenente la documentazione.


\end{itemize}

\vspace{2mm}



\begin{center}
    


\begin{tabular}{|>{\hspace{20pt}}c<{\hspace{20pt}}|>{\hspace{20pt}}c<{\hspace{20pt}}|}
\hline
\textbf{Rif.Issue} & \textbf{Dettaglio Decisione}\\
\hline
\href{https://github.com/SevenBitsSwe/7BitsDocs/issues/15}{Issue \#15} & Stesura VerbaleInterno-2024-11-07 \\
\hline
\href{https://github.com/SevenBitsSwe/7BitsDocs/issues/16}{Issue \#16} & Creazione del secondo Diario di Bordo\\
\hline
\href{https://github.com/SevenBitsSwe/7BitsDocs/issues/17}{Issue \#17} & Modificare il template dei verbali.\\
\hline
\href{https://github.com/SevenBitsSwe/7BitsDocs/issues/18}{Issue \#18} & Verifica e Approvazione VerbaleInterno-2024-11-07 \\
\hline 

\end{tabular}
\end{center}

\end{document}