\documentclass[10pt]{article}

\usepackage[table,xcdraw]{xcolor}
\usepackage[utf8]{inputenc}
\usepackage{tabularx}
\usepackage{hyperref}
\usepackage{array}
\usepackage{graphicx} % Per inserire immagini (loghi)
\usepackage{geometry} % Per personalizzare i margini
\usepackage{fancyhdr} % Per gestire intestazioni e piè di pagina
\usepackage{tikz}
\usepackage{ragged2e}
\usepackage{anyfontsize}
\usepackage{tabularx, etoolbox}
\usepackage{eso-pic} % Per aggiungere elementi grafici su tutte le pagine
\usepackage{float}
\usepackage{longtable}

\newcommand\version{0.0.1} %aggiunta versione come variabile

\graphicspath{{images/}}
%\graphicspath{{../images/}}

\setcounter{secnumdepth}{4}
\setcounter{tocdepth}{4}

%cambio misure della pagina
\geometry{a4paper,left=25mm,right=25mm,top=25mm,bottom=25mm}
\definecolor{colorePie}{HTML}{ebdfc7}
\pagestyle{fancy}
\fancyhf{}
\renewcommand{\headrulewidth}{0.4pt}
\lhead{
    \parbox[c]{1cm}{\includegraphics[width=1.1cm]{Sevenbitslogo.png}}
}
\rhead{\textcolor[HTML]{9e978a}{ SPECIFICA TECNICA v\version}
}
\setlength{\headheight}{25pt}
\cfoot{\thepage}

\renewcommand*\contentsname{Indice}
\renewcommand{\listfigurename}{Elenco delle figure}
\renewcommand{\listtablename}{Elenco delle tabelle}

\begin{document}

% Pagina del titolo
\begin{titlepage}
    \setcounter{page}{0}
    \centering
    % Inserisci il logo del gruppo (modifica il percorso dell'immagine)
    \includegraphics[width=7.2cm]{Sevenbitslogo.png} \\[2cm]

    % Titolo
     {\fontsize{40}{40}\bfseries Specifica Tecnica}\selectfont \\[3.9em]

    % Sottotitolo
    {\huge NearYou\\ \vspace{3mm }Smart custom advertising platform} \\[2.7em]

    % Email del gruppo
    {\large sevenbits.swe.unipd@gmail.com} \\[3em]

    % Spazio per il logo dell'università
    \hfill

    \AddToShipoutPictureBG{ % Imposta il triangolo con logo
        \ifnum\value{page}=0
        \begin{tikzpicture}[overlay]

            % Definisce un triangolo blu in basso a destra
            \fill[colorePie]
                (current page.south east) -- ++(-9cm,0) -- ++(9cm,9cm);

            % Inserisce il logo all'interno del triangolo
            \node[anchor=south east, xshift=-0.3cm, yshift=0.3cm] at (current page.south east) {
                \includegraphics[width=4.5cm]{LogoUnipd.png}
            };
        \end{tikzpicture}
        \fi
    }

\vfill % Aggiunge spazio verticale per centrare il contenuto
\end{titlepage}
\newpage
\clearpage
\setcounter{page}{1}

% Registro Modifiche
\begin{center}
 \textbf{Registro modifiche}\\   
\end{center}

\renewcommand{\arraystretch}{1.5}
\rowcolors{0}{gray!11}{white} % Aggiunge colore alternato alle righe

\begin{longtable}{|>{\centering\arraybackslash}m{1.5cm}|>{\centering\arraybackslash}m{2cm}|>{\centering\arraybackslash}m{2.5cm}|>{\centering\arraybackslash}m{2.5cm}|>{\centering\arraybackslash}m{5cm}|}
\hline
\textbf{Versione} & \textbf{Data} & \textbf{Autore} & \textbf{Verificatore} & \textbf{Descrizione}\\
\endhead
    \hline
    0.1.0 & 2025-02-26 & Leonardo Trolese  & Manuel Gusella & Inizio redazione del documento\\
    \hline
\end{longtable}
\rowcolors{0}{}{} % Riporta le righe alla colorazione originale

\newpage
\tableofcontents
\newpage
\listoffigures %elenco delle figure sarà da usare per ogni immagine
\newpage
\listoftables %lista delle tabelle presenti nel documento
\newpage
\begin{justify}

\section{Introduzione}
\subsection{Scopo del documento}
Il presente documento si propone come una risorsa completa per la comprensione degli aspetti tecnici e progettuali della piattaforma "NearYou", dedicata alla 
creazione di soluzioni di advertising personalizzato tramite intelligenza artificiale. L’obiettivo principale è fornire una descrizione dettagliata dell’architettura 
implementativa e di deployment, illustrando le tecnologie adottate e le motivazioni alla base delle scelte progettuali.\\
Nel contesto dell'architettura implementativa, il documento analizza nel dettaglio i moduli principali del sistema, i design pattern utilizzati. Saranno inclusi 
diagrammi delle classi, e una spiegazione dettagliata dei design pattern utilizzati e delle motivazioni di queste scelte.\\
Gli obiettivi di questo documento sono: motivare le decisioni architetturali, fungere da guida per lo sviluppo della piattaforma, e garantire la piena tracciabilità e 
copertura dei requisiti definiti nel documento di \textit{Analisi dei Requisiti\_v1.0.0}.\\
In sintesi, il documento intende essere un punto di riferimento essenziale per tutti gli attori coinvolti nel ciclo di vita del progetto, offrendo una visione chiara e 
strutturata delle fondamenta tecniche che sorreggono NearYou e delle logiche che ne determinano il funzionamento.\\

\subsection{Glossario}
Con l'intento di evitare ambiguità interpretative del linguaggio utilizzato, viene fornito un Glossario che si occupa di esplicitare il significato dei termini che riguardano il contesto del Progetto$_G$. I termini presenti nel glossario sono contrassegnati con una \textit{G} a pedice : Termine$_G$.\\
I termini composti, oltre alla $_G$ a pedice, saranno uniti da un "-" come segue: termine-composto$_G$.\\
Le definizioni sono presenti nell'apposito documento \textit{Glossario\_v1.0.0.pdf}.

\subsection{Riferimenti}
\subsubsection{Riferimenti normativi}
\begin{itemize}
    \item[-] Regolamento del Progetto$_G$ didattico  \\
    \textcolor{blue}{\texttt{\url{https://www.math.unipd.it/~tullio/IS-1/2024/Dispense/PD1.pdf}}}\\ (Consultato: 2025-02-10).
    \item[-] Capitolato$_G$ C4 - NearYou - Smart custom advertising platform\\
    \textcolor{blue}{\texttt{\url{https://www.math.unipd.it/~tullio/IS-1/2024/Progetto/C4p.pdf}}}\\ (Consultato: 2025-02-10).
    \item[-] \textit{Norme\_di\_Progetto\_v1.0.0.pdf}
\end{itemize}

\subsubsection{Riferimenti informativi}
\begin{itemize}
    \item[-] \textit{Glossario\_v1.0.0.pdf}
    \item[-] \textit{Analisi\_dei\_Requisiti\_v1.0.0.pdf}
    \item[-] Analisi-dei-Requisiti$_G$ - SWE 2024-25\\
    \textcolor{blue}{\texttt{\url{https://www.math.unipd.it/~tullio/IS-1/2024/Dispense/T05.pdf}}}\\ (Consultato: 2025-02-10).
    
    \item[-] Dependency Injection - SWE 2024-25\\    
    \textcolor{blue}{\texttt{\url{https://www.math.unipd.it/~rcardin/swea/2022/Design%20Pattern%20Architetturali%20-%20Dependency%20Injection.pdf}}}\\ (Consultato: 2025-02-26).
    
    \item[-] Design Pattern Creazionali - SWE 2024-25\\
    \textcolor{blue}{\texttt{\url{https://www.math.unipd.it/~rcardin/swea/2022/Design%20Pattern%20Creazionali.pdf}}}\\ (Consultato: 2025-02-26).

    \item[-] Design Pattern Strutturali - SWE 2024-25\\
    \textcolor{blue}{\texttt{\url{https://www.math.unipd.it/~rcardin/swea/2022/Design%20Pattern%20Strutturali.pdf}}}\\ (Consultato: 2025-02-26).

    \item[-] Software Architecture Patterns - SWE 2024-25\\
    \textcolor{blue}{\texttt{\url{https://www.math.unipd.it/~rcardin/swea/2022/Software%20Architecture%20Patterns.pdf}}}\\ (Consultato: 2025-02-26).

    \item[-] Verbali Interni
    \item[-] Verbali Esterni
\end{itemize}

\section{Tecnologie}


\section{API}


\section{Architettura di sistema}


\section{Architettura di deployment}


\section{Stato dei requisiti funzionali}


\end{justify}
\end{document}
