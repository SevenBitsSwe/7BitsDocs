\documentclass[10pt]{article}

\usepackage[utf8]{inputenc}
\usepackage{geometry}
\usepackage{tabularx}
\usepackage{graphicx}
\usepackage{hyperref}
\usepackage{array}  

%path per conversione in locale
%\graphicspath{{../../../images/}}

%path per quando caricare in repo
\graphicspath{{images/}}

%cambio misure della pagina
\geometry{a4paper,left=20mm,right=20mm,top=20mm}

\title{Verbale Esterno del meeting in data 2025-03-10} % Sostituire la data
\date{A.A 2024/2025}

\renewcommand*\contentsname{Indice}
\begin{document}
\maketitle
\center 
\includegraphics[width=0.25\textwidth]{LogoUnipd}\\
\includegraphics[width=0.25\textwidth]{Sevenbitslogo}\\
sevenbits.swe.unipd@gmail.com\\
\vspace{2mm}

\newpage
\raggedright
\tableofcontents

\newpage
\section{2025-03-10} % Inserire la data in formato yyyy-mm-dd
\subsection{Durata e partecipanti}
\begin{itemize}
\item Ora: 10:30 - 10:50; % Sostituire con l'orario corretto del meeting
\item Partecipanti: 	
	\begin{itemize}
        \item SevenBits:
        \begin{itemize}
            \item Giovanni Cristellon;
            \item Gusella Manuel;
            \item Peruzzi Uncas;
            \item Piva Riccardo;
            \item Pivetta Federico;
            \item Rubino Alfredo;
            \item Trolese Leonardo.
        \end{itemize}
            \item SyncLab:
            \begin{itemize}
                \item Dorigo Andrea;
                \item Pallaro Fabio.
	    \end{itemize}
	\end{itemize}
\item Piattaforma: Google meet (online)
\end{itemize}

\subsection{Ordine del giorno}
\begin{itemize}
    \item Aggiornamento azienda su stato progettazione schema delle classi;
    \item Aggiornamento azienda su stato avanzamento prodotto MVP.
\end{itemize}

\subsection{Oggetto}
Incontro organizzativo con Andrea Dorigo e Fabio Pallaro dell'azienda SyncLab, incentrato sull'allineare gruppo e azienda su quanto svolto dallo scorso SAL a ora.

\subsection{Sintesi}
All'inizio dell'incontro il gruppo ha presentato lo stato attuale della progettazione dello schema delle classi e ha informato l'azienda del futuro incontro con il professor Cardin per verificare la correttezza dell'architettura proposta.\\
Poi l'incontro è proseguito facendo vedere l'avanzamento del prodotto MVP e sono state fatte delle domande dal gruppo riguardo la dashboard e delle scelte stilistiche per l'implementazione.\\

\subsubsection{Obiettivi per la realizzazione dell'MVP}
La proponente ha inoltre richiesto al gruppo di definire in maniera precisa e tracciabile gli obiettivi da concretizzare con la realizzazione dell'MVP, e che differenzieranno
quest'ultimo dal PoC sviluppato dal team. Gli obiettivi individuati sono i seguenti:
\begin{itemize}
    \item Modifica del sistema di visualizzazione dei messaggi e adeguamento dello stesso con le richieste della proponente.
\end{itemize}

\subsection{Decisioni Prese}
È stato deciso di cambiare l'attuale visualizzazione del messaggio, facendo in modo che il messaggio segua l'utente e che poi scompaia nel momento in cui egli esce dal raggio di generazione del messaggio.\\
Il gruppo si è posto l'obiettivo di modificare la mappa e presentare al prossimo SAL il nuovo sistema di visualizzazione dei messaggi.\\

\section{Obiettivi prossimo SAL}
Gli obiettivi stabiliti per il prossimo SAL sono:
    \begin{itemize}
            \item Presentazione del nuovo sistema di visualizzazione dei messaggi.
    \end{itemize}
    \begin{center}
    \begin{tabular}{|>{\centering\arraybackslash}m{3cm}|>{\centering\arraybackslash}m{12cm}|}
	\hline
	\textbf{Rif.Issue} & \textbf{Dettaglio Decisione}\\
        \hline
            \href{https://github.com/SevenBitsSwe/MVP/issues/4}{Issue \#4} & Creazione tabella sensore DB \\
        \hline
            \href{https://github.com/SevenBitsSwe/MVP/issues/5}{Issue \#5} & Implementazione nuovo sistema UUID simulatore posizioni \\
        \hline
            \href{https://github.com/SevenBitsSwe/MVP/issues/6}{Issue \#6} & Implementazione logica di business\\
        \hline
            \href{https://github.com/SevenBitsSwe/MVP/issues/9}{Issue \#9} & Modifica visualizzazione messaggi\\
        \hline
    \end{tabular}
    \end{center}

%parte di firma
\vfill
\begin{minipage}{10cm}
Firma: \hrulefill \\
\vspace{2mm}
Data: \dotfill
\end{minipage}

\end{document}
