\documentclass[10pt]{article}

\usepackage[utf8]{inputenc}
\usepackage{geometry}
\usepackage{tabularx}
\usepackage{graphicx}
\usepackage{hyperref}
\usepackage{array}  

%path per conversione in locale
%\graphicspath{{../../../images/}}

%path per quando caricare in repo
\graphicspath{{images/}}

%cambio misure della pagina
\geometry{a4paper,left=20mm,right=20mm,top=20mm}

\title{Verbale Esterno del meeting in data 2025-03-17} % Sostituire la data
\date{A.A 2024/2025}

\renewcommand*\contentsname{Indice}
\begin{document}
\maketitle
\center 
\includegraphics[width=0.25\textwidth]{LogoUnipd}\\
\includegraphics[width=0.25\textwidth]{Sevenbitslogo}\\
sevenbits.swe.unipd@gmail.com\\
\vspace{2mm}

\newpage
\raggedright
\tableofcontents

\newpage
\section{2025-03-17} % Inserire la data in formato yyyy-mm-dd
\subsection{Durata e partecipanti}
\begin{itemize}
\item Ora: 10:30 - 10:50; % Sostituire con l'orario corretto del meeting
\item Partecipanti: 	
	\begin{itemize}
            \item SevenBits:
            \begin{itemize}
                \item Giovanni Cristellon;
                \item Gusella Manuel;
                \item Peruzzi Uncas;
                \item Piva Riccardo;
                \item Pivetta Federico;
                \item Rubino Alfredo;
                \item Trolese Leonardo.
	    \end{itemize}
            \item SyncLab:
            \begin{itemize}
                \item Dorigo Andrea;
                \item Zorzi Daniele;
                \item Pallaro Fabio.
	    \end{itemize}
	\end{itemize}
\item Piattaforma: Google meet (online)
\end{itemize}

\subsection{Ordine del giorno}
\begin{itemize}
    \item Condivisione delle modifiche alla progettazione dopo l'incontro con il Professor Cardin;
    \item Aggiornamento sullo stato di avanzamento dell’MVP;
    \item Definizione delle modalità di presentazione in azienda prima della revisione PB.
\end{itemize}

\subsection{Oggetto}
Incontro organizzativo con i proponenti Andrea Dorigo, Fabio Pallaro e Daniele Zorzi, incentrato sull'allineamento gruppo e azienda su quanto svolto dallo scorso SAL ad oggi.

\subsection{Sintesi}
Durante l'incontro sono stati presentati l’MVP e le modifiche apportate alla progettazione a seguito della riunione con il Professor Cardin. Inoltre, è stata discussa la problematica relativa alla scadenza dei token dell’API Groq. Infine, si è concordata la struttura della presentazione da esibire in azienda prima della revisione PB ed è stata fissata la data del prossimo SAL per il 24/03/2025 alle 10:30.

    \subsubsection{Progettazione}
    Riguardo alla progettazione, è stata presentata l'architettura ideata dal gruppo. Rispetto alla versione illustrata nel precedente SAL, per la parte di simulazione, è stato eliminato il pattern Observer, ritenuto non corretto per il nostro prodotto. Inoltre, i compiti di creazione dei sensori e di scelta della strategia di simulazione sono stati riassegnati dalle responsabilità dell'amministratore di sensori alle classi incaricate.
    
    \subsubsection{MVP}
    È stata mostrata l'ultima versione dell’MVP, con particolare attenzione alla mappa generale che presenta la nuova disposizione dei messaggi richiesta durante il precedente SAL.

    \subsubsection{Token API}
    È stata discussa l'ottimizzazione della generazione dei messaggi, poiché con 10 utenti si esaurirebbe troppo rapidamente il numero di token gratuiti concessi dall'API Groq. Si prevede quindi una riduzione del numero massimo di utenti a un valore inferiore, stimato intorno a 7/8, per rimanere entro i limiti dell'API gratuita.

    \subsubsection{Presentazione in azienda}
    Sono state discusse le modalità e la struttura della presentazione da effettuare in azienda prima della revisione PB. La presentazione dovrà bilanciare dettagli tecnici e funzionalità e si richiede la creazione di slide a supporto della discussione.

\subsection{Decisioni Prese}
È stato deciso di ridurre il numero massimo di utenti per rispettare i limiti imposti dall'API Groq gratuita. Inoltre, sono state concordate la struttura e le modalità della presentazione da effettuare in azienda.
La data del prossimo incontro SAL è stata fissata per il 24/03/2025 alle 10:30.

\subsection{Obiettivi prossimo SAL} 
Gli obiettivi stabiliti per il prossimo SAL sono:
    \begin{itemize}
            \item Mostrare la tabella delle attività presente nella dashboard generale;
            \item Continuare la stesura dei vari documenti;
            \item Continuare lo sviluppo dell'MVP e dei relativi test;
    \end{itemize}

%parte di firma
\vfill
\begin{minipage}{10cm}
Firma: \hrulefill \\
\vspace{2mm}
Data: \dotfill
\end{minipage}

\end{document}
