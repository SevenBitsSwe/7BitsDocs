\documentclass[10pt]{article}

\usepackage[utf8]{inputenc}
\usepackage{geometry}
\usepackage{tabularx}
\usepackage{graphicx}
\usepackage{hyperref}
\usepackage{array}  

%path per conversione in locale
%\graphicspath{{../../../images/}}

%path per quando caricare in repo
\graphicspath{{images/}}

%cambio misure della pagina
\geometry{a4paper,left=20mm,right=20mm,top=20mm}

\title{Verbale Esterno del meeting in data 2025-03-03} % Sostituire la data
\date{A.A 2024/2025}

\renewcommand*\contentsname{Indice}
\begin{document}
\maketitle
\center 
\includegraphics[width=0.25\textwidth]{LogoUnipd}\\
\includegraphics[width=0.25\textwidth]{Sevenbitslogo}\\
sevenbits.swe.unipd@gmail.com\\
\vspace{2mm}

\newpage
\raggedright
\tableofcontents

\newpage
\section{2025-03-03} % Inserire la data in formato yyyy-mm-dd
\subsection{Durata e partecipanti}
\begin{itemize}
\item Ora: 10:30 - 10:50; % Sostituire con l'orario corretto del meeting
\item Partecipanti: 	
	\begin{itemize}
        \item SevenBits:
        \begin{itemize}
            \item Gusella Manuel;
            \item Peruzzi Uncas;
            \item Piva Riccardo;
            \item Pivetta Federico;
            \item Rubino Alfredo;
            \item Trolese Leonardo.
        \end{itemize}
            \item SyncLab:
            \begin{itemize}
                \item Dorigo Andrea;
                \item Pallaro Fabio.
	    \end{itemize}
	\end{itemize}
\item Piattaforma: Google meet (online)
\end{itemize}

\subsection{Ordine del giorno}
\begin{itemize}
    \item Discussione esito RTB;
    \item Presentazione nuova data di consegna;
    \item Presentazione bozza dell'architettura della componente di simulazione dei sensori;
    \item Discussione sulla durata degli sprint da svolgere prima della consegna PB.
\end{itemize}

\subsection{Oggetto}
Incontro organizzativo con Andrea Dorigo e Fabio Pallaro dell'azienda SyncLab, finalizzato a pianificare lo sviluppo del progetto a seguito della consegna RTB, 
e allineare gruppo e azienda su quanto svolto fino ad ora.

\subsection{Sintesi}
Durante l'incontro la proponente è stata messa al corrente dello stato attuale dello sviluppo del progetto, e nello specifico dell'esito della consegna RTB ricevuto in data
2025-03-01. Il gruppo ha inoltre informato la proponente del cambiamento della data di consegna prevista per il progetto, pianificata ora per il 2025-04-04.\\
Ha seguito l'esposizione di quanto svolto fino ad ora in ambito di progettazione: nello specifico è stata presentata una bozza di architettura software per la componente
di simulazione dei sensori, che è stata approvata dall'azienda, con qualche indicazione sulle possibili librerie da usare per la gestione dei thread, oltre a quella 
standard.\\

\subsubsection{Obiettivi per la realizzazione dell'MVP}
La proponente ha inoltre richiesto al gruppo di definire in maniera precisa e tracciabile gli obiettivi da concretizzare con la realizzazione dell'MVP, e che differenzieranno
quest'ultimo dal PoC sviluppato dal team. Gli obiettivi individuati sono i seguenti:
\begin{itemize}
    \item Realizzazione di una dashboard specifica per utente che consenta la visualizzazione dei dati (messaggi generati e percorso seguito) accessibile dalla dashboard 
    principale;
    \item Realizzazione di una tabella che per ogni punto di interesse registrato mostri il numero di messaggi generati nel mese corrente e ordini i punti di interesse in base a questo ultimo dato;
    \item Aumento del numero di utenti simulati contemporaneamente fino a $\simeq10$ (attualmente 1).
\end{itemize}

\subsection{Decisioni Prese}
\'E stato deciso di ridurre la durata delle iterazioni da due settimane a una fino alla consegna PB. Questo cambiamento avverrà dal prossimo sprint, che inizierà in data 2025-03-07. 
Il gruppo si è posto l'obiettivo di completare e presentare la proposta di architettura per la componente di simulazione dei sensori per il prossimo SAL (fissato con 
l'azienda in data 2025-03-10).

\section{Obiettivi prossimo SAL} 
Gli obiettivi stabiliti per il prossimo SAL sono:
    \begin{itemize}
            \item Presentazione della proposta di architettura per la componente di simulazione dei sensori adattata al multithreading;
    \end{itemize}
    \begin{center}
    \begin{tabular}{|>{\centering\arraybackslash}m{3cm}|>{\centering\arraybackslash}m{12cm}|}
	\hline
	\textbf{Rif.Issue} & \textbf{Dettaglio Decisione}\\
	    \hline
            \href{https://github.com/SevenBitsSwe/7BitsDocs/issues/161}{Issue \#161} & Aggiornamento bozza architettura simulazione sensori con aggiunta di multithreading\\
        \hline
            \href{https://github.com/SevenBitsSwe/7BitsDocs/issues/160}{Issue \#160} & Redazione verbale esterno 2025-03-03\\
        \hline
    \end{tabular}
    \end{center}

%parte di firma
\vfill
\begin{minipage}{10cm}
Firma: \hrulefill \\
\vspace{2mm}
Data: \dotfill
\end{minipage}

\end{document}