\documentclass[10pt]{article}

\usepackage[utf8]{inputenc}
\usepackage{geometry}
\usepackage{tabularx}
\usepackage{graphicx}
\usepackage{hyperref}
\usepackage{array}  

%path per conversione in locale
%\graphicspath{{../../../images/}}

%path per quando caricare in repo
\graphicspath{{images/}}

%cambio misure della pagina
\geometry{a4paper,left=20mm,right=20mm,top=20mm}

\title{Verbale Esterno del meeting in data 2025-03-24} % Sostituire la data
\date{A.A 2024/2025}

\renewcommand*\contentsname{Indice}
\begin{document}
\maketitle
\center 
\includegraphics[width=0.25\textwidth]{LogoUnipd}\\
\includegraphics[width=0.25\textwidth]{Sevenbitslogo}\\
sevenbits.swe.unipd@gmail.com\\
\vspace{2mm}

\newpage
\raggedright
\tableofcontents

\newpage
\section{2025-03-24} % Inserire la data in formato yyyy-mm-dd
\subsection{Durata e partecipanti}
\begin{itemize}
\item Ora: 10:30 - 10:45; % Sostituire con l'orario corretto del meeting
\item Partecipanti: 	
	\begin{itemize}
            \item SevenBits:
            \begin{itemize}
                \item Giovanni Cristellon;
                \item Gusella Manuel;
                \item Peruzzi Uncas;
                \item Piva Riccardo;
                \item Pivetta Federico;
                \item Rubino Alfredo;
                \item Trolese Leonardo.
	    \end{itemize}
            \item SyncLab:
            \begin{itemize}
                \item Dorigo Andrea;
                \item Pallaro Fabio.
	    \end{itemize}
	\end{itemize}
\item Piattaforma: Google meet (online)
\end{itemize}

\subsection{Ordine del giorno}
\begin{itemize}
    \item Aggiornamento sullo stato di avanzamento dell’MVP;
    \item Definizione della data di presentazione del prodotto \textit{Near You} in azienda prima della revisione PB.
\end{itemize}

\subsection{Oggetto}
Incontro organizzativo con i proponenti Andrea Dorigo, Fabio Pallaro, incentrato sull'allineamento gruppo e azienda su quanto svolto dallo scorso SAL ad oggi.

\subsection{Sintesi}
Durante l'incontro sono state esplicate le modifiche apportate al progetto a seguito dell'ultimo sprint. Inoltre, il gruppo ha deciso di fissare una data definitiva per la presentazione del prodotto finale in azienda.

    
    \subsubsection{MVP}
    È stata aggiornata l'azienda sullo stato di sviluppo dell'MVP. In particolare, è stata aggiunta una tabella nell'interfaccia grafica che tiene traccia del numero di pubblicità generate per ciascuna azienda. Il gruppo ha fatto presente che, essendo l'MVP nella fase finale dello sviluppo, non ci sono altre modifiche significative degne di essere discusse, in quanto ci si sta concentrando sul completamento della redazione dei documenti necessari per la consegna finale.

    \subsubsection{Presentazione in azienda}
    Sono state discusse e confermate le modalità di presentazione del prodotto sviluppato dal gruppo. La presentazione non dovrà essere incentrata sugli aspetti tecnici e prevede, al termine, una dimostrazione pratica della demo del prodotto. Il gruppo ha proposto una data per la presentazione in presenza in azienda, concordando la scelta in base alle varie disponibilità della Proponente.

\subsection{Decisioni Prese}
È stato deciso di svolgere l'incontro il giorno 2025-04-02, alle 16:00 presso \textit{Synclab Srl}; il gruppo si impegnerà a effettuare una presentazione con delle slide, accompagnata dalla dimostrazione del prodotto. Questa riunione sostituirà il consueto SAL che solitamente si svolgeva il lunedì. In caso di dubbi durante questo sprint, le domande saranno poste tramite il canale di comunicazione\textit{Discord}.

\subsection{Obiettivi prossimo SAL} 
Gli obiettivi stabiliti per il prossimo SAL sono:
    \begin{itemize}
            \item Preparare una presentazione del prodotto Near You;
            \item Continuare la stesura dei vari documenti;
            \item Continuare lo sviluppo dell'MVP e dei relativi test di sistema e di integrazione.
    \end{itemize}

%parte di firma
\vfill
\begin{minipage}{10cm}
Firma: \hrulefill \\
\vspace{2mm}
Data: \dotfill
\end{minipage}

\end{document}