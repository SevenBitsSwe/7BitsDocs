\documentclass[10pt]{article}

\usepackage[utf8]{inputenc}
\usepackage{tabularx}
\usepackage{hyperref}
\usepackage{array}  
\usepackage{graphicx}
\usepackage{geometry} 
\usepackage{fancyhdr} 
\usepackage{tikz}
\usepackage{ragged2e}
\usepackage{anyfontsize}
\usepackage[table,xcdraw]{xcolor}
\usepackage{tabularx, etoolbox} 
\usepackage{eso-pic}
\usepackage{float}
\usepackage{longtable}

\graphicspath{{images}}
%\graphicspath{{../images/}}

\setcounter{secnumdepth}{4}


%cambio misure della pagina
\geometry{a4paper,left=25mm,right=25mm,top=25mm,bottom=25mm}
%ebdfc7
\definecolor{colorePie}{HTML}{ebdfc7}
\pagestyle{fancy}
\fancyhf{}
\renewcommand{\headrulewidth}{0.4pt}
\lhead{
    \parbox[c]{1cm}{\includegraphics[width=1.1cm]{Sevenbitslogo.png}}
}
\rhead{\textcolor[HTML]{9e978a}{LETTERA DI PRESENTAZIONE}
}
\setlength{\headheight}{25pt}
\cfoot{\thepage}


\renewcommand*\contentsname{Indice}
\renewcommand{\listfigurename}{Elenco delle figure}
\renewcommand{\listtablename}{Elenco delle tabelle}

\begin{document}

% Pagina del titolo
\begin{titlepage}
    \setcounter{page}{0}
    \centering
    % Inserisci il logo del gruppo (modifica il percorso dell'immagine)
    \includegraphics[width=7.2cm]{Sevenbitslogo.png} \\[2cm] 
    
    % Titolo
     {\fontsize{30}{30}\bfseries Lettera di Presentazione PB}\selectfont \\[3em]
    % Email del gruppo
    {\large sevenbits.swe.unipd@gmail.com} \\[3em]
    
    % Sottotitolo
    {\huge Componenti} \\[2.7em]
    {\large 
     \begin{tabular}{l l l}
       Cristellon & Giovanni & 1216730\\
       Gusella & Manuel & 2076430\\
       Peruzzi & Uncas & 2068239\\
       Piva & Riccardo & 2079241\\
       Pivetta & Federico & 2009693\\
       Rubino & Alfredo & 2076435\\
       Trolese & Leonardo & 2068238\\ 
     \end{tabular}
    } \\[3em]
    
    % Spazio per il logo dell'università
    \hfill
      
    \AddToShipoutPictureBG{ % Imposta il triangolo con logo
        \ifnum\value{page}=0
        \begin{tikzpicture}[overlay]
        
            % Definisce un triangolo blu in basso a destra
            \fill[colorePie] 
                (current page.south east) -- ++(-9cm,0) -- ++(9cm,9cm);
            
            % Inserisce il logo all'interno del triangolo
            \node[anchor=south east, xshift=-0.3cm, yshift=0.3cm] at (current page.south east) {
                \includegraphics[width=4.5cm]{LogoUnipd.png}
            };
        \end{tikzpicture}
        \fi
    }
        
\vfill % Aggiunge spazio verticale per centrare il contenuto
\end{titlepage}
\newpage
\noindent
Egregio prof. Vardanega,\\
Egregio prof. Cardin,\\
con la presente lettera il gruppo SevenBits è lieto di comunicarVi la propria candidatura alla revisione di avanzamento Product Baseline, per il capitolato denominato:
\begin{center}
\textcolor{blue}{\texttt{\href{https://www.math.unipd.it/~tullio/IS-1/2024/Progetto/C4.pdf}{C4 - NearYou Smart custom advertising platform}}}\\
\end{center}
proposto dall'azienda Sync Lab.\\
Tutta la documentazione riguardante il lavoro svolto è consultabile nel sito web del nostro gruppo:
\begin{center}
\textcolor{blue}{\texttt{\url{https://sevenbitsswe.github.io/7BitsDocs/documentation.html}}}\\
\end{center}
All'interno della sezione ``PB'', oltre alla presente lettera, sono inclusi i seguenti documenti:
\begin{itemize}
\item Documentazione Esterna:
  \begin{itemize}
    \item Analisi dei Requisiti v2.0.0;
    \item Piano di Progetto v2.0.0;
    \item Piano di Qualifica v2.0.0;
    \item Manuale Utente v.1.0.0;
    \item Specifica Tecnica v1.0.0;
    \item Verbali esterni.
  \end{itemize}
\item Documentazione Interna:
  \begin{itemize}
    \item Norme di Progetto v2.0.0;
    \item Glossario v2.0.0;
    \item Verbali interni.
  \end{itemize}
\end{itemize}

\noindent
Viene inoltre reso disponibile il Minimum Viable Product, approvato dall'azienda proponente in data 2025-04-02 e disponibile al seguente link:
\begin{center}
\textcolor{blue}{\texttt{\url{https://github.com/SevenBitsSwe/MVP}}}\\
\end{center}
Con la presente lettera desideriamo aggiornarVi in merito ai costi e alle tempistiche del progetto.\\
La spesa complessiva sostenuta ammonta a \textbf{12.510 €}, risultando inferiore rispetto al budget iniziale, pari a \textbf{12.950 €}, con un risparmio complessivo di \textbf{440 €}.\\
Tuttavia, si è verificato un lieve ritardo rispetto a quanto dichiarato alla revisione RTB, la data di ultima consegna era infatti prevista per il \textbf{4 aprile 2025} ma è stata spostata al \textbf{11 aprile 2025} a causa della necessità di ultimare le verifiche di una parte della documentazione. Nonostante il lieve slittamento della consegna finale, desideriamo sottolineare che la consegna del prodotto MVP all'azienda e della documentazione richiesta per la revisione con il professor Cardin sono state completate con successo entro la data prevista del \textbf{4 aprile 2025}.\\
Di seguito viene riportata la tabella riassuntiva delle ore di lavoro impiegate da ogni membro del team:
\begin{longtable}{|>{\centering\arraybackslash}m{3.5cm}|>{\centering\arraybackslash}m{4cm}|>{\centering\arraybackslash}m{4cm}|}
\hline
\rowcolor[gray]{0.8}
\textbf{Membro} & \textbf{Ore Impiegate} & \textbf{Ore Previste}\\
\endhead
\hline
Cristellon Giovanni & 87 & 95\\
\hline
Gusella Manuel & 89 & 95\\
\hline
Peruzzi Uncas & 91 & 92\\
\hline
Piva Riccardo & 93 & 95\\
\hline
Pivetta Federico & 93 & 95\\
\hline
Rubino Alfredo & 84 & 85\\
\hline
Trolese Leonardo & 90 & 92\\ 
\hline
\end{longtable}

\begin{longtable}{|c|c|}
\hline
\cellcolor[gray]{0.8} Totale Utilizzate & 627 ore \\
\hline
\cellcolor[gray]{0.8} Rimanenti & 22 ore \\
\hline
\end{longtable}

\vspace{2mm}
\noindent
Cordiali saluti,\\
Gruppo SevenBits
\end{document}
