\documentclass[12pt]{article}

\usepackage[utf8]{inputenc}
\usepackage{geometry}
\usepackage{tabularx}
\usepackage{graphicx}
\usepackage[table,xcdraw]{xcolor}

\graphicspath{{images/}}

%cambio misure della pagina
\geometry{a4paper,left=20mm,right=20mm,top=20mm}

\title{Glossario}
\date{A.A 2024/2025}

\renewcommand*\contentsname{Indice}
\begin{document}
%contenuti principali
\maketitle
\center 
\includegraphics[width=0.25\textwidth]{LogoUnipd}\\
\includegraphics[width=0.25\textwidth]{Sevenbitslogo}\\
sevenbits.swe.unipd@gmail.com\\
\vspace{2mm}

\raggedright
\tableofcontents
\newpage

\section{Introduzione}
Questo documento ha lo scopo di fornire una raccolta di termini specifici e delle loro definizioni. Il glossario è stato pensato per facilitare la comprensione dei concetti chiave utilizzati nei vari documenti redatti. Per questa ragione ogni termine elencato viene accompagnato da una definizione chiara e concisa.\\

\section{Tabella Glossario}

% Tabella Glossario
\renewcommand{\arraystretch}{1.5} % gestisce altezza delle righe
\setlength{\tabcolsep}{10pt} % padding orizzontale alle celle

\begin{tabularx}{\textwidth}{|>{\centering\arraybackslash}l|X|}
\hline
\rowcolor[gray]{0.9}
\textbf{Termine} & \textbf{Definizione}\\
\hline
\setlength{\arrayrulewidth}{0.5mm} % ritorno spessore normale
Esempio di Termine & Esempio di Definizione.\\

\hline
\end{tabularx}

\end{document}