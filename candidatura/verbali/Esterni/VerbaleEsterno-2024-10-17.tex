\documentclass[12pt]{article}

\usepackage[utf8]{inputenc}
\usepackage{geometry}
\usepackage{tabularx}
\usepackage{graphicx}

\graphicspath{{images/}}

%cambio misure della pagina
\geometry{a4paper,left=20mm,right=20mm,top=20mm}

\title{Verbale Esterno del meeting in data 17/10/2024}
\date{A.A 2024/2025}

\renewcommand*\contentsname{Indice}
\begin{document}
%contenuti principali
\maketitle
\center 
\includegraphics[width=0.25\textwidth]{LogoUnipd}\\
\includegraphics[width=0.25\textwidth]{Sevenbitslogo}\\
sevenbits.swe.unipd@gmail.com\\
\vspace{2mm}

\textbf{Registro modifiche}\\
\vspace{2mm}
\begin{tabular}{|l|l|l|l|l|l|}
\hline
\textbf{Versione} & \textbf{Data} & \textbf{Descrizione} & \textbf{Ruolo} & \textbf{Componente} \\
\hline
1.0 & 17/10 & Stesura del verbale & Scrittori & Gusella Manuel\\
\hline
1.0 & 17/10 & Stesura del verbale & Scrittori & Pivetta Federico\\
\hline
& & & Relatori & Giovanni Cristellon\\
\hline
\end{tabular}

\raggedright
\tableofcontents
\newpage
\section{17/10/2024}
\subsection{Durata e partecipanti}
\begin{itemize}
\item Ora: 16:00 - 16:25;
\item Partecipanti: 	
	\begin{itemize}
	\item Gusella Manuel;
	\item Cristellon Giovanni;
	\item Peruzzi Uncas;
	\item Piva Riccardo;
	\item Pivetta Federico.
	\end{itemize}
\item Azienda: Azzurro Digitale.
\item Piattaforma: Google meet (online)
\end{itemize}
\subsection{Oggetto}
Incontro con Vallini Giorgio, Daniele Martina, Gottardello Mattia e Boscaro Nicola dell'azienda Azzurro Digitale per chiarimenti sul Capitolato 9
\subsection{Sintesi}
Primo colloquio esterno con l'azienda Azzurro Digitale del Capitolato 9.\\
\vspace{2mm}
Sono state fatte domande specifiche sul progetto:
\begin{enumerate}
\item Considerando il tetto alle ore che possiamo dedicare al progetto imposto dal professore/committente è eventualmente possibile ritrattare i requisiti minimi e il MVP se necessario in futuro?
\item Quali caratteristiche minime devono essere incluse nel Proof of Concept? Basta lavorare con un unica fonte di dati o sono necessarie tutte le integrazioni (Confluence, Jira e Github)?
\item Nel capitolato vengono menzionati dei test automatizzati, è previsto un livello minimo di coverage perché consideriate il progetto valido?
\item Si parla di proporre l'utilizzo della metodologia agile durante lo svolgimento del progetto, volevamo sapere se ci verrà fornito un qualche tipo di aiuto da parte del vostro team? Se si, che aiuto nello specifico? 
\item Si parla di mettere a disposizione il proprio know-how tecnico e tecnologico, volevamo sapere se ci saranno dei brevi corsi/lezioni preparatorie già programmate dall'azienda? 
\item Nel documento ci sono delle proposte in merito alle tecnologie da utilizzare, siamo tenuti ad utilizzare quelle citate oppure possiamo utilizzarne anche altre?
\item La repo github citata nel documento avrà già a disposizione del materiale al suo interno o sarà completamente vuota? 
\item Che quantità di richieste in concorrenza vi aspettate indicativamente possa rispondere il bot?
\item Qual’è il volume atteso dei dati di cui sarà necessario fare RAG?
\end{enumerate}

Risposte:
\begin{enumerate}
\item Loro come proponenti non hanno inserito nessun vincolo rispetto alle ore da dedicare al progetto.
\item Per il PoC basta dimostrare la fattibilità del progetto limitandosi ad una fonte di informazione o a tutte e tre prese singolarmente, quindi senza doverle già usare assieme.
\item Per la parte di test coverage hanno richiesto di avvicinarci il 
più possibile al 100\%, tenendo conto che gli unici test scritti richiesti riguardano solo la parte di codice scritto da noi e non su quello che viene restituito dalle eventualli API utilizzate per il progetto.
\item Attraverso degli incontri settimanali, dove loro assumeranno il ruolo di clienti e noi di creatori del progetto, vogliono istruirci a mettere in atto la metodologia agile.
\item Loro hanno confermato che non ci saranno corsi di formazione.
\item Per quanto riguarda l'utilizzo delle tecnologie necessarie al raggiungimento del risultato richiesto sono disponibili ad aiutarci rispondendo ad eventuali domande, facendo presente che le tecnologie citate sono semplici consigli e non delle richieste.
\item Quella repo serve semplicemente per lavorarci sopra e molto probabilmente non conterrà nulla.
\item Per quanto riguarda le richieste in concorrenza, non si è parlato di questo tipo di problema, gia il fatto di avere un applicativo che funziona per una singola utenza è considerato un buon punto di partenza, mentre garantire un utilizzo parallelo viene sicuramente considerato un plus opzionale.
\item Il volume dei dati comprende tutte e tre le piattaforme citate nel documento, in linea di massima la mole dei dati varia in base ai permessi di visualizzare su ogni determinata piattaforma della singola utenza.
\end{enumerate}

%parte di firma
\vfill
\begin{minipage}{10cm}
Firma: \hrulefill \\
\vspace{2mm}
Data: \dotfill
\end{minipage}

\end{document}
