\documentclass[12pt]{article}

\usepackage[utf8]{inputenc}
\usepackage{geometry}
\usepackage{tabularx}
\usepackage{graphicx}

\graphicspath{{images/}}

%cambio misure della pagina
\geometry{a4paper,left=20mm,right=20mm,top=20mm}

\title{Verbale Interno del meeting in data 25/10/2024}
\date{A.A 2024/2025}

\renewcommand*\contentsname{Indice}
\begin{document}
%contenuti principali
\maketitle
\center 
\includegraphics[width=0.25\textwidth]{LogoUnipd}\\
\includegraphics[width=0.25\textwidth]{Sevenbitslogo}\\
sevenbits.swe.unipd@gmail.com\\
\vspace{2mm}

\textbf{Registro modifiche}\\
\vspace{2mm}
\begin{tabular}{|l|l|l|l|l|l|}
\hline
\textbf{Versione} & \textbf{Data} & \textbf{Descrizione} & \textbf{Ruolo} & \textbf{Componente} \\
\hline
1.0 & 25/10 & Stesura del verbale & Scrittori & Piva Riccardo\\
\hline
1.0 & 25/10 &  Stesura del verbale & Scrittori & Pivetta Federico\\
\hline
1.0 & 25/10 &  Redazione Verbale & Relatori & Peruzzi Uncas\\
\hline
\end{tabular}

\raggedright
\tableofcontents
\newpage
\section{25/10/2024}
\subsection{Durata e partecipanti}
\begin{itemize}
\item Ora: 12:00 - 12:25;
\item Partecipanti: 	
	\begin{itemize}
	\item Cristellon Giovanni;
	\item Peruzzi Uncas;
	\item Piva Riccardo;
	\item Pivetta Federico;
	\item Rubino Alfredo;
	\end{itemize}
\item Piattaforma: Google Meet (online)
\end{itemize}
\subsection{Oggetto}
Incontro con Gregorio Piccoli dell'azienda Zucchetti per chiarimenti sul capitolato 1.\\
\subsection{Sintesi}
Colloquio esterno con l'azienda Zucchetti del capitolato 1. Hanno fatto loro un'introduzione rispondendo alla maggior parte delle domande che avevamo preparato.\\
\vspace{2mm}
Segue il riassunto della loro introduzione:\\
\vspace{2mm}
Il proponente ha fatto presente che sono già in possesso di sistemi a cui poter porre delle domande e ricevere delle risposte, ma ne vuole verificare la pertitnenza; ogni volta che si cambia Large Language Model (anche detto "LLM") cambia la risposta fornita e questo è un problema.\newline 
Un esempio proposto riguardava frasi contenenti barcode che creavano criticità e andando a risolvere questo problema si erano presentate altre incongruenze su temi completamente diversi e scollegati, sebbene in precedenza collaudati con successo. La loro proposta è quella di confrontare le risposte, sicuramente mediante l'utilizzo di LLM e magari adoperare un'analisi che riguarda il conteggio dei caratteri o altre note tecniche di comparazione. Parte della richiesta consiste, in un secondo momento, nell'esporre un metodo per la presentazione dei risultati mediante l'utilizzo di grafici piuttosto che con diagrammi o tabelle.\\
\vspace{2mm}
Sono state fatte quindi delle domande conclusive per rispondere ai dubbi rimasti:
\begin{enumerate}
\item In che modalità verranno effettuati gli incontri?
\item Avete qualche consiglio in merito alle tecnologie che verranno adoperate durante lo svolgimento del progetto?
\end{enumerate}

\pagebreak
Risposte:
\begin{enumerate}
\item Gli incontri possono essere effettuati a nostra discrezione: in presenza, presso la sede principale (Padova centro) oppure a distanza, mediante delle piattaforme come ad esempio Google Meet.
\item Sicuramente è preferibile lo sviluppo di un applicativo web tramite tecnologie che sappiamo gia trattare per non incorrere nell'utilizzo di framework troppo convoluti. Un'altro consiglio riguarda la cura dei casi d'uso nella prima parte del progetto(RTB) e dell'adozione dei corretti pattern architetturali nella seconda parte(PB).
\end{enumerate}

%parte di firma
\vfill
\begin{minipage}{10cm}
Firma: \hrulefill \\
\vspace{2mm}
Data: \dotfill
\end{minipage}

\end{document}
